When looking at results from length-based assessments, we do need to be careful with conclusions, and consider the ecology and dynamics in all of the fisheries targeting the species under assessment, in all of the habitats utilized by all of the life stages of these species. Snappers for example, like most of our other target species, have pelagic eggs and larvae. The larval pelagic stage lasts for 4 to 6 weeks, when larvae are between 1 and 2 cm long. Eggs and larvae can be displaced over great distances, and pre-settlers actively swim in specific directions, and towards specific habitats, during this time. At the end of their pelagic migration, juvenile snappers settle on nursery grounds.

After settlement, juveniles of many species of snappers remain on nursery grounds for a period of several years, and then move to other areas joining sub-adults at specific habitats until they reach maturity, and eventually the adult population, usually at the deepest range of their distribution, on the slopes of the continental shelf.

It is important to realize that these fish can and will be targeted by various fisheries during all these phases of their life, with different gear types, in all the habitats that they occupy. In Indonesia that even includes the small pelagic pre-settlers, which are often found in catches by small meshed lift net boats using light attraction to catch (very) small pelagic fish. It should be clear that even if one fishery is shown to harvest mainly large adult fish of a specific species, this does not necessarily mean that the species as a whole is being fished sustainably across its entire range of life stages and habitats.

Relative abundance of specific size classes in one fishery may not change in the case of another fishery decimating juveniles, and in such case only a decline in the total numbers in the catch, or rather in the catch per unit of effort, will show that there is a problem somewhere. It is therefore recommended to keep track of catch per unit of effort by species (for target fleets and species) as an independent second source of information to back up conclusions from length-based assessments.
