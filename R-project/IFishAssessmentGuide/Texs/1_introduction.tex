This ``\textit{Length-Based Assessment Guide for Target Species in Indonesian Deep Slope ``Snapper'' Fisheries}'' was prepared for The Nature Conservancy's (TNC's) Indonesia Fisheries Conservation Program, in support of TNC's Deep Slope ``Snapper'' Fisheries Conservation Project. In the early stages of this program it was recognized that all stakeholders involved in these fisheries (including fishers, buyers, processors, traders, retailers, consumers, managers, NGO workers, government agencies, scientific and educational institutions, etc.) would benefit from the development of (1) a dedicated fish identification guide for the Indonesian deep slope fisheries, and (2) a guide that explains available tools for length-based assessment of the status and trends in these fisheries.

The TNC ``\textit{Top 100 Species Identification Guide for snappers, groupers and emperors in Indonesian deep slope fisheries}'' (Mous et al., 2017) was produced after taxonomic analysis of catches of deep slope drop line and bottom long line fisheries in Central and Eastern Indonesia between 2014 and 2017. This species ID guide for the deep slope fisheries is now available through the following link:

\textbf{CLICK: }\href{http://72.14.187.103:8080/ifish/pub/TNC_FishID.pdf}{Link to on-line E-Book Species ID Guide}

The species ID provides a first good inventory of target species in the deep slope fisheries in Indonesia, with clear images for each target species in the fisheries, together with correct scientific names and a range of common names used in the fisheries and in the trade. At the completion of this guide, it was also clear that additional imagery would be useful for correct identification on board, on landing sites and at monitoring stations. A separate ``Illustration Guide'' was prepared for this purpose, providing additional images ``trawled'' from the internet, selected for high quality and best possible presentation and colors of live or fresh animals.

\textbf{CLICK: }\href{http://72.14.187.103:8080/ifish/pub/DeepSlopeSpeciesIllustrationGuide.pdf}{Link to on-line E-Book Species Illustration  Guide}

The current length-based assessment guide includes simple length-based tools for the assessment of the target fisheries, as well as values of life history parameters for the main target species. This guide needs to be used together with the above mentioned species identification guide, and is meant to:

\begin{enumerate}[noitemsep,topsep=0pt,parsep=0pt,partopsep=0pt,leftmargin=*]
\item Provide up-to-date science-based information on species and fisheries.
\item Support species specific length-based assessments of data poor deep slope fisheries.
\item Define length-based life history characteristics, to enable length-based assessments.
\item Provide values by species for length-based life history parameters including:
	\begin{itemize}[noitemsep,topsep=0pt,parsep=0pt,partopsep=0pt,leftmargin=*]
	\item Maximum attainable total length (Lmax), based on records or estimates from images,
	\item Asymptotic length (Linf), defined as the mean length in a cohort fish, at the time when all individuals in that cohort have stopped growing,
	\item Length at which 50\% of individuals are mature (Lmat) and contributing to reproduction,
	\item Optimum length for harvesting (Lopt) of a species in terms of maximizing yield.
	\end{itemize}
\item Provide simple tools for length-based assessments, using the above length-based life history parameters in combination with catch length frequencies by species.
\item Stimulate discussion on management options among stakeholders and support management decision making based on length-based assessments.
\item Be readily accessible and comprehensible for all stakeholders mentioned above.
\end{enumerate}
