Adult stages of many target species in deep slope fisheries remain at well defined locations, at the edge of the continental shelf. These adult populations do not migrate to spawn or for other reasons. Deep water snappers and other deep water predators form feeding aggregations at edges of drop offs and canyons, seamounts and other highly predictable locations. This makes them extremely vulnerable to fishing, much more so than species which are spread out over the flat surface of the continental shelf. Overfishing can happen very quickly at those locations, much faster than the time it takes to collect and analyze data, formulate conclusions and management advice, and ultimately take management action. The locations where adult fish aggregate need to be managed very carefully. Access to these areas needs to be restricted to prevent overfishing. Some of these locations could effectively be set aside as ``No Take'' areas to protect spawning biomass.

Due to the spatial segregation between size groups in the populations, the fisheries can be size selective to some extent. Fishermen can take conscious decisions to target sub adults and juveniles and will do so normally when densities of larger mature animals on deep water fishing grounds have declined. As such, a policy among fish traders to buy and trade (or not to buy and trade) certain size classes can directly influence the sustainability of the fisheries when the buying behavior affects the behavior of fishers.

Stakeholders and managers should all prevent the targeting, selling, buying and trading of immature fish. Putting a premium on ``plate size fish'' for species which are not yet mature at such size, can be highly destructive to the stock as fishers are incentivized that way to target undersized fish. Incentives for fishers need to be geared towards catching mostly mature specimen of all target species. Fishers can decide to move on to a different location or different fishing depth when they find that they are fishing an aggregation of juvenile fish. They will do so only though if this makes immediate economic sense to them or if regulations on minimum sizes are in place and being enforced.

The choice of hook size also plays an important role in the selectivity of the fisheries, especially in combination with the choice of fishing location and target species. Small hooks with smaller baits, fished with thinner lines, in general catch smaller fish than large hooks with big baits fished with heavier lines. Fishing for deep water snapper in new locations often starts with large hooks and at fairly great depths. The main target species are the large deepwater snappers and within those species the larger specimen were targeted first. As adult populations at the deepest fishing grounds declined though, fishers explored different habitats, usually at somewhat shallower locations, with smaller hooks and smaller baits. This resulted in smaller specimen of the target species to become more dominant in the catch. This situation became worse when traders started to pay premium price for under-sized ``plate sized'' snappers.

Selectivity is influenced by a combination of hook size and fishing location (depth and habitat) but the species range is so great in the Indonesian deep slope fisheries, that management by species is impossible. Length-based assessments need to be carried out over the range of target species to find out what the patterns look like and management options need to be selected that take into account this multi species character of the fisheries. Management solutions are not straight forward and a precautionary approach necessitates wide-ranging management actions.
