Before applying the estimates for parameter values resulting from life history invariants in length based assessments of the fisheries, we verified those values first with available reliable values from comparable latitudes for the most important and most abundant species across the most important families in our fisheries. For the estimate of Lmax, the growing data set containing hundreds of thousands of CODRS images became an increasingly important tool for verification. For a number of species it was observed that sizes were obtained beyond what was previously reported at latitudes comparable to Indonesia and in these cases we could use our CODRS images directly to establish our estimates for Lmax. For a number of species we have also been able to find reliable verification of Lmax in the Australian ``North Coast Fish Identification Guide (Rome and Newman, 2010)'' as well as various other reliable sources, include some that were used also by Martinez-Andrade (2003).

For an increasing number of species, CODRS images are showing sizes in the catch at least up to or around Linf, or about 10\% below the estimated maximum attainable size. Linf is the mean size in the cohort when it stops growing and therefore a size more common in the population than the actual maximum obtainable size. We would expect, with very great sampling effort (a good part of the entire fisheries) to ultimately at least find specimen approaching Linf, whereas specimen at Lmax are extremely rare of course, especially in heavily fished situations. This pattern is indeed being observed from CODRS data analysis for most if not all important species in our fisheries.

For a few species we have not (yet) observed specimen close to their estimated Linf but these are not very important species in the deep slope fisheries. For these species we are confident that they grow much larger than what we see in the deep slope catch but we do not have ready explanations on why very large fish are missing from the sample. For example for the cobia and the black jewfish this is the case and for these species additional checks will follow.

The most important and most abundant species in the deep slope drop- and long-line fisheries for snappers, groupers and emperors in Indonesia is the Goldband Snapper or Goldband Jobfish, \textit{Pristipomoides multidens} (Table 3.1). This species can reach an estimated maximum size of up to 100 cm (Total Length) in Indonesian waters and this is verified as the maximum size for the same species in neighboring Northern Australian waters by Rome and Newman (2010).

With our method of estimating the asymptotic size for our target species this means an estimated Linf of 90 cm for \textit{Pristipomoides multidens} in Indonesia. This would be the mean size in the cohort if it was left to grow infinitely. With a very large sample size we would expect to ultimately find a specimen approaching Linf and indeed the largest specimen photographed in our CODRS program was 90 cm.

Using 0.59*Linf as the estimator for Lmat in deep water Lutjanidae (Newman et al, 2016), we could estimate the length at 50\% maturity for \textit{Pristipomoides multidens} to be reached at 53 cm total length. This converts to around 46 cm fork length for Lmat in this species and these values are not significantly different (almost the same, 55 cm TL and 47 cm FL) from what is reported for this species for Northern Australian waters. This is another very important verification of the approach.

\clearpage
\newpage

%
% This table keep generate by manual
%
\setlength{\tabcolsep}{5pt}
\captionsetup{width=1\textwidth, justification=centering}
\setlength{\LTpost}{0pt}
% latex table generated in R 3.2.2 by xtable 1.7-4 package
% Mon Jan 23 07:56:53 2017
{\small
\begin{longtable}{ccccccccc}
\caption{Ranking and Sample Sizes (January 2017) of 50 Most Abundant Species \\in Indonesian Deep Water Hook-And-Line Fisheries} \\ 
  \hline
Rank & ID\# & Species & Nsample & Lx-codrs & Lmax & Linf & Lopt & Lm50 \\ 
  \hline
1 & 7 & Pristipomoides multidens & 63893 & 90 & 100 & 90 & 71 & 53 \\ 
  2 & 8 & Pristipomoides typus & 30362 & 82 & 85 & 77 & 60 & 45 \\ 
  3 & 45 & Epinephelus areolatus & 25381 & 49 & 50 & 45 & 28 & 21 \\ 
  4 & 9 & Pristipomoides filamentosus & 23751 & 86 & 90 & 81 & 64 & 48 \\ 
  5 & 1 & Aphareus rutilans & 14083 & 112 & 125 & 113 & 88 & 66 \\ 
  6 & 34 & Paracaesio kusakarii & 13597 & 80 & 85 & 77 & 60 & 45 \\ 
  7 & 18 & Lutjanus malabaricus & 12746 & 90 & 100 & 90 & 71 & 53 \\ 
  8 & 4 & Etelis sp. & 10727 & 120 & 130 & 117 & 92 & 69 \\ 
  9 & 10 & Pristipomoides sieboldii & 10008 & 55 & 60 & 54 & 42 & 32 \\ 
  10 & 23 & Pinjalo lewisi & 8916 & 51 & 55 & 50 & 39 & 29 \\ 
  11 & 22 & Lutjanus erythropterus & 7724 & 69 & 70 & 63 & 49 & 37 \\ 
  12 & 6 & Etelis coruscans & 7437 & 120 & 130 & 117 & 92 & 69 \\ 
  13 & 20 & Lutjanus timorensis & 6440 & 60 & 60 & 54 & 42 & 32 \\ 
  14 & 71 & Gymnocranius grandoculis & 5758 & 74 & 80 & 72 & 48 & 36 \\ 
  15 & 27 & Lutjanus vitta & 5134 & 43 & 45 & 41 & 32 & 24 \\ 
  16 & 70 & Wattsia mossambica & 4345 & 59 & 60 & 54 & 36 & 27 \\ 
  17 & 35 & Paracaesio stonei & 4086 & 67 & 70 & 63 & 49 & 37 \\ 
  18 & 19 & Lutjanus sebae & 3549 & 96 & 100 & 90 & 71 & 53 \\ 
  19 & 5 & Etelis radiosus & 2274 & 103 & 105 & 95 & 74 & 56 \\ 
  20 & 43 & Epinephelus morrhua & 2049 & 71 & 75 & 68 & 41 & 31 \\ 
  21 & 32 & Paracaesio gonzalesi & 1903 & 51 & 55 & 50 & 39 & 29 \\ 
  22 & 39 & Cephalopholis sonnerati & 1861 & 52 & 55 & 50 & 30 & 23 \\ 
  23 & 51 & Epinephelus chlorostigma & 1842 & 63 & 75 & 68 & 41 & 31 \\ 
  24 & 2 & Aprion virescens & 1782 & 107 & 110 & 99 & 78 & 58 \\ 
  25 & 88 & Glaucosoma buergeri & 1579 & 67 & 70 & 63 & 42 & 32 \\ 
  26 & 84 & Seriola rivoliana & 1456 & 120 & 135 & 122 & 81 & 61 \\ 
  27 & 16 & Lutjanus argentimaculatus & 1436 & 95 & 100 & 90 & 71 & 53 \\ 
  28 & 85 & Erythrocles schlegelii & 1394 & 90 & 90 & 81 & 54 & 41 \\ 
  29 & 68 & Lethrinus amboinensis & 1175 & 56 & 60 & 54 & 36 & 27 \\ 
  30 & 33 & Paracaesio xanthura & 1099 & 49 & 50 & 45 & 35 & 27 \\ 
  31 & 17 & Lutjanus bohar & 1099 & 88 & 90 & 81 & 64 & 48 \\ 
  32 & 67 & Lethrinus olivaceus & 1059 & 97 & 100 & 90 & 60 & 45 \\ 
  33 & 21 & Lutjanus gibbus & 954 & 48 & 50 & 45 & 35 & 27 \\ 
  34 & 90 & Diagramma pictum & 900 & 81 & 85 & 77 & 51 & 38 \\ 
  35 & 76 & Carangoides chrysophrys & 822 & 80 & 80 & 72 & 48 & 36 \\ 
  36 & 80 & Caranx sexfasciatus & 808 & 82 & 85 & 77 & 51 & 38 \\ 
  37 & 65 & Lethrinus lentjan & 793 & 50 & 55 & 50 & 33 & 25 \\ 
  38 & 58 & Epinephelus amblycephalus & 787 & 78 & 80 & 72 & 44 & 33 \\ 
  39 & 72 & Gymnocranius griseus & 783 & 44 & 45 & 41 & 27 & 20 \\ 
  40 & 31 & Symphorus nematophorus & 775 & 97 & 100 & 90 & 71 & 53 \\ 
  41 & 30 & Lipocheilus carnolabrum & 764 & 73 & 75 & 68 & 53 & 40 \\ 
  42 & 91 & Cookeolus japonicus & 744 & 62 & 65 & 59 & 39 & 29 \\ 
  43 & 87 & Dentex carpenteri & 717 & 42 & 45 & 41 & 27 & 20 \\ 
  44 & 86 & Argyrops spinifer & 709 & 50 & 55 & 50 & 33 & 25 \\ 
  45 & 53 & Epinephelus heniochus & 623 & 56 & 60 & 54 & 33 & 25 \\ 
  46 & 82 & Elagatis bipinnulata & 597 & 104 & 110 & 99 & 66 & 50 \\ 
  47 & 77 & Carangoides gymnostethus & 533 & 84 & 90 & 81 & 54 & 41 \\ 
  48 & 14 & Pristipomoides flavipinnis & 515 & 54 & 60 & 54 & 42 & 32 \\ 
  49 & 61 & Plectropomus leopardus & 506 & 72 & 80 & 72 & 44 & 33 \\ 
  50 & 81 & Caranx tille & 502 & 88 & 90 & 81 & 54 & 41 \\ 
   \hline
\hline
\end{longtable}
}
 
\noindent{\small{\textbf{Nsample} is total sample including SWMS and CODRS data (mostly CODRS).
\textbf{Lx-codrs} = Largest specimen with verifiable ID and size from CODRS photo.
\textbf{Lmax} = maximum attainable total length at Indonesian lattitudes.
\textbf{Linf} = 0.9 * Lmax (with 10\% dispersion around mean size in cohort).
\textbf{Lm50} = Size at 50\% maturity.
\textbf{Lm50} = 0.59 * Linf for deep water lutjanidae (Newman et al., 2016).
\textbf{Lm50} = 0.46 * Linf for deep water Epinephelidae (Newman et al., 2016).
\textbf{Lm50} = 0.5 * Linf for Other Species (pooled literature).
\textbf{Lopt} = 1.33 * Lmat for range of demersals (Cope and Punt, 2009). All sizes in Total Length.}}
\clearpage
\newpage