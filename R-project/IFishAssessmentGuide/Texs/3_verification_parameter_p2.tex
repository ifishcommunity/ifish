There is less information available on the co-occurring and very similar Sharptooth Jobfish, \textit{Pristipomoides typus}, which is the second most important species in the fisheries. This species is very similar to the Goldband Snapper, often mixed in the trade, and there is no reason to assume different values for life history invariable values. Rome and Newman (2010) report a maximum size of 80 cm from Northern Australia, but slightly larger fish were encountered in our CODRS program. Therewith a highly reliable estimate of 85 cm for the maximum size followed for this species. Resulting values for Linf, Lopt and Lmat followed from the above explained approach.

The fourth most abundant species in the combined catches from drop- and long-line fisheries in Indonesia is the Rosy Jobfish or ``Opakapaka'', \textit{Pristipomoides filamentosus}. Reported to grow up to 80 cm in Northern Australia, but specimen encountered up to 86 cm total length in Indonesia, leading to an estimated maximum total length in our waters of about 90 cm. This then leads to an estimate for Linf of 81 cm and Lmat of 48 cm total length which is well within the range of values reported in the literature for this species from comparable latitudes (reviewed a.o. also by Martinez Andrade in 2003 and by Newman et al. in 2016).

Another important species of Pristipomoides in the deep drop-line and bottom long-line snapper fisheries is the Lavender Jobfish or Lavender Snapper or ``Kalekale'', \textit{Pristipomoides sieboldii}. This species is ranked number 9 on the list of most abundant species in our target fisheries. The Lavender Jobfish grows to a maximum total length of 60 cm in Indonesian waters which is the same as reported for North Australian waters (Rome and Newman, 2010). The largest specimen recorded in our CODRS program was 55 cm in total length, at the time of writing of this report, which is just over the estimated asymptotic length of 54 cm for this species, fully in line with what we would expect to find from a very large sample size (about 10,000 specimen at this time). \textit{Pristipomoides sieboldii} was reported to mature (50\%) at 29 cm Fork Length, (Demartini and Lau, 1999), equivalent to about 33 cm Total Length. This aligns perfectly again with the estimate of 32 cm Total Length following the application of the life history invariant approach.

There are four more species of Pristipomoides in the Top 100 species distribution covering the deep slope catches by drop-line and long-line vessels in Indonesia. These species are \textit{Pristipomoides flavipinnis}, \textit{P. argyrogrammicus}, \textit{P. zonatus},  and \textit{P. auricilla}. These species are not abundant or important in the catch and they are not as well studied as the more abundant species mentioned above. But still we could confidently estimate their life history parameters based on what we know about the maximum attainable size and the good fit of the model for all the other species above.

The Areolate Grouper or Squaretail Rockcod, \textit{Epinephelus areolatus}, is the most important species of grouper (Epinephelidae) in the deep slope fisheries in Indonesia. And it is also the third most abundant species in these fisheries overall. Also for the groupers we found that the parameter values estimated with the life history invariant approach are aligned very well with reliable values for important and well researched species. \textit{Epinephelus areolatus} is widely reported to grow to a maximum size of 50 cm and this value is also verified by our CODRS observations which show the largest fish in the catch to measure 49 cm at the time of writing of this report. With above explained approach of estimating Linf and Lmat this results in an estimated length at 50\% female maturity for this species of 21 cm total length which also fully aligns with available literature values.
\clearpage
\newpage
For some species of groupers the estimated maximum attainable size as reported in various literature sources does not seem to be supported by any verifiable data, such as for example for \textit{Epinephelus morrhua}, the 2nd most important grouper in the deep slope fisheries in Indonesia and number 20 in the Top 100 of all species. Excessively large values for Lmax appear in the literature for this species, possibly due to misidentifications. We found an estimate of 75 cm to be closest to what we could verify, and this number was confirmed in a review published by SPC in 1997 (Shakeel and Hudha, 1997).

The fifth most abundant species in the deep slope drop-line and bottom long-line fisheries is the Rusty Jobfish or ``Lehi'', \textit{Aphareus rutilans}, a large deep water snapper which grows to about 125 cm Total Length in Indonesian waters, based on available evidence that was verifiable from images. Martinez-Andrade (2003) reported a maximum attainable size of 126 cm Total Length as the mean value from a number of studies, which is well aligned with our estimate for Indonesia.  Larger maximum sizes (up to 150 cm) have been reported elsewhere, but such sizes could not be verified for our general latitude. The asymptotic length for this species as estimated from 0.9*Lmax is 113 cm for Indonesian waters, which is a size that could be verified from the largest specimen recorded from the CODRS program which measured 112 cm Total Length. Our estimate for length at 50\% maturity in this species is 66 cm and this again aligns completely with the 66 cm as reported by Martinez-Andrade (2003).

The deep water snappers of the genus Etelis are very important species in our target fisheries, especially in catches from the deepest part of the depth range that is exploited by bottom long-line and drop line fishers in Eastern Indonesia. The Ruby Snapper or ``Ehu'', \textit{Etelis carbunculus}, was recently discovered to be a separate small species that until a few years ago was mixed with a very similar but much larger species, the Giant Ruby Snapper, \textit{Etelis sp.}, which is yet to be scientifically named although it has been exploited throughout the Pacific for many years.

The ``true'' \textit{Etelis carbunculus} grows to a maximum size of about 60 cm and its asymptotic length is about 54 cm. The largest specimen encountered in our CODRS program to date was 55 cm which aligns again very well with estimates for Linf and Lmax. Estimated total length at 50\% maturity is 32 cm using the life history invariable approach. Literature data from Hawaii indicate a size at 50\% maturity of 28 cm Fork Length (Demartini and Lau, 1999) for this species, which equals about 31 cm Total Length and therefore aligns again perfectly. Newman et al. (2016) refer to a number of other studies showing a size at maturity of 30 cm Fork Length for this species from various locations, also right close to our result from the life history invariant approach.

Whereas the Ruby Snapper, \textit{Etelis carbunculus}, actually turned out to be a fairly uncommon species in Indonesian deep slope catches, the Giant Ruby Snapper, \textit{Etelis sp.}, although yet to be named scientifically, is the 8th most abundant species in the catch. And due to its size, it is more important yet in total weight. Due to the recent developments in the taxonomy in these two species of Etelis, there are no direct literature references life history parameter values to compare with for \textit{Etelis sp.}, but we can look at values previously reported for \textit{Etelis carbunculus} which clearly do not belong to that species if they are outside the maximum size it can reach.

The Giant Ruby Snapper, \textit{Etelis sp.}, can reach a maximum Total Length of up to 130 cm, as was reported for neighboring North Australian waters, where it was previously misidentified as \textit{Etelis carbunculus} (Rome and Newman, 2010). The estimated asymptotic length for this species is 117 cm, which could be verified from the largest specimen recorded from our CODRS program, measuring 120 cm. Application of the life history invariant approach resulted in an estimate of 69 cm for the Total Length at 50\% maturity for this species. Polovina and Shomura (1990) reported 61 cm Fork Length as the size at maturity for \textit{Etelis carbunculus} which we now know must have been \textit{Etelis sp.}. This 61 cm Fork Length converts to 67 cm Total Length and aligns very closely again with the estimate we obtained from the life history invariant approach.

The other 2 Etelis species in the Indonesian deep slope fisheries are the Flame Snapper or ``Onaga'', \textit{Etelis coruscans}, and the Pale Snapper, \textit{Etelis radiosus}. The Flame Snapper, \textit{Etelis coruscans} is an important and well known target species, ranked number 12 in order of abundance in the catch. The Pale Snapper, \textit{Etelis radiosus}, is less abundant and was only more recently described. Little research has been done on \textit{Etelis radiosus}, which is ranked number 19 in the Top 100 for our fisheries and which has previously been mixed with other Etelis species both in the trade and in fisheries research. Fortunately we can confidently apply life history invariant approach to estimate parameter values from its maximum size of 105 cm, quite a bit larger than what was previously reported for this species but verified by CODRS images of specimen reaching up to 103 cm in the Indonesian deep slope catches.

Because of great variance in relative length of the long ``Flame'' upper lobe of the tail fin in \textit{Etelis coruscans}, we decided to measure total length up to the tip of the lower lobe, which has a much more stable relative length. This may cause some discrepancy with values for total length reported elsewhere in the literature, although Fork Length is more often used and we can still convert that for reliable comparisons. The maximum size attained in Indonesia for \textit{Etelis coruscans} is about 130 cm measured to the tip of the lower lobe of the tail. This leads to an estimated 117 cm asymptotic length which could be verified with CODRS data showing specimen up to 120 cm in the catch. The estimated length at 50\% maturity for this species is 69 cm Total Length from the life history invariant approach. This estimate falls well within a range of values reported in the literature for this species (Everson et al., 1989; Martinez-Andrade, 2003, Newman et al., 2016).

A number of important Lutjanus species have been traded as a general group combined under ``Red Snappers'' since many years. This includes includes \textit{Lutjanus malabaricus}, number 7 in the Top 100 of our fisheries and \textit{L. erythropterus}, \textit{L. timorensis}, and \textit{L. sebae}, which take positions, 11, 13 and 18 respectively. Several other species of Lutjanus are included in the Top 100, including one more at position number 15, \textit{Lutjanus vitta}, which is not red in color and which is (perhaps therefore) less preferred in the trade. Less abundant Lutjanus species which are often mixed into ``Red Snapper'' products, especially into filleted products, include \textit{L. argentimalus} (27), \textit{L. bohar} (31), \textit{L. gibbus} (33), \textit{L. bitaeniatus} (57), \textit{L. lemniscatus} (61), and \textit{L. russelli} (71).

We have focused on the most important species to verify our life history parameter value estimates. As reported also for North Australian and some other tropical waters, \textit{Lutjanus malabaricus} can reach a maximum total length of 100 cm (e.g. Rome and Newman, 2010). Our estimate of asymptotic length of 90 cm was confirmed by CODRS images of specimen in the catch up to 90 cm total length. A fairly wide range of values is reported in the literature on length at 50\% maturity. Our estimate of 53 cm fits well within the reported range and is very close to estimates reported from Vanuatu as well as from Northern Australia and the Arafura Sea (Martinez-Andrade, 2003).

The Crimson Snapper, \textit{Lutjanus erythropterus}, is next in line for the Lutjanus species, at number 11 in the Top 100 most abundant species in catches by our fisheries. This red snapper can reach a maximum total length of 70 cm in Indonesian waters, although smaller maximum sizes are reported from Australia. Our CODRS data confirm \textit{Lutjanus erythropterus} in the catch of up to 69 cm. The estimated asymptotic length is 63 cm and size at 50\% maturity for this snapper is 37 cm. This estimate for size at maturity is in line with Lmat sizes reported from the Great Barrier Reef in Australia (Martinez-Andrade, 2003).

The highly priced \textit{Lutjanus erythropterus} is often mixed in the ``red snapper'' trade with the poorly known and only recently (1987) described but highly abundant Slender Pinjalo (or ``Red Pinjalo''), \textit{Pinjalo lewisi}, which is number 10 in our Top 100, an amazingly high ranking for a species that was described only 30 years ago. Additional mixing takes place with the larger and better known Pinjalo Snapper, \textit{Pinjalo pinjalo}, which is however much less abundant in catches by our target fisheries. \textit{Pinjalo lewisi} grows to a maximum length of 55 cm in Indonesian waters, slightly larger than commonly reported in the literature. This can be verified from CODRS images showing specimen up to 52 cm in total length. Little is known about this abundant snapper species, but size at maturity is commonly assumed to follow the same trend as observed for other species of Lutjanidae.

Other important ``red snappers'' include the Timor Snapper, \textit{Lutjanus timorensis}, which is number 13 in the Top 100 list, and the Red Emperor, \textit{Lutjanus sebae}, which is ranked number 18. Both these species are commonly mixed with other red snappers in the trade but could in fact be separated quite easily based on visible external differences. The Timor Snapper and Red Emperor grow to maximum total lengths of 60 cm and 100 cm respectively. These maximum sizes are verified for Northern Australian waters (Rome and Newman, 2010) as well as from other studies. Estimated asymptotic lengths of 54 cm and 90 cm respectively for \textit{Lutjanus timorensis} and \textit{Lutjanus sebae} are also well within the range of available literature values (e.g. Martinez-Andrade, 2003) and further verification was obtained through CODRS data and images showing specimen in the catch for these 2 species up to 60 cm and 96 cm respectively. Literature values for length at 50\% maturity for \textit{Lutjanus sebae} mostly range between 48 cm and 55 cm total length for our latitudes (e.g. Martinez-Andrade, 2003), placing our estimate of 53 cm for this species well within this range. Information on size at maturity for \textit{Lutjanus timorensis} is scarce, but limited information shows mature individuals of both sexes to be common at sizes above 28 cm Fork Length which aligns very well with our estimate of 32 cm Total Length for Lmat in this species.

Apart from the high priced ``red snappers'' in the Top 20 most abundant list, there are a number of poorly known snapper species that are underappreciated in the trade, possibly due to their dull colors, although eating quality is very good for these fish. These very common but not well known species include the Brownstripe Snapper, \textit{Lutjanus vitta}, at number 15, the highly abundant Saddle Back Snapper, \textit{Paracaesio kusakarii}, at number 6 and its closely related cousin, the very similar looking Cocoa Snapper, only recently (1983) described as \textit{Paracaesio stonei}, at number 17. The maximum attainable lengths in Indonesian waters for these 3 species are 45 cm, 85 cm and 70 cm respectively. Somewhat larger than commonly reported in the literature (although \textit{P. kusakarii} and \textit{P. Stonei} have been mixed up in some reports) but clearly verified from CODRS data and images showing specimen in the catch up to 42 cm, 80 cm and 67 cm respectively for these 3 snappers. Our data provide strong support for estimates of asymptotic lengths of 41 cm, 77 cm and 63 cm respectively for these 3 poorly known but important species. Our estimate of Lmat for the Brownstripe Snapper (24 cm) is the same as what was reported from Malaysia while similar sizes (23 cm) were reported from the Philippines and Northern Australia (Martinez-Andrade, 2003). Very little is known about maturation in the Paracaesio species, but FishBase quotes a report on Lmat for \textit{P. stonei} at 40 cm, which is just above our estimate of 37 cm for this species and just below our estimate of 45 cm for \textit{P. kusakarii}. It is unclear however if mixing of these two very similar species may have occurred.

The most abundant species of emperors in our target fisheries are high quality fish which are however not very well known as specific species in the trade and usually are classified generally as ``emperor'' or sometimes even as ``white snapper''. The Blue-Lined Emperor, \textit{Gymnocranius grandoculis}, is the most abundant and important emperor at number 14 on the list, while the Mozambique Large-Eye Bream, \textit{Wattsia mossambica} is placed number 16. The maximum attainable sizes for these two emperors are 80 cm and 60 cm total length respectively. These maximum sizes are confirmed also for North Australian waters (Rome and Newman, 2010), with the note that Wattsia mossambica is assumed to grow up to just 55 cm there while our CODRS data show that somewhat larger sizes are attained in Indonesia. CODRS data include specimen in the catch of \textit{Gymnocranius grandoculis} and \textit{Wattsia mossambica} of up to 74 cm and 59 cm respectively, strongly supporting our estimates of 72 cm and 54 cm for Linf for these 2 species respectively.

The above described species cover the Top 20 most abundant species in the catch of our fisheries and some more. These species also cover close to 90\% of the catch and therewith we have been able to verify the validity of the life history invariant approach for the major part of the catch. This was possible through cross-checks with reliable literature and though validation of Lmax and Linf estimates directly from our own CODRS images. These CODRS images have shown maximum sizes reached by target species sometimes to be in excess of what has previously been reported.
