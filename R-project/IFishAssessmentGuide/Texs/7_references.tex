To develop the guidelines and findings in this document, we used a wide variety of sources: scientific articles from peer-reviewed journals (especially meta-analysis, but also species specific), project reports, presentations and other ``grey literature'', technical reports from various institutions, websites of research and other institutions and fishing companies, and even blogs and comments posted by recreational and other fishers. Sources sometimes contradicted each other, and we found this was often caused by mistakes in species identification or by different interpretation of technical terms or by analyses that were based on incomplete or otherwise inadequate data sets.

Carefully documenting all these inconsistencies, and providing a complete list of all references would have slowed the process down considerably, and it would have made this document unwieldy and inaccessible to all but very determined readers. Hence, we decided to present our findings and guidelines without a meticulous review of corroborating or contradicting sources, and in the list of references below we only present a small subset of the sources we used. Whereas we feel that the guidelines and findings we present here will enable sound fishery management, we encourage readers to triangulate our guidelines with those from other sources. We also encourage users to use our guidelines and findings mainly as a starting point for discussions on fisheries impact and status, to be refined whenever additional information becomes available or is deemed necessary.

\section{Selected Sources Referenced in The Text}
\begin{itemize}[noitemsep,topsep=0pt,parsep=0pt,partopsep=0pt]
\item Binohlan, C. and R. Froese. 2009. Empirical equations for estimating maximum length from length at first maturity. Journal of Applied Ichthyology, 25(5): 611-613.
\item Cope, J.M., and A.E. Punt. 2009. Length-based reference points for data-limited situations: applications and restrictions. Mar. Coast. Fish. Dyn. Mgmt. Ecosys. Sci. 1:169-186.
\item Everson, A.R., Williams, H.A., and B.M Ito, 1989. Maturation and Reproduction in Two Hawaiian Eteline Snappers, Uku, Aprion virescens, and Onaga, Etelis coruscans. Fish. Bull., 87: 877--888.
\item Martinez-Andrade F., 2003. A comparison of life histories and ecological aspects among snappers (Pisces: lutjanidae). Dissertation http://etd.lsu.edu/docs/available/etd-1113103-230518/unrestricted/Martinez-Andrade\_dis.pdf
\item Mous, P.J., Gede, W., Pramana, R., Wibisono E. and J.S. Pet, 2017. 100 Species identification guide for deepwater hook-and-line fisheries targeting snappers, groupers and emperors in Indonesia. The Nature Conservancy Indonesia Fisheries Conservation Program, Denpasar, Bali, Indonesia.
\item Nadon, M.O. and J.S. Ault, 2016. A stepwise stochastic simulation approach to estimate life history parameters for data-poor fisheries. Can. J. Fish. Aquat. Sci. 73: 1-11.
\item Newman, S.J., Williams, A.J., Wakefield, C.B., Nicol, S.J., Taylor, B.M. and J.M. O'Malley, 2016.    Review of the life history characteristics, ecology and fisheries for deep-water tropical demersal fish in the Indo-Pacific region. Reviews in Fish Biology and Fisheries 26(3): 537-562.
\item Rome, B.M. and S.J. Newman, 2010. North Coast Fish Identification Guide. Fisheries Occasional Publication No. 80. Dept of Fisheries, Perth, Western Australia.
\item Shakeel, H. and A. Hudha, 1997. Exploitation of reef resources, grouper and other food fishes in the Maldives. SPC Live Reef Fish Information Bulletin \#2, May 1997.

\end{itemize}