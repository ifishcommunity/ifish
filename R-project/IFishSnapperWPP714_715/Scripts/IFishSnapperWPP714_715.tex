\documentclass{report}\usepackage[]{graphicx}\usepackage[]{color}
%% maxwidth is the original width if it is less than linewidth
%% otherwise use linewidth (to make sure the graphics do not exceed the margin)
\makeatletter
\def\maxwidth{ %
  \ifdim\Gin@nat@width>\linewidth
    \linewidth
  \else
    \Gin@nat@width
  \fi
}
\makeatother

\definecolor{fgcolor}{rgb}{0.345, 0.345, 0.345}
\newcommand{\hlnum}[1]{\textcolor[rgb]{0.686,0.059,0.569}{#1}}%
\newcommand{\hlstr}[1]{\textcolor[rgb]{0.192,0.494,0.8}{#1}}%
\newcommand{\hlcom}[1]{\textcolor[rgb]{0.678,0.584,0.686}{\textit{#1}}}%
\newcommand{\hlopt}[1]{\textcolor[rgb]{0,0,0}{#1}}%
\newcommand{\hlstd}[1]{\textcolor[rgb]{0.345,0.345,0.345}{#1}}%
\newcommand{\hlkwa}[1]{\textcolor[rgb]{0.161,0.373,0.58}{\textbf{#1}}}%
\newcommand{\hlkwb}[1]{\textcolor[rgb]{0.69,0.353,0.396}{#1}}%
\newcommand{\hlkwc}[1]{\textcolor[rgb]{0.333,0.667,0.333}{#1}}%
\newcommand{\hlkwd}[1]{\textcolor[rgb]{0.737,0.353,0.396}{\textbf{#1}}}%

\usepackage{framed}
\makeatletter
\newenvironment{kframe}{%
 \def\at@end@of@kframe{}%
 \ifinner\ifhmode%
  \def\at@end@of@kframe{\end{minipage}}%
  \begin{minipage}{\columnwidth}%
 \fi\fi%
 \def\FrameCommand##1{\hskip\@totalleftmargin \hskip-\fboxsep
 \colorbox{shadecolor}{##1}\hskip-\fboxsep
     % There is no \\@totalrightmargin, so:
     \hskip-\linewidth \hskip-\@totalleftmargin \hskip\columnwidth}%
 \MakeFramed {\advance\hsize-\width
   \@totalleftmargin\z@ \linewidth\hsize
   \@setminipage}}%
 {\par\unskip\endMakeFramed%
 \at@end@of@kframe}
\makeatother

\definecolor{shadecolor}{rgb}{.97, .97, .97}
\definecolor{messagecolor}{rgb}{0, 0, 0}
\definecolor{warningcolor}{rgb}{1, 0, 1}
\definecolor{errorcolor}{rgb}{1, 0, 0}
\newenvironment{knitrout}{}{} % an empty environment to be redefined in TeX

\usepackage{alltt}
\usepackage{geometry}
\geometry{a4paper,textwidth=15.92cm,textheight=24.62cm}
\usepackage{verbatim}
\usepackage{rotating}
\usepackage{lscape}
\usepackage{hyperref}
\usepackage[T1]{fontenc}
%\usepackage[nodayofweek]{datetime}
\usepackage[ddmmyy]{datetime}
\usepackage{enumitem}
\usepackage{setspace}
\usepackage{graphicx,wrapfig,lipsum}
\usepackage{fancyhdr}

\makeatletter
\newcommand{\verbatimfont}[1]{\def\verbatim@font{#1}}%
\makeatother

% following to define a new section style that takes less space
\makeatletter % make "at" ("@") letter, important for the code that follows
\renewcommand\chapter{\@startsection%
{chapter}{1}{0pt}%name, level, indent
{-\baselineskip}%beforeskip
{0.2\baselineskip}%afterskip
{\raggedright\bf}}%
\makeatother

\makeatletter
\renewcommand\section{\@startsection%
{section}{2}{0pt}%name, level, indent
{-\baselineskip}%beforeskip
{0.2\baselineskip}%afterskip
{\raggedright\bf}}%
\makeatother

\makeatletter
\setlength{\@fptop}{0pt}
\makeatother

\setlength{\tabcolsep}{3pt}
\setlength{\parskip}{1em}
\renewcommand*\contentsname{Table of contents}

\newcommand{\MONTH}{%
  \ifcase\the\month
  \or January
  \or February
  \or March
  \or April
  \or May
  \or June
  \or July
  \or August
  \or September
  \or October
  \or November
  \or December
  \fi}

\renewcommand{\dateseparator}{}

%\definecolor{red}{rgb}{0.8,0,0}
%\definecolor{yellow}{rgb}{0.8,0.8,0}
\definecolor{green}{rgb}{0,0.8,0}
\definecolor{gray}{rgb}{0.5,0.5,0.5}

\pagestyle{fancy}
\fancyhf{}
\fancyhead[L]{\textcolor{gray}{THE NATURE CONSERVANCY INDONESIA FISHERIES CONSERVATION PROGRAM \\ AR\_714715\_{\today}}}
\rfoot{\thepage}
\IfFileExists{upquote.sty}{\usepackage{upquote}}{}
\begin{document}
% use "echo=False" to remove R code from the output


\newgeometry{left=1cm,top=1cm,right=1cm,bottom=1cm}
\begin{titlepage}
\begin{flushleft}
	\textsf{Report Code: AR\_WPP573\_{\today}}
\end{flushleft}

\vspace*{2cm}
\begin{flushright}
        {\Large\textsf{Length-Based Assessment of Data-Poor Multi-Species Deep Slope Fisheries in 
        \\[0.2cm] Fisheries Management Area (WPP) 573 in South East Indonesia}}\\[0.2cm]
        \rule{\linewidth}{0.5mm}
        \textsf{DRAFT - NOT FOR DISTRIBUTION. TNC-IFCP Technical Paper}\\[2cm]
        \textsf{Peter J. Mous, Jos S. Pet\\[1cm]
        {\MakeUppercase{\MONTH}} {\the\day}, {\the\year}}
\end{flushright}

\begin{center}
\includegraphics[width=.9\linewidth]{/root/R-project/IFishSnapperWPP573/Images/sizing.jpg}
\end{center}

\vfill

\noindent
\begin{minipage}[b]{\linewidth}
\noindent
\centering
$\vcenter{\hbox{\includegraphics[width=.3\linewidth]{/root/R-project/IFishSnapperWPP573/Images/usaid.png}}}$
\hfill
$\vcenter{\hbox{\includegraphics[width=.3\linewidth]{/root/R-project/IFishSnapperWPP573/Images/tnc.png}}}$
\hfill
$\vcenter{\hbox{\includegraphics[width=.3\linewidth]{/root/R-project/IFishSnapperWPP573/Images/pnci.png}}}$
\end{minipage}
\end{titlepage}
\restoregeometry

\vspace*{\fill}
\begin{sffamily}
\noindent\large For inquiries, please contact Dr. Peter Mous at pmous@tnc.org or Dr. Jos Pet at pet.jos@gmail.com\\
\end{sffamily}

\noindent
\fbox{\begin{minipage}[b]{\linewidth}
\begin{sffamily}
\begin{spacing}{0.4}
\textbf{The Nature Conservancy Indonesia Fisheries Conservation Program}\\[0.1cm]
Jl. Pura Segara, Pelabuhan Raya Benoa\\
Denpasar 3012\\
Bali, Indonesia\\
Ph. +62-361-244524, fax +62-361-244532\\[1cm]

\textbf{People and Nature Consulting International}\\[0.1cm]
Grahalia Tiying Gading 18, Suite 2\\
Jalan Tukad Pancoran, Panjer\\
Denpasar 80225
Bali, Indonesia\\
Ph. +62-361-257246\\
\end{spacing}
\end{sffamily}
\end{minipage}}
\clearpage

\tableofcontents

\vfill

\large

\chapter{Introduction}
This ``\textit{Length-Based Assessment Guide for Target Species in Indonesian Deep Slope ``Snapper'' Fisheries}'' was prepared for The Nature Conservancy's (TNC's) Indonesia Fisheries Conservation Program, in support of TNC's Deep Slope ``Snapper'' Fisheries Conservation Project. In the early stages of this program it was recognized that all stakeholders involved in these fisheries (including fishers, buyers, processors, traders, retailers, consumers, managers, NGO workers, government agencies, scientific and educational institutions, etc.) would benefit from the development of (1) a dedicated fish identification guide for the Indonesian deep slope fisheries, and (2) a guide that explains available tools for length-based assessment of the status and trends in these fisheries.

The TNC ``\textit{Top 100 Species Identification Guide for snappers, groupers and emperors in Indonesian deep slope fisheries}'' (Mous et al., 2017) was produced after taxonomic analysis of catches of deep slope drop line and bottom long line fisheries in Central and Eastern Indonesia between 2014 and 2017. This species ID guide for the deep slope fisheries is now available through the following link:

\textbf{CLICK: }\href{http://72.14.187.103:8080/ifish/pub/TNC_FishID.pdf}{Link to on-line E-Book Species ID Guide}

The species ID provides a first good inventory of target species in the deep slope fisheries in Indonesia, with clear images for each target species in the fisheries, together with correct scientific names and a range of common names used in the fisheries and in the trade. At the completion of this guide, it was also clear that additional imagery would be useful for correct identification on board, on landing sites and at monitoring stations. A separate ``Illustration Guide'' was prepared for this purpose, providing additional images ``trawled'' from the internet, selected for high quality and best possible presentation and colors of live or fresh animals.

\textbf{CLICK: }\href{http://72.14.187.103:8080/ifish/pub/DeepSlopeSpeciesIllustrationGuide.pdf}{Link to on-line E-Book Species Illustration  Guide}

The current length-based assessment guide includes simple length-based tools for the assessment of the target fisheries, as well as values of life history parameters for the main target species. This guide needs to be used together with the above mentioned species identification guide, and is meant to:

\begin{enumerate}[noitemsep,topsep=0pt,parsep=0pt,partopsep=0pt,leftmargin=*]
\item Provide up-to-date science-based information on species and fisheries.
\item Support species specific length-based assessments of data poor deep slope fisheries.
\item Define length-based life history characteristics, to enable length-based assessments.
\item Provide values by species for length-based life history parameters including:
	\begin{itemize}[noitemsep,topsep=0pt,parsep=0pt,partopsep=0pt,leftmargin=*]
	\item Maximum attainable total length (Lmax), based on records or estimates from images,
	\item Asymptotic length (Linf), defined as the mean length in a cohort fish, at the time when all individuals in that cohort have stopped growing,
	\item Length at which 50\% of individuals are mature (Lmat) and contributing to reproduction,
	\item Optimum length for harvesting (Lopt) of a species in terms of maximizing yield.
	\end{itemize}
\item Provide simple tools for length-based assessments, using the above length-based life history parameters in combination with catch length frequencies by species.
\item Stimulate discussion on management options among stakeholders and support management decision making based on length-based assessments.
\item Be readily accessible and comprehensible for all stakeholders mentioned above.
\end{enumerate}


% latex table generated in R 3.2.2 by xtable 1.7-4 package
% Wed Apr 19 00:14:16 2017
\begin{table}[ht]
\centering
\caption{Sample Sizes, Length-Weight Relationships \& Trading Limits Snapper Fisheries Indonesia} 
{\small
\begin{tabular}{ccccccccccc}
  \hline
  { } & { } & {Reported} & { } & { } & {Length} & {Converted} & {Plotted} & \multicolumn{3}{c}{Sample Sizes}\\
      { } & { } & {Trade} & \multicolumn{2}{c}{ W = a L\textsuperscript{b}} & {Type} & {Trade} & {Trade} & \multicolumn{3}{c}{Assessment}\\
      { } & { } & {Limit} & { } & { } & {for a \& b} & {Limit} & {Limit} & \multicolumn{3}{c}{WPP 714+715}\\
      {\#ID} & {Species} & {Weight (g)} & {a} & {b} & {TL-FL-SL} & {L(cm)} & {TL(cm)} & {2015} & {2016} & {2017}\\ \hline
1 & Aphareus rutilans & 1000 & 0.015 & 2.961 & FL & 42.20 & 49.61 & 0 & 10340 & 520 \\ 
  2 & Aprion virescens & 1000 & 0.023 & 2.886 & FL & 40.49 & 45.90 & 0 & 1401 & 20 \\ 
  4 & Etelis sp. & 500 & 0.022 & 2.950 & FL & 30.16 & 32.84 & 0 & 5433 & 231 \\ 
  5 & Etelis radiosus & 1000 & 0.056 & 2.689 & FL & 38.05 & 43.15 & 0 & 1982 & 65 \\ 
  6 & Etelis coruscans & 500 & 0.041 & 2.758 & FL & 30.28 & 37.85 & 0 & 3150 & 179 \\ 
  7 & Pristipomoides multidens & 500 & 0.020 & 2.944 & FL & 31.18 & 34.92 & 0 & 5688 & 2378 \\ 
  8 & Pristipomoides typus & 500 & 0.014 & 2.916 & TL & 36.16 & 36.16 & 0 & 1234 & 318 \\ 
  9 & Pristipomoides filamentosus & 500 & 0.038 & 2.796 & FL & 29.70 & 33.27 & 0 & 11690 & 195 \\ 
  10 & Pristipomoides sieboldii & 300 & 0.022 & 2.942 & FL & 25.52 & 29.21 & 0 & 996 & 37 \\ 
  15 & Lutjanus argentimaculatus & 500 & 0.034 & 2.792 & FL & 31.22 & 31.78 & 0 & 973 & 223 \\ 
  16 & Lutjanus bohar & 500 & 0.016 & 3.059 & FL & 29.70 & 31.31 & 0 & 529 & 21 \\ 
  17 & Lutjanus malabaricus & 500 & 0.009 & 3.137 & FL & 33.11 & 33.11 & 0 & 615 & 6950 \\ 
  18 & Lutjanus sebae & 500 & 0.009 & 3.208 & FL & 29.97 & 31.26 & 0 & 426 & 358 \\ 
  19 & Lutjanus timorensis & 500 & 0.009 & 3.137 & FL & 33.11 & 33.34 & 0 & 3389 & 387 \\ 
  20 & Lutjanus gibbus & 500 & 0.015 & 3.091 & FL & 28.87 & 31.09 & 0 & 788 & 2 \\ 
  21 & Lutjanus erythropterus & 500 & 0.024 & 2.870 & FL & 31.79 & 31.79 & 0 & 4350 & 1386 \\ 
  22 & Pinjalo lewisi & 300 & 0.014 & 2.970 & FL & 28.42 & 29.64 & 0 & 2281 & 354 \\ 
  27 & Lutjanus vitta & 300 & 0.017 & 2.978 & FL & 26.72 & 27.64 & 0 & 227 & 0 \\ 
  31 & Symphorus nematophorus & 1000 & 0.015 & 3.046 & FL & 38.63 & 40.18 & 0 & 316 & 46 \\ 
  32 & Paracaesio gonzalesi & 300 & 0.020 & 3.050 & FL & 23.24 & 24.96 & 0 & 878 & 12 \\ 
  33 & Paracaesio xanthura & 300 & 0.023 & 3.000 & SL & 23.64 & 27.39 & 0 & 379 & 54 \\ 
  34 & Paracaesio kusakarii & 500 & 0.011 & 3.135 & FL & 30.96 & 34.80 & 0 & 4298 & 281 \\ 
  35 & Paracaesio stonei & 500 & 0.024 & 2.960 & FL & 28.78 & 32.35 & 0 & 1665 & 131 \\ 
  39 & Cephalopholis sonnerati & 300 & 0.015 & 3.058 & TL & 25.78 & 25.78 & 0 & 635 & 16 \\ 
  42 & Epinephelus radiatus & 300 & 0.061 & 2.624 & FL & 25.59 & 25.59 & 0 & 211 & 8 \\ 
  43 & Epinephelus morrhua & 300 & 0.061 & 2.624 & FL & 25.59 & 25.59 & 0 & 724 & 45 \\ 
  45 & Epinephelus areolatus & 300 & 0.011 & 3.048 & FL & 28.18 & 28.77 & 0 & 1438 & 44 \\ 
  48 & Epinephelus bilobatus & 300 & 0.014 & 2.990 & TL & 27.82 & 27.82 & 0 & 214 & 2 \\ 
  50 & Epinephelus coioides & 1500 & 0.011 & 3.084 & TL & 46.94 & 46.94 & 0 & 258 & 118 \\ 
  51 & Epinephelus chlorostigma & 500 & 0.015 & 2.940 & FL & 34.62 & 34.62 & 0 & 412 & 50 \\ 
  54 & Epinephelus stictus & 300 & 0.027 & 3.000 & SL & 22.37 & 28.24 & 0 & 373 & 3 \\ 
  57 & Epinephelus undulosus & 1500 & 0.015 & 2.940 & FL & 50.31 & 50.31 & 0 & 243 & 2 \\ 
  58 & Epinephelus amblycephalus & 1500 & 0.012 & 3.057 & TL & 45.99 & 45.99 & 0 & 449 & 42 \\ 
  62 & Variola albimarginata & 300 & 0.012 & 3.079 & FL & 26.68 & 30.44 & 0 & 270 & 9 \\ 
  67 & Lethrinus olivaceus & 300 & 0.029 & 2.851 & FL & 25.49 & 27.50 & 0 & 789 & 15 \\ 
  68 & Lethrinus amboinensis & 300 & 0.029 & 2.851 & FL & 25.49 & 28.06 & 0 & 634 & 10 \\ 
  70 & Wattsia mossambica & 500 & 0.040 & 2.824 & FL & 28.21 & 29.34 & 0 & 2504 & 48 \\ 
  71 & Gymnocranius grandoculis & 500 & 0.032 & 2.885 & FL & 28.43 & 30.53 & 0 & 1665 & 131 \\ 
  76 & Carangoides chrysophrys & 1000 & 0.027 & 2.902 & FL & 37.68 & 42.12 & 0 & 527 & 369 \\ 
  77 & Carangoides gymnostethus & 1000 & 0.046 & 2.746 & FL & 37.88 & 41.55 & 0 & 263 & 6 \\ 
  79 & Caranx ignobilis & 2000 & 0.027 & 2.913 & FL & 46.78 & 54.36 & 0 & 259 & 4 \\ 
  81 & Caranx sexfasciatus & 2000 & 0.032 & 2.930 & FL & 43.43 & 49.51 & 0 & 619 & 97 \\ 
  82 & Caranx tille & 2000 & 0.032 & 2.930 & FL & 43.43 & 49.51 & 0 & 402 & 79 \\ 
  83 & Elagatis bipinnulata & 1000 & 0.013 & 2.920 & FL & 46.53 & 55.37 & 0 & 416 & 18 \\ 
  85 & Seriola rivoliana & 2000 & 0.006 & 3.170 & FL & 54.23 & 60.03 & 0 & 995 & 91 \\ 
  86 & Erythrocles schlegelii & 1500 & 0.011 & 3.040 & FL & 48.55 & 53.60 & 0 & 1157 & 101 \\ 
  88 & Dentex carpenteri & 300 & 0.023 & 2.930 & FL & 25.42 & 27.66 & 0 & 521 & 30 \\ 
  89 & Glaucosoma buergeri & 500 & 0.045 & 2.725 & TL & 30.40 & 30.40 & 0 & 496 & 67 \\ 
  91 & Diagramma pictum & 500 & 0.014 & 2.988 & FL & 33.08 & 36.71 & 0 & 303 & 24 \\ 
  93 & Cookeolus japonicus & 300 & 0.014 & 3.000 & TL & 27.58 & 27.58 & 0 & 262 & 7 \\ 
   \hline
\end{tabular}
}
\end{table}

\clearpage
\newpage

\chapter{Materials and methods for data collection, analysis and reporting}
\section{SPOT Trace vessel tracking}

Fishing grounds are determined by placing Spot Trace units on all fishing boats participating in this program. When in motion, Spot Trace units automatically report an hourly location, and when at rest for more than 24 hours, they relay daily status reports. Location and status report messages are automatically recorded in I-Fish Community, an online database running PostgreSQL with a user interface programmed in Java and analysis and reporting procedures in R and Latex.

Fishing vessels with Spot Trace units on board generate accurate data on fishing grounds and specific fishing locations within fishing grounds. Traditionally, fishing ground data were often collected from logbook data or captain interviews. However, logbook and interview data are sometimes unclear, inaccurate and can easily be falsified. The Spot Trace enables us to match catch data with exact fishing locations, while at the same time providing additional safety features on board the fishing vessels. To mitigate IUU fishing accusations, having the Spot Trace onboard can also be used as proof of legal fishing within Indonesian waters.

\section{Crew Operated Data Recording System}

Data on species and size distributions of complete catches are needed for accurate length based stock assessments. Such data on individual fishing trips are collected via Crew Operated Data Recording Systems or CODRS. This catch data is geo-referenced as the CODRS works in tandem with the Spot Trace vessel tracking system. Crews of fishing vessels are contracted to take images on project-supplied digital cameras of all fish in the catch, positioned over measuring boards. This procedure takes place when batches of fish are taken from chiller boxes on deck, before they are packed on ice in the hold. The crew photographs all the fish in this manner and at the end of the trip hands in the storage chip from the camera to a project staff who analyzes the images back at the fisheries station. Analysis of the images includes ID of the species and reading of the length of the fish as displayed on the measuring board. Double checking with owner and trader data on total catches, and comparison with weights as calculated from fish lengths, ensures that we are capturing length frequencies of the total catch. It is essential to ensure that no species or size classes are missing before analysis. If estimated catch weight from CODRS data differs more than 30% from estimates based on boat owner data, the catch is not included in the length based assessment, to remove any chance of bias.

\begin{center}
\graphicspath{{/root/R-project/IFishSnapperWPP718/Images/}}
\includegraphics[scale=0.7]{codrs.jpg}

Figure 3. Fishing crew preparing fish on a measuring board.
\end{center}

\begin{center}
\graphicspath{{/root/R-project/IFishSnapperWPP718/Images/}}
\includegraphics[scale=0.5]{codrs2.png}

Figure 4. Fish photographed by fishing crew on board as part of CODRS.
\end{center}

\clearpage
\newpage

\section{I-Fish Community}

I-Fish Community only stores data that are relevant to fisheries management, whereas data on processed volume and sales, from the Smart Weighing and Measuring System, remain on servers at processing companies. Access to the I-Fish Community database is controlled by user name and password. I-Fish Community has different layers of privacy, which is contingent on the user's role in the supply chain. For instance, boat owners may view exact location of their boats, but not of the boats of other owners.

I-Fish Community has an automatic length-frequency distribution reporting system for length-based assessment of the fishery by species. The database generates length frequency distribution graphs for each species, together with life history parameters including length at maturity (Lmat), optimum harvest size (Lopt), asymptotic length- (Linf), and maximum total length (Lmax), as well as size limits used in the trade. These "trade limit" lengths are derived from general buying behavior (minimal weight) of processing companies. The weights are converted into lengths by using species-specific length- weight relationships.

Each graph (length frequency distribution by species) is accompanied by an automated length-based assessment. Any I-Fish Community user can access these graphs and the conclusions from the assessments in real time. The report is updated daily and produces a length based assessment for the 50 most abundant target species in the fishery, based on complete catches recorded on board by fishing boat crews. The graphs show the position of the catch length frequency distributions relative to various life history parameter values and trading limits for each species.

Immature fish, small mature fish, large mature fish, and a subset of large mature fish, namely "mega-spawners", which are fish larger than 1.1 times the optimum harvest size (Froese 2004), make up the specific size groups used in our length based assessment. For all fish of each species in the catch, the percentage in each category is calculated for further use in the length based assessment. These percentages are calculated and presented as the first step in the length based assessment as follows: W\% is immature (smaller than the length at maturity), X\% is small matures (at or above size at maturity but smaller than the optimum harvest size), and Y\% is large mature fish (at or above optimum harvest size). The percentage of mega-spawners is Z\%.

The automated assessment comprises of six elements from the catch length frequencies. These elements all work with length based indicators of various kinds to draw conclusions from species specific length frequencies in the catch.

\textit{1. Proportion of immature fish in the catch.}

With 0\% immature fish in the catch as an ideal target (Froese, 2004), a target of 10\% or less is considered a reasonable indicator for sustainable (or safe) harvesting (Fujita et al., 2012; Vasilakopoulos et al., 2011). Zhang et al. (2009) consider 20\% immature fish in the catch as an indicator for a fishery at risk, in their approach to an ecosystem based fisheries assessment. Results from meta-analysis over multiple fisheries showed stock status over a range of stocks to fall below precautionary limits at 30\% or more immature fish in the catch (Vasilakopoulos et al., 2011). The fishery is considered highly at risk when more than 50\% of the fish in the catch are immature (Froese et al, 2016).

\clearpage
\newpage

IF "\% immature" is lower than or equal to 10\% THEN:\\[0cm]
"At least 90\% of the fish in the catch are mature specimens that have spawned at least once before they were caught. The fishery does not depend on immature size classes for this species and is considered safe for this indicator. This fishery will not be causing overfishing through over harvesting of juveniles for this species. Risk level is low."

ELSE, IF "\% immature" is greater than 10\% AND "\% immature" is lower than or equal to 20\% THEN:\\[0cm]
"Between 10\% and 20\% of the fish in the catch are juveniles that have not yet reproduced. There is no immediate concern in terms of overfishing through over harvesting of juveniles, but the fishery needs to be monitored closely for any further increase in this indicator and incentives need to be geared towards targeting larger fish. Risk level is medium."

ELSE, IF "\% immature" is greater than 20\% AND "\% immature" is lower than or equal to 30\% THEN:\\[0cm]
"Between 20\% and 30\% of the fish in the catch are specimens that have not yet reproduced. This is reason for concern in terms of potential overfishing through overharvesting of juveniles, if fishing pressure is high and percentages immature fish would further rise. Targeting larger fish and avoiding small fish in the catch will promote a sustainable fishery. Risk level is medium."

ELSE, IF "\% immature" is greater than 30\% AND "\% immature" is lower than or equal to 50\% THEN:\\[0cm]
"Between 30\% and 50\% of the fish in the catch are immature and have not had a chance to reproduce before capture. The fishery is in immediate danger of overfishing through overharvesting of juveniles, if fishing pressure is high.  Catching small and immature fish needs to be actively avoided and a limit on overall fishing pressure is warranted. Risk level is high."

ELSE, IF "\% immature" is greater than 50\% THEN:\\[0cm]
"The majority of the fish in the catch have not had a chance to reproduce before capture. This fishery is most likely overfished already if fishing mortality is high for all size classes in the population. An immediate shift away from targeting juvenile fish and a reduction in overall fishing pressure is essential to prevent collapse of the stock. Risk level is high."

\textit{2. Minimum size as traded compared to length and maturity.}

We use a comparison between the trade limit (minimum size accepted by the trade) and the size at maturity as an indicator for incentives from the trade for either unsustainable targeting of juveniles or for more sustainable targeting of mature fish that have spawned at least once. We consider a trade limit at 10\% below or above the length at maturity to be significantly different from the length at maturity and we consider trade limits to provide incentives for targeting of specific sizes of fish through price differentiation.

IF "TradeLimit" is lower than 0.9 * L-mat THEN:\\[0cm]
"The trade limit is significantly lower than the length at first maturity.  This means that the trade encourages capture of immature fish, which impairs sustainability. Risk level is high."

\clearpage
\newpage

ELSE, IF "TradeLimit" is greater than or equal to 0.9 * L-mat AND "TradeLimit" is lower than or equal to 1.1 * L-mat THEN:\\[0cm]
"The trade limit is about the same as the length at first maturity.  This means that the trade puts a premium on fish that have spawned at least once, which improves sustainability of the fishery. Risk level is medium."

ELSE, IF "TradeLimit" is greater than 1.1 * L-mat THEN:\\[0cm]
"The trade limit is significantly higher than length at first maturity.  This means that the trade puts a premium on fish that have spawned at least once. The trade does not cause any concern of recruitment overfishing for this species. Risk level is low."

\textit{3. Current exploitation level.}

We use the current exploitation level expressed as the percentage of fish in the catch below the optimum harvest size as an indicator for fisheries status. We consider a proportion of 65\% of the fish (i.e. the vast majority in numbers) in the catch below the optimum harvest size as an indicator for growth overfishing. We also consider a majority in the catch around or above the optimum harvest size as an indicator for minimizing the impact of fishing (Froese et al., 2016). This indicator will be achieved when less than 50\% of the fish in the catch are below the optimum harvest size.

IF "\% immature + \% small mature" is greater than or equal to 65\% THEN:\\[0cm]
"The vast majority of the fish in the catch have not yet achieved their growth potential. The harvest of small fish promotes growth overfishing and the size distribution for this species indicates that over exploitation through growth overfishing may already be happening. Risk level is high."

ELSE, IF "\% immature + \% small mature" is lower than or equal to 50\% THEN:\\[0cm]
"The majority of the catch consists of size classes around or above the optimum harvest size. This means that the impact of the fishery is minimized for this species. Potentially higher yields of this species could be achieved by catching them at somewhat smaller size, although capture of smaller specimen may take place already in other fisheries. Risk level is low."

ELSE, IF "\% immature + \% small mature" is greater than 50\% AND "\% immature + \% small mature" is lower than 65\% THEN:\\[0cm]
"The bulk of the catch includes age groups that have just matured and are about to achieve their full growth potential. This indicates that the fishery is probably at least being fully exploited. Risk level is medium."

\textit{4. Proportion of mega spawners in the catch.}

Mega spawners are fish larger than 1.1 times the optimum harvest size. We consider a proportion of 30\% or more mega spawners in the catch to be a sign of a healthy population (Froese, 2004), whereas lower proportions are increasingly leading to concerns, with proportions below 20\% indicating great risk to the fishery.

IF "\% mega spawners" is greater than 30\% THEN:\\[0cm]
"More than 30\% of the catch consists of mega spawners which indicates that this fish population is in good health unless large amounts of much smaller fish from the same population are caught by other fisheries. Risk level is low."

\clearpage
\newpage

ELSE, IF "\% mega spawners" is greater than 20\% AND "\% mega spawners" is lower than or equal to 30\% THEN:\\[0cm]
"The percentage of mega spawners is between 20 and 30\%.  There is no immediate reason for concern, though fishing pressure may be significantly reducing the percentage of mega-spawners, which may negatively affect the reproductive output of this population. Risk level is medium."

ELSE, IF "\% mega spawners" is lower than or equal to 20\%, THEN:\\[0cm]
"Less than 20\% of the catch comprises of mega spawners.  This indicates that the population may be severely affected by the fishery, and that there is a substantial risk of recruitment overfishing through over harvesting of the mega spawners, unless large numbers of mega spawners would be surviving at other habitats. There is no reason to assume that this is the case and therefore a reduction of fishing effort may be necessary in this fishery. Risk level is high.

\textit{5. Take less than nature.}

Rule number one to minimize the impact of fishing (Froese et al., 2016) teaches us to "take less than nature" by ensuring that mortality caused by fishing is less than the natural rate of mortality. We consider a fishing mortality of less than half the natural mortality to be necessary to minimize the impact of fishing. We estimated the instantaneous total mortality (Z) from the equilibrium Beverton-Holt estimator from length data using Ehrhardt and Ault (1992) bias-correction, implemented through the function bheq2 of the R Fishmethods package. We estimated the natural rate of mortality (M) using Froese and Pauly (2000) empirical formula with asymptotic length as estimated by species and an ambient water temperature at fishing depth estimated at about 20 degrees Celcius. With an asymptotic length for a snapper of about 80cm this results in an M of about 0.4, which aligns well with the mean of reported values from the literature (Martinez-Andrade, 2003). The fishing mortality F follows as the difference between total and natural mortality.

IF "fishing mortality" is greater than or equal to "natural mortality" THEN:\\[0cm]
Mortality caused by fishing is greater than or equal to the natural rate of mortality. This means that impact of fishing is severe and that fishing is unlikely to be sustainable at the current level of effort. Risk level is high.

IF "fishing mortality" is lower than "natural mortality" AND "fishing mortality" is greater than 0.5 times "natural mortality" THEN:\\[0cm]
Mortality caused by fishing is lower than the natural rate of mortality but more than half of natural mortality. This means that impact of fishing is considerable and trends in various indicators need to be watched carefully while any increase in fishing effort needs to be prevented. Risk level is medium.

IF "fishing mortality" is lower than or equal to 0.5 times "natural mortality" THEN:\\[0cm]
Mortality caused by fishing is at or below a level equal to half the natural rate of mortality. This means that impact of fishing is minimized and this fishery is currently probably operating at a sustainable level of effort. Risk level is low.

\clearpage
\newpage

\textit{6. Spawning Potential Ratio.}

As an indicator for Spawning Potential Ratio (SPR, Quinn and Deriso, 1999), we used the estimated spawning stock biomass divided by the spawning stock biomass of that population it it would have been pristine (see, for example, Meester et al 2001). We calculated SPR on a per-recruit basis from life-history parameters Z, F, K (von Bertalanffy), and Linf. We estimated Z and F as explained above and K from Lopt, using the method presented in Froese and Binohlan 2000.

In a perfect world, fishery biologists would know what the appropriate SPR should be for every harvested stock based on the biology of that stock. Generally, however, not enough is known about managed stocks to be so precise. However, studies show that some stocks (depending on the species of fish) can maintain themselves if the spawning stock biomass per recruit can be kept at 20 to 35\% (or more) of what it was in the un-fished stock. Lower values of SPR may lead to severe stock declines (Wallace and Fletcher, 2001). Froese et al. (2016) considered a total population biomass B of half the pristine population biomass Bo to be the lower limit reference point for stock size, minimizing the impact of fishing. Using SPR and B/Bo estimates from our own data set, this Froese et al. (2016) lower limit reference point correlates with an SPR of about 40\%, not far from but slightly more conservative than the Wallace and Fletcher (2001) reference point. We chose an SPR of 40\% as our reference point for high risk and after similar comparisons we consider and SPR between 25\% and 40\% to represent a medium risk situation.

IF "SPR" is lower than 25\% THEN:\\[0cm]
"SPR is less than 25\%. The fishery probably over-exploits the stock, and there is a substantial risk that the fishery will cause severe decline of the stock if fishing effort is not reduced. Risk level is high."

ELSE, IF "SPR" is greater than or equal to 25\% AND "SPR" is lower than 40\% THEN:\\[0cm]
"SPR is between 25\% and 40\%. The stock is heavily exploited, and there is some risk that the fishery will cause further decline of the stock. Risk level is medium."

ELSE, IF "SPR" is greater than or equal to 40\% THEN:\\[0cm]
"SPR is more than 40\%. The stock is probably not over exploited, and the risk that the fishery will cause further stock decline is small. Risk level is low."

\newpage

\chapter{Fishing grounds and traceability}
The Spot Trace data from the Timor Sea and Arafura Sea fisheries illustrate a classic "fishing the line" phenomenon. Many vessels fish right at the Indonesia - Australia border, on the edge of better managed fishing grounds on the Australian side, where fish densities are expected to be higher. Several drop line fishers were observed to operate illegally in Australian waters and some of these have been arrested by Australian patrol boats in 2015. Some drop line long line vessels have also been observed to illegally fish in Timor Leste waters. There is apparently little or no enforcement of fisheries regulations in Timor Leste waters and especially the Joint Petroleum Development Area or JPDA (an area in Timor Leste waters where a resource sharing agreement for seabed resources is in place with Australia) is frequently targeted illegally by Indonesian vessels.

The Spot Trace data from WPP 718 and surrounding areas show great mobility of the medium-scale snapper fishing boats, making trips to fishing grounds that are up to 1,000 kilometers away from home ports. Not only are these fleets highly mobile in terms of their trips from home port, they are also flexible in changing their base of operations from one port to another, changing from landing at home port to offloading on transport vessels in remote ports or offloading for air cargo at yet other places. Decision making on movements by boat owners can be based on fisheries technical issues such as catch rates or weather, but also on administrative issues like licensing or enforcement of rules against under-marking in Gross Tonnage. Most recently we are observing movement of staging ports but also of processing capacity to remote areas in the east such as the island of Penambulai, East of the Aru Islands. Fish is landed there and moved onto transport vessels bound for processing plants elsewhere in the country.

Therefore the fish that is processed in major processing centers like Probolinggo comes from a number of different fleets that operate throughout the waters of Eastern Indonesian, including also WPP 718. For the purpose of this report, all fishing trips, recorded (from SPOT data) within WPP 718, mostly from long line and drop line operations, were included in the analysis for this WPP. This includes fishing trips originating from outside the WPP, for example from Probolinggo, Bali or Kema.

Potential IUU issues include the operation by various fleets outside Indonesian waters in the East Timorese - Australian JPDA as well as in strictly Australian waters. Additional issues include the under marking of medium scale vessels to below 30GT, the licensing of the various fleets for various WPP and the operation of fleets from remote ports inside Marine Protected Areas throughout Eastern Indonesia. All this needs to be discussed with fishing boat captains and boat owners to prevent issues of supply line "pollution" with IUU fish from thee protected areas.

\begin{center}
\graphicspath{{/root/R-project/IFishSnapperWPP718/Images/}}
\includegraphics[scale=0.4]{JPDA-Map.png}
\end{center}

\begin{wrapfigure}{r}{8cm}
\includegraphics[width=1\linewidth]{/root/R-project/IFishSnapperWPP718/Images/JPDA-Map-bw.png}
\end{wrapfigure}

Figure 5. Timor Sea and Arafura Sea fishing grounds with current boundaries between Indonesia, East Timor and Australia.

a) The dotted line is the Australia - Indonesia Seabed Boundary. The pink line (PFSEL) is the Australia - Indonesia Fisheries Boundary. Indonesian vessels are allowed to fish in the grey area between the pink line and the dotted line, but not below the PFSEL. The light blue line is the boundary of the East Timor - Australia Zone of Cooperation which covers East Timorese fishing grounds where Indonesian fishing vessels are not allowed to fish. Australia does not enforce fisheries regulations here.

b) The shaded area between the Seabed Boundary and the Fisheries Boundary is Australian seabed, where fishers from Indonesia are allowed to fish. The Australian - East Timor zone of cooperation or �Joint Petroleum Development Area� (JPDA) is not open to fishers from Indonesia. East Timor is responsible for fishery surveillance within the JPDA.

Source: Australian Surveying \& Land Information Group (AUSLIG) Commonwealth Department of Industry Science and Resources. MAP 96/523.21.1.

\clearpage
\newpage

\begin{center}
\graphicspath{{/root/R-project/IFishSnapperWPP718/Images/}}
\includegraphics[width=15cm]{SpotTrace-AreaD-Dropline.png}

Figure 6. Tracks of drop line fishing boats from various ports, operating in the Arafura Sea (WPP 718) and the JPDA (East Timor waters).
\end{center}

\begin{center}
\graphicspath{{/root/R-project/IFishSnapperWPP718/Images/}}
\includegraphics[width=15cm]{SpotTrace-AreaD-Longline.png}

Figure 7. Tracks of bottom long line fishing boats from various ports, operating in the Arafura Sea (WPP 718) and the JPDA (East Timor waters)
\end{center}

\newpage

\chapter{Species-specific length-based assessments}
\verbatimfont{\normalfont\rmfamily}
\definecolor{fgcolor}{rgb}{0,0,0}
\begin{knitrout}
\definecolor{shadecolor}{rgb}{1, 1, 1}\color{fgcolor}
\includegraphics[width=\maxwidth]{/root/R-project/IFishSnapperWPP714_715/Plots/plot-LFD-1} 

\includegraphics[width=\maxwidth]{/root/R-project/IFishSnapperWPP714_715/Plots/plot-LFD-2} 
\begin{kframe}\begin{verbatim}
The percentages of Aphareus rutilans (ID #1, Lutjanidae) in 2016, n = 10,340
Immature (< 66cm): 33%
Small mature (>= 66cm, < 88cm): 60%
Large mature (>= 88cm): 7%
Mega spawner (>= 96.8cm): 1% (subset of large mature fish)
Spawning Potential Ratio: 3%
 
Between 30% and 50% of the fish in the catch are immature and have not had a chance
to reproduce before capture. The fishery is in immediate danger of overfishing
through overharvesting of juveniles, if fishing pressure is high.  Catching small and
immature fish needs to be actively avoided and a limit on overall fishing pressure is
warranted. Risk level is high.

The trade limit is significantly lower than the length at first maturity.  This means
that the trade encourages capture of immature fish, which impairs sustainability.
Risk level is high.

The vast majority of the fish in the catch have not yet achieved their growth
potential. The harvest of small fish promotes growth overfishing and the size
distribution for this species indicates that over exploitation through growth
overfishing may already be happening. Risk level is high.

Less than 20% of the catch comprises of mega spawners.  This indicates that the
population may be severely affected by the fishery, and that there is a substantial
risk of recruitment overfishing through over harvesting of the mega spawners, unless
large numbers of mega spawners would be surviving at other habitats. There is no
reason to assume that this is the case and therefore a reduction of fishing effort
may be necessary in this fishery. Risk level is high.
 
Mortality caused by fishing is greater than or equal to the natural rate of
mortality. This means that impact of fishing is severe and that fishing is unlikely
to be sustainable at the current level of effort. Risk level is high.
 
SPR is less than 25%. The fishery probably over-exploits the stock, and there is a
substantial risk that the fishery will cause severe decline of the stock if fishing
effort is not reduced. Risk level is high.
 
Trends in relative abundance by size group for Aphareus rutilans (ID #1, Lutjanidae),
as calculated from linear regressions. The P value indicates the chance that this
calculated trend is merely a result of stochastic variance.
% Immature rising over recent years, situation deteriorating. P: not available
% Large Mature no trend over recent years, situation stable. P: not available
% Mega Spawner no trend over recent years, situation stable. P: not available
% SPR rising over recent years, situation improving. P: not available
\end{verbatim}
\end{kframe}
\includegraphics[width=\maxwidth]{/root/R-project/IFishSnapperWPP714_715/Plots/plot-LFD-3} 

\includegraphics[width=\maxwidth]{/root/R-project/IFishSnapperWPP714_715/Plots/plot-LFD-4} 
\begin{kframe}\begin{verbatim}
\end{verbatim}
\end{kframe}
\clearpage
\newpage
\begin{kframe}\begin{verbatim}The percentages of Aprion virescens (ID #2, Lutjanidae) in 2016, n = 1,401
Immature (< 58cm): 8%
Small mature (>= 58cm, < 78cm): 45%
Large mature (>= 78cm): 47%
Mega spawner (>= 85.8cm): 16% (subset of large mature fish)
Spawning Potential Ratio: 6%
 
At least 90% of the fish in the catch are mature specimens that have spawned at least
once before they were caught. The fishery does not depend on immature size classes
for this species and is considered safe for this indicator. This fishery will not be
causing overfishing through over harvesting of juveniles for this species. Risk level
is low.

The trade limit is significantly lower than the length at first maturity.  This means
that the trade encourages capture of immature fish, which impairs sustainability.
Risk level is high.

The bulk of the catch includes age groups that have just matured and are about to
achieve their full growth potential. This indicates that the fishery is probably at
least being fully exploited. Risk level is medium.

Less than 20% of the catch comprises of mega spawners.  This indicates that the
population may be severely affected by the fishery, and that there is a substantial
risk of recruitment overfishing through over harvesting of the mega spawners, unless
large numbers of mega spawners would be surviving at other habitats. There is no
reason to assume that this is the case and therefore a reduction of fishing effort
may be necessary in this fishery. Risk level is high.
 
Mortality caused by fishing is greater than or equal to the natural rate of
mortality. This means that impact of fishing is severe and that fishing is unlikely
to be sustainable at the current level of effort. Risk level is high.
 
SPR is less than 25%. The fishery probably over-exploits the stock, and there is a
substantial risk that the fishery will cause severe decline of the stock if fishing
effort is not reduced. Risk level is high.
 
Trends in relative abundance by size group for Aprion virescens (ID #2, Lutjanidae),
as calculated from linear regressions. The P value indicates the chance that this
calculated trend is merely a result of stochastic variance.
% Immature falling over recent years, situation improving. P: not available
% Large Mature falling over recent years, situation deteriorating. P: not available
% Mega Spawner rising over recent years, situation improving. P: not available
% SPR falling over recent years, situation deteriorating. P: not available
\end{verbatim}
\end{kframe}
\includegraphics[width=\maxwidth]{/root/R-project/IFishSnapperWPP714_715/Plots/plot-LFD-5} 

\includegraphics[width=\maxwidth]{/root/R-project/IFishSnapperWPP714_715/Plots/plot-LFD-6} 
\begin{kframe}\begin{verbatim}
\end{verbatim}
\end{kframe}
\clearpage
\newpage
\begin{kframe}\begin{verbatim}The percentages of Etelis sp. (ID #4, Lutjanidae) in 2016, n = 5,433
Immature (< 69cm): 71%
Small mature (>= 69cm, < 92cm): 27%
Large mature (>= 92cm): 2%
Mega spawner (>= 101.2cm): 0% (subset of large mature fish)
Spawning Potential Ratio: 2%
 
The majority of the fish in the catch have not had a chance to reproduce before
capture. This fishery is most likely overfished already if fishing mortality is high
for all size classes in the population. An immediate shift away from targeting
juvenile fish and a reduction in overall fishing pressure is essential to prevent
collapse of the stock. Risk level is high.

The trade limit is significantly lower than the length at first maturity.  This means
that the trade encourages capture of immature fish, which impairs sustainability.
Risk level is high.

The vast majority of the fish in the catch have not yet achieved their growth
potential. The harvest of small fish promotes growth overfishing and the size
distribution for this species indicates that over exploitation through growth
overfishing may already be happening. Risk level is high.

Less than 20% of the catch comprises of mega spawners.  This indicates that the
population may be severely affected by the fishery, and that there is a substantial
risk of recruitment overfishing through over harvesting of the mega spawners, unless
large numbers of mega spawners would be surviving at other habitats. There is no
reason to assume that this is the case and therefore a reduction of fishing effort
may be necessary in this fishery. Risk level is high.
 
Mortality caused by fishing is greater than or equal to the natural rate of
mortality. This means that impact of fishing is severe and that fishing is unlikely
to be sustainable at the current level of effort. Risk level is high.
 
SPR is less than 25%. The fishery probably over-exploits the stock, and there is a
substantial risk that the fishery will cause severe decline of the stock if fishing
effort is not reduced. Risk level is high.
 
Trends in relative abundance by size group for Etelis sp. (ID #4, Lutjanidae), as
calculated from linear regressions. The P value indicates the chance that this
calculated trend is merely a result of stochastic variance.
% Immature falling over recent years, situation improving. P: not available
% Large Mature falling over recent years, situation deteriorating. P: not available
% Mega Spawner no trend over recent years, situation stable. P: not available
% SPR no trend over recent years, situation stable. P: not available
\end{verbatim}
\end{kframe}
\includegraphics[width=\maxwidth]{/root/R-project/IFishSnapperWPP714_715/Plots/plot-LFD-7} 

\includegraphics[width=\maxwidth]{/root/R-project/IFishSnapperWPP714_715/Plots/plot-LFD-8} 
\begin{kframe}\begin{verbatim}
\end{verbatim}
\end{kframe}
\clearpage
\newpage
\begin{kframe}\begin{verbatim}The percentages of Etelis radiosus (ID #5, Lutjanidae) in 2016, n = 1,982
Immature (< 56cm): 40%
Small mature (>= 56cm, < 74cm): 49%
Large mature (>= 74cm): 11%
Mega spawner (>= 81.4cm): 4% (subset of large mature fish)
Spawning Potential Ratio: 6%
 
Between 30% and 50% of the fish in the catch are immature and have not had a chance
to reproduce before capture. The fishery is in immediate danger of overfishing
through overharvesting of juveniles, if fishing pressure is high.  Catching small and
immature fish needs to be actively avoided and a limit on overall fishing pressure is
warranted. Risk level is high.

The trade limit is significantly lower than the length at first maturity.  This means
that the trade encourages capture of immature fish, which impairs sustainability.
Risk level is high.

The vast majority of the fish in the catch have not yet achieved their growth
potential. The harvest of small fish promotes growth overfishing and the size
distribution for this species indicates that over exploitation through growth
overfishing may already be happening. Risk level is high.

Less than 20% of the catch comprises of mega spawners.  This indicates that the
population may be severely affected by the fishery, and that there is a substantial
risk of recruitment overfishing through over harvesting of the mega spawners, unless
large numbers of mega spawners would be surviving at other habitats. There is no
reason to assume that this is the case and therefore a reduction of fishing effort
may be necessary in this fishery. Risk level is high.
 
Mortality caused by fishing is greater than or equal to the natural rate of
mortality. This means that impact of fishing is severe and that fishing is unlikely
to be sustainable at the current level of effort. Risk level is high.
 
SPR is less than 25%. The fishery probably over-exploits the stock, and there is a
substantial risk that the fishery will cause severe decline of the stock if fishing
effort is not reduced. Risk level is high.
 
Trends in relative abundance by size group for Etelis radiosus (ID #5, Lutjanidae),
as calculated from linear regressions. The P value indicates the chance that this
calculated trend is merely a result of stochastic variance.
% Immature no trend over recent years, situation stable. P: not available
% Large Mature rising over recent years, situation improving. P: not available
% Mega Spawner falling over recent years, situation deteriorating. P: not available
% SPR rising over recent years, situation improving. P: not available
\end{verbatim}
\end{kframe}
\includegraphics[width=\maxwidth]{/root/R-project/IFishSnapperWPP714_715/Plots/plot-LFD-9} 

\includegraphics[width=\maxwidth]{/root/R-project/IFishSnapperWPP714_715/Plots/plot-LFD-10} 
\begin{kframe}\begin{verbatim}
\end{verbatim}
\end{kframe}
\clearpage
\newpage
\begin{kframe}\begin{verbatim}The percentages of Etelis coruscans (ID #6, Lutjanidae) in 2016, n = 3,150
Immature (< 69cm): 68%
Small mature (>= 69cm, < 92cm): 28%
Large mature (>= 92cm): 5%
Mega spawner (>= 101.2cm): 1% (subset of large mature fish)
Spawning Potential Ratio: 3%
 
The majority of the fish in the catch have not had a chance to reproduce before
capture. This fishery is most likely overfished already if fishing mortality is high
for all size classes in the population. An immediate shift away from targeting
juvenile fish and a reduction in overall fishing pressure is essential to prevent
collapse of the stock. Risk level is high.

The trade limit is significantly lower than the length at first maturity.  This means
that the trade encourages capture of immature fish, which impairs sustainability.
Risk level is high.

The vast majority of the fish in the catch have not yet achieved their growth
potential. The harvest of small fish promotes growth overfishing and the size
distribution for this species indicates that over exploitation through growth
overfishing may already be happening. Risk level is high.

Less than 20% of the catch comprises of mega spawners.  This indicates that the
population may be severely affected by the fishery, and that there is a substantial
risk of recruitment overfishing through over harvesting of the mega spawners, unless
large numbers of mega spawners would be surviving at other habitats. There is no
reason to assume that this is the case and therefore a reduction of fishing effort
may be necessary in this fishery. Risk level is high.
 
Mortality caused by fishing is greater than or equal to the natural rate of
mortality. This means that impact of fishing is severe and that fishing is unlikely
to be sustainable at the current level of effort. Risk level is high.
 
SPR is less than 25%. The fishery probably over-exploits the stock, and there is a
substantial risk that the fishery will cause severe decline of the stock if fishing
effort is not reduced. Risk level is high.
 
Trends in relative abundance by size group for Etelis coruscans (ID #6, Lutjanidae),
as calculated from linear regressions. The P value indicates the chance that this
calculated trend is merely a result of stochastic variance.
% Immature rising over recent years, situation deteriorating. P: not available
% Large Mature no trend over recent years, situation stable. P: not available
% Mega Spawner no trend over recent years, situation stable. P: not available
% SPR rising over recent years, situation improving. P: not available
\end{verbatim}
\end{kframe}
\includegraphics[width=\maxwidth]{/root/R-project/IFishSnapperWPP714_715/Plots/plot-LFD-11} 

\includegraphics[width=\maxwidth]{/root/R-project/IFishSnapperWPP714_715/Plots/plot-LFD-12} 
\begin{kframe}\begin{verbatim}
\end{verbatim}
\end{kframe}
\clearpage
\newpage
\begin{kframe}\begin{verbatim}The percentages of Pristipomoides multidens (ID #7, Lutjanidae) in 2016, n = 5,688
Immature (< 53cm): 31%
Small mature (>= 53cm, < 71cm): 63%
Large mature (>= 71cm): 7%
Mega spawner (>= 78.1cm): 1% (subset of large mature fish)
Spawning Potential Ratio: 7%
 
Between 30% and 50% of the fish in the catch are immature and have not had a chance
to reproduce before capture. The fishery is in immediate danger of overfishing
through overharvesting of juveniles, if fishing pressure is high.  Catching small and
immature fish needs to be actively avoided and a limit on overall fishing pressure is
warranted. Risk level is high.

The trade limit is significantly lower than the length at first maturity.  This means
that the trade encourages capture of immature fish, which impairs sustainability.
Risk level is high.

The vast majority of the fish in the catch have not yet achieved their growth
potential. The harvest of small fish promotes growth overfishing and the size
distribution for this species indicates that over exploitation through growth
overfishing may already be happening. Risk level is high.

Less than 20% of the catch comprises of mega spawners.  This indicates that the
population may be severely affected by the fishery, and that there is a substantial
risk of recruitment overfishing through over harvesting of the mega spawners, unless
large numbers of mega spawners would be surviving at other habitats. There is no
reason to assume that this is the case and therefore a reduction of fishing effort
may be necessary in this fishery. Risk level is high.
 
Mortality caused by fishing is greater than or equal to the natural rate of
mortality. This means that impact of fishing is severe and that fishing is unlikely
to be sustainable at the current level of effort. Risk level is high.
 
SPR is less than 25%. The fishery probably over-exploits the stock, and there is a
substantial risk that the fishery will cause severe decline of the stock if fishing
effort is not reduced. Risk level is high.
 
Trends in relative abundance by size group for Pristipomoides multidens (ID #7,
Lutjanidae), as calculated from linear regressions. The P value indicates the chance
that this calculated trend is merely a result of stochastic variance.
% Immature falling over recent years, situation improving. P: not available
% Large Mature rising over recent years, situation improving. P: not available
% Mega Spawner rising over recent years, situation improving. P: not available
% SPR rising over recent years, situation improving. P: not available
\end{verbatim}
\end{kframe}
\includegraphics[width=\maxwidth]{/root/R-project/IFishSnapperWPP714_715/Plots/plot-LFD-13} 

\includegraphics[width=\maxwidth]{/root/R-project/IFishSnapperWPP714_715/Plots/plot-LFD-14} 
\begin{kframe}\begin{verbatim}
\end{verbatim}
\end{kframe}
\clearpage
\newpage
\begin{kframe}\begin{verbatim}The percentages of Pristipomoides typus (ID #8, Lutjanidae) in 2016, n = 1,234
Immature (< 45cm): 10%
Small mature (>= 45cm, < 60cm): 57%
Large mature (>= 60cm): 33%
Mega spawner (>= 66cm): 14% (subset of large mature fish)
Spawning Potential Ratio: 15%
 
At least 90% of the fish in the catch are mature specimens that have spawned at least
once before they were caught. The fishery does not depend on immature size classes
for this species and is considered safe for this indicator. This fishery will not be
causing overfishing through over harvesting of juveniles for this species. Risk level
is low.

The trade limit is significantly lower than the length at first maturity.  This means
that the trade encourages capture of immature fish, which impairs sustainability.
Risk level is high.

The vast majority of the fish in the catch have not yet achieved their growth
potential. The harvest of small fish promotes growth overfishing and the size
distribution for this species indicates that over exploitation through growth
overfishing may already be happening. Risk level is high.

Less than 20% of the catch comprises of mega spawners.  This indicates that the
population may be severely affected by the fishery, and that there is a substantial
risk of recruitment overfishing through over harvesting of the mega spawners, unless
large numbers of mega spawners would be surviving at other habitats. There is no
reason to assume that this is the case and therefore a reduction of fishing effort
may be necessary in this fishery. Risk level is high.
 
Mortality caused by fishing is greater than or equal to the natural rate of
mortality. This means that impact of fishing is severe and that fishing is unlikely
to be sustainable at the current level of effort. Risk level is high.
 
SPR is less than 25%. The fishery probably over-exploits the stock, and there is a
substantial risk that the fishery will cause severe decline of the stock if fishing
effort is not reduced. Risk level is high.
 
Trends in relative abundance by size group for Pristipomoides typus (ID #8,
Lutjanidae), as calculated from linear regressions. The P value indicates the chance
that this calculated trend is merely a result of stochastic variance.
% Immature falling over recent years, situation improving. P: not available
% Large Mature rising over recent years, situation improving. P: not available
% Mega Spawner rising over recent years, situation improving. P: not available
% SPR rising over recent years, situation improving. P: not available
\end{verbatim}
\end{kframe}
\includegraphics[width=\maxwidth]{/root/R-project/IFishSnapperWPP714_715/Plots/plot-LFD-15} 

\includegraphics[width=\maxwidth]{/root/R-project/IFishSnapperWPP714_715/Plots/plot-LFD-16} 
\begin{kframe}\begin{verbatim}
\end{verbatim}
\end{kframe}
\clearpage
\newpage
\begin{kframe}\begin{verbatim}The percentages of Pristipomoides filamentosus (ID #9, Lutjanidae) in 2016, n = 11,690
Immature (< 48cm): 55%
Small mature (>= 48cm, < 64cm): 37%
Large mature (>= 64cm): 8%
Mega spawner (>= 70.4cm): 2% (subset of large mature fish)
Spawning Potential Ratio: 7%
 
The majority of the fish in the catch have not had a chance to reproduce before
capture. This fishery is most likely overfished already if fishing mortality is high
for all size classes in the population. An immediate shift away from targeting
juvenile fish and a reduction in overall fishing pressure is essential to prevent
collapse of the stock. Risk level is high.

The trade limit is significantly lower than the length at first maturity.  This means
that the trade encourages capture of immature fish, which impairs sustainability.
Risk level is high.

The vast majority of the fish in the catch have not yet achieved their growth
potential. The harvest of small fish promotes growth overfishing and the size
distribution for this species indicates that over exploitation through growth
overfishing may already be happening. Risk level is high.

Less than 20% of the catch comprises of mega spawners.  This indicates that the
population may be severely affected by the fishery, and that there is a substantial
risk of recruitment overfishing through over harvesting of the mega spawners, unless
large numbers of mega spawners would be surviving at other habitats. There is no
reason to assume that this is the case and therefore a reduction of fishing effort
may be necessary in this fishery. Risk level is high.
 
Mortality caused by fishing is greater than or equal to the natural rate of
mortality. This means that impact of fishing is severe and that fishing is unlikely
to be sustainable at the current level of effort. Risk level is high.
 
SPR is less than 25%. The fishery probably over-exploits the stock, and there is a
substantial risk that the fishery will cause severe decline of the stock if fishing
effort is not reduced. Risk level is high.
 
Trends in relative abundance by size group for Pristipomoides filamentosus (ID #9,
Lutjanidae), as calculated from linear regressions. The P value indicates the chance
that this calculated trend is merely a result of stochastic variance.
% Immature falling over recent years, situation improving. P: not available
% Large Mature rising over recent years, situation improving. P: not available
% Mega Spawner rising over recent years, situation improving. P: not available
% SPR rising over recent years, situation improving. P: not available
\end{verbatim}
\end{kframe}
\includegraphics[width=\maxwidth]{/root/R-project/IFishSnapperWPP714_715/Plots/plot-LFD-17} 

\includegraphics[width=\maxwidth]{/root/R-project/IFishSnapperWPP714_715/Plots/plot-LFD-18} 
\begin{kframe}\begin{verbatim}
\end{verbatim}
\end{kframe}
\clearpage
\newpage
\begin{kframe}\begin{verbatim}The percentages of Pristipomoides sieboldii (ID #10, Lutjanidae) in 2016, n = 996
Immature (< 32cm): 4%
Small mature (>= 32cm, < 42cm): 77%
Large mature (>= 42cm): 18%
Mega spawner (>= 46.2cm): 3% (subset of large mature fish)
Spawning Potential Ratio: 2%
 
At least 90% of the fish in the catch are mature specimens that have spawned at least
once before they were caught. The fishery does not depend on immature size classes
for this species and is considered safe for this indicator. This fishery will not be
causing overfishing through over harvesting of juveniles for this species. Risk level
is low.

The trade limit is about the same as the length at first maturity.  This means that
the trade puts a premium on fish that have spawned at least once, which improves
sustainability of the fishery. Risk level is medium.

The vast majority of the fish in the catch have not yet achieved their growth
potential. The harvest of small fish promotes growth overfishing and the size
distribution for this species indicates that over exploitation through growth
overfishing may already be happening. Risk level is high.

Less than 20% of the catch comprises of mega spawners.  This indicates that the
population may be severely affected by the fishery, and that there is a substantial
risk of recruitment overfishing through over harvesting of the mega spawners, unless
large numbers of mega spawners would be surviving at other habitats. There is no
reason to assume that this is the case and therefore a reduction of fishing effort
may be necessary in this fishery. Risk level is high.
 
Mortality caused by fishing is greater than or equal to the natural rate of
mortality. This means that impact of fishing is severe and that fishing is unlikely
to be sustainable at the current level of effort. Risk level is high.
 
SPR is less than 25%. The fishery probably over-exploits the stock, and there is a
substantial risk that the fishery will cause severe decline of the stock if fishing
effort is not reduced. Risk level is high.
 
Trends in relative abundance by size group for Pristipomoides sieboldii (ID #10,
Lutjanidae), as calculated from linear regressions. The P value indicates the chance
that this calculated trend is merely a result of stochastic variance.
% Immature falling over recent years, situation improving. P: not available
% Large Mature falling over recent years, situation deteriorating. P: not available
% Mega Spawner falling over recent years, situation deteriorating. P: not available
% SPR falling over recent years, situation deteriorating. P: not available
\end{verbatim}
\end{kframe}
\includegraphics[width=\maxwidth]{/root/R-project/IFishSnapperWPP714_715/Plots/plot-LFD-19} 

\includegraphics[width=\maxwidth]{/root/R-project/IFishSnapperWPP714_715/Plots/plot-LFD-20} 
\begin{kframe}\begin{verbatim}
\end{verbatim}
\end{kframe}
\clearpage
\newpage
\begin{kframe}\begin{verbatim}The percentages of Lutjanus argentimaculatus (ID #15, Lutjanidae) in 2016, n = 973
Immature (< 53cm): 9%
Small mature (>= 53cm, < 71cm): 68%
Large mature (>= 71cm): 23%
Mega spawner (>= 78.1cm): 8% (subset of large mature fish)
Spawning Potential Ratio: 17%
 
At least 90% of the fish in the catch are mature specimens that have spawned at least
once before they were caught. The fishery does not depend on immature size classes
for this species and is considered safe for this indicator. This fishery will not be
causing overfishing through over harvesting of juveniles for this species. Risk level
is low.

The trade limit is significantly lower than the length at first maturity.  This means
that the trade encourages capture of immature fish, which impairs sustainability.
Risk level is high.

The vast majority of the fish in the catch have not yet achieved their growth
potential. The harvest of small fish promotes growth overfishing and the size
distribution for this species indicates that over exploitation through growth
overfishing may already be happening. Risk level is high.

Less than 20% of the catch comprises of mega spawners.  This indicates that the
population may be severely affected by the fishery, and that there is a substantial
risk of recruitment overfishing through over harvesting of the mega spawners, unless
large numbers of mega spawners would be surviving at other habitats. There is no
reason to assume that this is the case and therefore a reduction of fishing effort
may be necessary in this fishery. Risk level is high.
 
Mortality caused by fishing is greater than or equal to the natural rate of
mortality. This means that impact of fishing is severe and that fishing is unlikely
to be sustainable at the current level of effort. Risk level is high.
 
SPR is less than 25%. The fishery probably over-exploits the stock, and there is a
substantial risk that the fishery will cause severe decline of the stock if fishing
effort is not reduced. Risk level is high.
 
Trends in relative abundance by size group for Lutjanus argentimaculatus (ID #15,
Lutjanidae), as calculated from linear regressions. The P value indicates the chance
that this calculated trend is merely a result of stochastic variance.
% Immature rising over recent years, situation deteriorating. P: not available
% Large Mature falling over recent years, situation deteriorating. P: not available
% Mega Spawner falling over recent years, situation deteriorating. P: not available
% SPR no trend over recent years, situation stable. P: not available
\end{verbatim}
\end{kframe}
\includegraphics[width=\maxwidth]{/root/R-project/IFishSnapperWPP714_715/Plots/plot-LFD-21} 

\includegraphics[width=\maxwidth]{/root/R-project/IFishSnapperWPP714_715/Plots/plot-LFD-22} 
\begin{kframe}\begin{verbatim}
\end{verbatim}
\end{kframe}
\clearpage
\newpage
\begin{kframe}\begin{verbatim}The percentages of Lutjanus bohar (ID #16, Lutjanidae) in 2016, n = 529
Immature (< 48cm): 28%
Small mature (>= 48cm, < 64cm): 50%
Large mature (>= 64cm): 22%
Mega spawner (>= 70.4cm): 8% (subset of large mature fish)
Spawning Potential Ratio: 15%
 
Between 20% and 30% of the fish in the catch are specimens that have not yet
reproduced. This is reason for concern in terms of potential overfishing through
overharvesting of juveniles, if fishing pressure is high and percentages immature
fish would further rise. Targeting larger fish and avoiding small fish in the catch
will promote a sustainable fishery. Risk level is medium.

The trade limit is significantly lower than the length at first maturity.  This means
that the trade encourages capture of immature fish, which impairs sustainability.
Risk level is high.

The vast majority of the fish in the catch have not yet achieved their growth
potential. The harvest of small fish promotes growth overfishing and the size
distribution for this species indicates that over exploitation through growth
overfishing may already be happening. Risk level is high.

Less than 20% of the catch comprises of mega spawners.  This indicates that the
population may be severely affected by the fishery, and that there is a substantial
risk of recruitment overfishing through over harvesting of the mega spawners, unless
large numbers of mega spawners would be surviving at other habitats. There is no
reason to assume that this is the case and therefore a reduction of fishing effort
may be necessary in this fishery. Risk level is high.
 
Mortality caused by fishing is greater than or equal to the natural rate of
mortality. This means that impact of fishing is severe and that fishing is unlikely
to be sustainable at the current level of effort. Risk level is high.
 
SPR is less than 25%. The fishery probably over-exploits the stock, and there is a
substantial risk that the fishery will cause severe decline of the stock if fishing
effort is not reduced. Risk level is high.
 
Trends in relative abundance by size group for Lutjanus bohar (ID #16, Lutjanidae),
as calculated from linear regressions. The P value indicates the chance that this
calculated trend is merely a result of stochastic variance.
% Immature no trend over recent years, situation stable. P: not available
% Large Mature rising over recent years, situation improving. P: not available
% Mega Spawner rising over recent years, situation improving. P: not available
% SPR falling over recent years, situation deteriorating. P: not available
\end{verbatim}
\end{kframe}
\includegraphics[width=\maxwidth]{/root/R-project/IFishSnapperWPP714_715/Plots/plot-LFD-23} 

\includegraphics[width=\maxwidth]{/root/R-project/IFishSnapperWPP714_715/Plots/plot-LFD-24} 
\begin{kframe}\begin{verbatim}
\end{verbatim}
\end{kframe}
\clearpage
\newpage
\begin{kframe}\begin{verbatim}The percentages of Lutjanus malabaricus (ID #17, Lutjanidae) in 2016, n = 615
Immature (< 53cm): 14%
Small mature (>= 53cm, < 71cm): 51%
Large mature (>= 71cm): 35%
Mega spawner (>= 78.1cm): 17% (subset of large mature fish)
Spawning Potential Ratio: 27%
 
Between 10% and 20% of the fish in the catch are juveniles that have not yet
reproduced. There is no immediate concern in terms of overfishing through over
harvesting of juveniles, but the fishery needs to be monitored closely for any
further increase in this indicator and incentives need to be geared towards targeting
larger fish. Risk level is medium.

The trade limit is significantly lower than the length at first maturity.  This means
that the trade encourages capture of immature fish, which impairs sustainability.
Risk level is high.

The vast majority of the fish in the catch have not yet achieved their growth
potential. The harvest of small fish promotes growth overfishing and the size
distribution for this species indicates that over exploitation through growth
overfishing may already be happening. Risk level is high.

Less than 20% of the catch comprises of mega spawners.  This indicates that the
population may be severely affected by the fishery, and that there is a substantial
risk of recruitment overfishing through over harvesting of the mega spawners, unless
large numbers of mega spawners would be surviving at other habitats. There is no
reason to assume that this is the case and therefore a reduction of fishing effort
may be necessary in this fishery. Risk level is high.
 
Mortality caused by fishing is greater than or equal to the natural rate of
mortality. This means that impact of fishing is severe and that fishing is unlikely
to be sustainable at the current level of effort. Risk level is high.
 
SPR is between 25% and 40%. The stock is heavily exploited, and there is some risk
that the fishery will cause further decline of the stock. Risk level is medium.
 
Trends in relative abundance by size group for Lutjanus malabaricus (ID #17,
Lutjanidae), as calculated from linear regressions. The P value indicates the chance
that this calculated trend is merely a result of stochastic variance.
% Immature no trend over recent years, situation stable. P: not available
% Large Mature falling over recent years, situation deteriorating. P: not available
% Mega Spawner falling over recent years, situation deteriorating. P: not available
% SPR falling over recent years, situation deteriorating. P: not available
\end{verbatim}
\end{kframe}
\includegraphics[width=\maxwidth]{/root/R-project/IFishSnapperWPP714_715/Plots/plot-LFD-25} 

\includegraphics[width=\maxwidth]{/root/R-project/IFishSnapperWPP714_715/Plots/plot-LFD-26} 
\begin{kframe}\begin{verbatim}
\end{verbatim}
\end{kframe}
\clearpage
\newpage
\begin{kframe}\begin{verbatim}The percentages of Lutjanus sebae (ID #18, Lutjanidae) in 2016, n = 426
Immature (< 53cm): 18%
Small mature (>= 53cm, < 71cm): 53%
Large mature (>= 71cm): 29%
Mega spawner (>= 78.1cm): 15% (subset of large mature fish)
Spawning Potential Ratio: 26%
 
Between 10% and 20% of the fish in the catch are juveniles that have not yet
reproduced. There is no immediate concern in terms of overfishing through over
harvesting of juveniles, but the fishery needs to be monitored closely for any
further increase in this indicator and incentives need to be geared towards targeting
larger fish. Risk level is medium.

The trade limit is significantly lower than the length at first maturity.  This means
that the trade encourages capture of immature fish, which impairs sustainability.
Risk level is high.

The vast majority of the fish in the catch have not yet achieved their growth
potential. The harvest of small fish promotes growth overfishing and the size
distribution for this species indicates that over exploitation through growth
overfishing may already be happening. Risk level is high.

Less than 20% of the catch comprises of mega spawners.  This indicates that the
population may be severely affected by the fishery, and that there is a substantial
risk of recruitment overfishing through over harvesting of the mega spawners, unless
large numbers of mega spawners would be surviving at other habitats. There is no
reason to assume that this is the case and therefore a reduction of fishing effort
may be necessary in this fishery. Risk level is high.
 
Mortality caused by fishing is greater than or equal to the natural rate of
mortality. This means that impact of fishing is severe and that fishing is unlikely
to be sustainable at the current level of effort. Risk level is high.
 
SPR is between 25% and 40%. The stock is heavily exploited, and there is some risk
that the fishery will cause further decline of the stock. Risk level is medium.
 
Trends in relative abundance by size group for Lutjanus sebae (ID #18, Lutjanidae),
as calculated from linear regressions. The P value indicates the chance that this
calculated trend is merely a result of stochastic variance.
% Immature rising over recent years, situation deteriorating. P: not available
% Large Mature falling over recent years, situation deteriorating. P: not available
% Mega Spawner falling over recent years, situation deteriorating. P: not available
% SPR falling over recent years, situation deteriorating. P: not available
\end{verbatim}
\end{kframe}
\includegraphics[width=\maxwidth]{/root/R-project/IFishSnapperWPP714_715/Plots/plot-LFD-27} 

\includegraphics[width=\maxwidth]{/root/R-project/IFishSnapperWPP714_715/Plots/plot-LFD-28} 
\begin{kframe}\begin{verbatim}
\end{verbatim}
\end{kframe}
\clearpage
\newpage
\begin{kframe}\begin{verbatim}The percentages of Lutjanus timorensis (ID #19, Lutjanidae) in 2016, n = 3,389
Immature (< 32cm): 9%
Small mature (>= 32cm, < 42cm): 69%
Large mature (>= 42cm): 22%
Mega spawner (>= 46.2cm): 7% (subset of large mature fish)
Spawning Potential Ratio: 7%
 
At least 90% of the fish in the catch are mature specimens that have spawned at least
once before they were caught. The fishery does not depend on immature size classes
for this species and is considered safe for this indicator. This fishery will not be
causing overfishing through over harvesting of juveniles for this species. Risk level
is low.

The trade limit is about the same as the length at first maturity.  This means that
the trade puts a premium on fish that have spawned at least once, which improves
sustainability of the fishery. Risk level is medium.

The vast majority of the fish in the catch have not yet achieved their growth
potential. The harvest of small fish promotes growth overfishing and the size
distribution for this species indicates that over exploitation through growth
overfishing may already be happening. Risk level is high.

Less than 20% of the catch comprises of mega spawners.  This indicates that the
population may be severely affected by the fishery, and that there is a substantial
risk of recruitment overfishing through over harvesting of the mega spawners, unless
large numbers of mega spawners would be surviving at other habitats. There is no
reason to assume that this is the case and therefore a reduction of fishing effort
may be necessary in this fishery. Risk level is high.
 
Mortality caused by fishing is greater than or equal to the natural rate of
mortality. This means that impact of fishing is severe and that fishing is unlikely
to be sustainable at the current level of effort. Risk level is high.
 
SPR is less than 25%. The fishery probably over-exploits the stock, and there is a
substantial risk that the fishery will cause severe decline of the stock if fishing
effort is not reduced. Risk level is high.
 
Trends in relative abundance by size group for Lutjanus timorensis (ID #19,
Lutjanidae), as calculated from linear regressions. The P value indicates the chance
that this calculated trend is merely a result of stochastic variance.
% Immature falling over recent years, situation improving. P: not available
% Large Mature rising over recent years, situation improving. P: not available
% Mega Spawner rising over recent years, situation improving. P: not available
% SPR rising over recent years, situation improving. P: not available
\end{verbatim}
\end{kframe}
\includegraphics[width=\maxwidth]{/root/R-project/IFishSnapperWPP714_715/Plots/plot-LFD-29} 

\includegraphics[width=\maxwidth]{/root/R-project/IFishSnapperWPP714_715/Plots/plot-LFD-30} 
\begin{kframe}\begin{verbatim}
\end{verbatim}
\end{kframe}
\clearpage
\newpage
\begin{kframe}\begin{verbatim}The percentages of Lutjanus gibbus (ID #20, Lutjanidae) in 2016, n = 788
Immature (< 27cm): 1%
Small mature (>= 27cm, < 35cm): 44%
Large mature (>= 35cm): 54%
Mega spawner (>= 38.5cm): 30% (subset of large mature fish)
Spawning Potential Ratio: 16%
 
At least 90% of the fish in the catch are mature specimens that have spawned at least
once before they were caught. The fishery does not depend on immature size classes
for this species and is considered safe for this indicator. This fishery will not be
causing overfishing through over harvesting of juveniles for this species. Risk level
is low.

The trade limit is significantly higher than length at first maturity.  This means
that the trade puts a premium on fish that have spawned at least once. The trade does
not cause any concern of recruitment overfishing for this species. Risk level is low.

The majority of the catch consists of size classes around or above the optimum
harvest size. This means that the impact of the fishery is minimized for this
species. Potentially higher yields of this species could be achieved by catching them
at somewhat smaller size, although capture of smaller specimen may take place already
in other fisheries. Risk level is low.

The percentage of mega spawners is between 20 and 30%.  There is no immediate reason
for concern, though fishing pressure may be significantly reducing the percentage of
mega-spawners, which may negatively affect the reproductive output of this
population. Risk level is medium.
 
Mortality caused by fishing is greater than or equal to the natural rate of
mortality. This means that impact of fishing is severe and that fishing is unlikely
to be sustainable at the current level of effort. Risk level is high.
 
SPR is less than 25%. The fishery probably over-exploits the stock, and there is a
substantial risk that the fishery will cause severe decline of the stock if fishing
effort is not reduced. Risk level is high.
 
Trends in relative abundance by size group for Lutjanus gibbus (ID #20, Lutjanidae),
as calculated from linear regressions. The P value indicates the chance that this
calculated trend is merely a result of stochastic variance.
% Immature falling over recent years, situation improving. P: not available
% Large Mature falling over recent years, situation deteriorating. P: not available
% Mega Spawner rising over recent years, situation improving. P: not available
% SPR falling over recent years, situation deteriorating. P: not available
\end{verbatim}
\end{kframe}
\includegraphics[width=\maxwidth]{/root/R-project/IFishSnapperWPP714_715/Plots/plot-LFD-31} 

\includegraphics[width=\maxwidth]{/root/R-project/IFishSnapperWPP714_715/Plots/plot-LFD-32} 
\begin{kframe}\begin{verbatim}
\end{verbatim}
\end{kframe}
\clearpage
\newpage
\begin{kframe}\begin{verbatim}The percentages of Lutjanus erythropterus (ID #21, Lutjanidae) in 2016, n = 4,350
Immature (< 37cm): 0%
Small mature (>= 37cm, < 49cm): 6%
Large mature (>= 49cm): 94%
Mega spawner (>= 53.9cm): 70% (subset of large mature fish)
Spawning Potential Ratio: 52%
 
At least 90% of the fish in the catch are mature specimens that have spawned at least
once before they were caught. The fishery does not depend on immature size classes
for this species and is considered safe for this indicator. This fishery will not be
causing overfishing through over harvesting of juveniles for this species. Risk level
is low.

The trade limit is significantly lower than the length at first maturity.  This means
that the trade encourages capture of immature fish, which impairs sustainability.
Risk level is high.

The majority of the catch consists of size classes around or above the optimum
harvest size. This means that the impact of the fishery is minimized for this
species. Potentially higher yields of this species could be achieved by catching them
at somewhat smaller size, although capture of smaller specimen may take place already
in other fisheries. Risk level is low.

More than 30% of the catch consists of mega spawners which indicates that this fish
population is in good health unless large amounts of much smaller fish from the same
population are caught by other fisheries. Risk level is low.
 
Mortality caused by fishing is greater than or equal to the natural rate of
mortality. This means that impact of fishing is severe and that fishing is unlikely
to be sustainable at the current level of effort. Risk level is high.
 
SPR is more than 40%. The stock is probably not over exploited, and the risk that the
fishery will cause further stock decline is small. Risk level is low.
 
Trends in relative abundance by size group for Lutjanus erythropterus (ID #21,
Lutjanidae), as calculated from linear regressions. The P value indicates the chance
that this calculated trend is merely a result of stochastic variance.
% Immature no trend over recent years, situation stable. P: not available
% Large Mature no trend over recent years, situation stable. P: not available
% Mega Spawner rising over recent years, situation improving. P: not available
% SPR rising over recent years, situation improving. P: not available
\end{verbatim}
\end{kframe}
\includegraphics[width=\maxwidth]{/root/R-project/IFishSnapperWPP714_715/Plots/plot-LFD-33} 

\includegraphics[width=\maxwidth]{/root/R-project/IFishSnapperWPP714_715/Plots/plot-LFD-34} 
\begin{kframe}\begin{verbatim}
\end{verbatim}
\end{kframe}
\clearpage
\newpage
\begin{kframe}\begin{verbatim}The percentages of Pinjalo lewisi (ID #22, Lutjanidae) in 2016, n = 2,281
Immature (< 29cm): 0%
Small mature (>= 29cm, < 39cm): 54%
Large mature (>= 39cm): 45%
Mega spawner (>= 42.9cm): 18% (subset of large mature fish)
Spawning Potential Ratio: 11%
 
At least 90% of the fish in the catch are mature specimens that have spawned at least
once before they were caught. The fishery does not depend on immature size classes
for this species and is considered safe for this indicator. This fishery will not be
causing overfishing through over harvesting of juveniles for this species. Risk level
is low.

The trade limit is about the same as the length at first maturity.  This means that
the trade puts a premium on fish that have spawned at least once, which improves
sustainability of the fishery. Risk level is medium.

The bulk of the catch includes age groups that have just matured and are about to
achieve their full growth potential. This indicates that the fishery is probably at
least being fully exploited. Risk level is medium.

Less than 20% of the catch comprises of mega spawners.  This indicates that the
population may be severely affected by the fishery, and that there is a substantial
risk of recruitment overfishing through over harvesting of the mega spawners, unless
large numbers of mega spawners would be surviving at other habitats. There is no
reason to assume that this is the case and therefore a reduction of fishing effort
may be necessary in this fishery. Risk level is high.
 
Mortality caused by fishing is greater than or equal to the natural rate of
mortality. This means that impact of fishing is severe and that fishing is unlikely
to be sustainable at the current level of effort. Risk level is high.
 
SPR is less than 25%. The fishery probably over-exploits the stock, and there is a
substantial risk that the fishery will cause severe decline of the stock if fishing
effort is not reduced. Risk level is high.
 
Trends in relative abundance by size group for Pinjalo lewisi (ID #22, Lutjanidae),
as calculated from linear regressions. The P value indicates the chance that this
calculated trend is merely a result of stochastic variance.
% Immature no trend over recent years, situation stable. P: not available
% Large Mature rising over recent years, situation improving. P: not available
% Mega Spawner rising over recent years, situation improving. P: not available
% SPR no trend over recent years, situation stable. P: not available
\end{verbatim}
\end{kframe}
\includegraphics[width=\maxwidth]{/root/R-project/IFishSnapperWPP714_715/Plots/plot-LFD-35} 

\includegraphics[width=\maxwidth]{/root/R-project/IFishSnapperWPP714_715/Plots/plot-LFD-36} 
\begin{kframe}\begin{verbatim}
\end{verbatim}
\end{kframe}
\clearpage
\newpage
\begin{kframe}\begin{verbatim}The percentages of Lutjanus vitta (ID #27, Lutjanidae) in 2016, n = 227
Immature (< 24cm): 4%
Small mature (>= 24cm, < 32cm): 37%
Large mature (>= 32cm): 59%
Mega spawner (>= 35.2cm): 29% (subset of large mature fish)
Spawning Potential Ratio: 28%
 
At least 90% of the fish in the catch are mature specimens that have spawned at least
once before they were caught. The fishery does not depend on immature size classes
for this species and is considered safe for this indicator. This fishery will not be
causing overfishing through over harvesting of juveniles for this species. Risk level
is low.

The trade limit is significantly higher than length at first maturity.  This means
that the trade puts a premium on fish that have spawned at least once. The trade does
not cause any concern of recruitment overfishing for this species. Risk level is low.

The majority of the catch consists of size classes around or above the optimum
harvest size. This means that the impact of the fishery is minimized for this
species. Potentially higher yields of this species could be achieved by catching them
at somewhat smaller size, although capture of smaller specimen may take place already
in other fisheries. Risk level is low.

The percentage of mega spawners is between 20 and 30%.  There is no immediate reason
for concern, though fishing pressure may be significantly reducing the percentage of
mega-spawners, which may negatively affect the reproductive output of this
population. Risk level is medium.
 
Mortality caused by fishing is greater than or equal to the natural rate of
mortality. This means that impact of fishing is severe and that fishing is unlikely
to be sustainable at the current level of effort. Risk level is high.
 
SPR is between 25% and 40%. The stock is heavily exploited, and there is some risk
that the fishery will cause further decline of the stock. Risk level is medium.
 
Trends in relative abundance by size group for Lutjanus vitta (ID #27, Lutjanidae),
as calculated from linear regressions. The P value indicates the chance that this
calculated trend is merely a result of stochastic variance.
% Immature no trend over recent years, situation stable. P: not available
% Large Mature no trend over recent years, situation stable. P: not available
% Mega Spawner no trend over recent years, situation stable. P: not available
% SPR no trend over recent years, situation stable. P: not available
\end{verbatim}
\end{kframe}
\includegraphics[width=\maxwidth]{/root/R-project/IFishSnapperWPP714_715/Plots/plot-LFD-37} 

\includegraphics[width=\maxwidth]{/root/R-project/IFishSnapperWPP714_715/Plots/plot-LFD-38} 
\begin{kframe}\begin{verbatim}
\end{verbatim}
\end{kframe}
\clearpage
\newpage
\begin{kframe}\begin{verbatim}The percentages of Symphorus nematophorus (ID #31, Lutjanidae) in 2016, n = 316
Immature (< 53cm): 0%
Small mature (>= 53cm, < 71cm): 8%
Large mature (>= 71cm): 92%
Mega spawner (>= 78.1cm): 63% (subset of large mature fish)
Spawning Potential Ratio: 53%
 
At least 90% of the fish in the catch are mature specimens that have spawned at least
once before they were caught. The fishery does not depend on immature size classes
for this species and is considered safe for this indicator. This fishery will not be
causing overfishing through over harvesting of juveniles for this species. Risk level
is low.

The trade limit is significantly lower than the length at first maturity.  This means
that the trade encourages capture of immature fish, which impairs sustainability.
Risk level is high.

The majority of the catch consists of size classes around or above the optimum
harvest size. This means that the impact of the fishery is minimized for this
species. Potentially higher yields of this species could be achieved by catching them
at somewhat smaller size, although capture of smaller specimen may take place already
in other fisheries. Risk level is low.

More than 30% of the catch consists of mega spawners which indicates that this fish
population is in good health unless large amounts of much smaller fish from the same
population are caught by other fisheries. Risk level is low.
 
Mortality caused by fishing is lower than the natural rate of mortality but more than
half of natural mortality. This means that impact of fishing is considerable and
trends in various indicators need to be watched carefully while any increase in
fishing effort needs to be prevented. Risk level is medium.
 
SPR is more than 40%. The stock is probably not over exploited, and the risk that the
fishery will cause further stock decline is small. Risk level is low.
 
Trends in relative abundance by size group for Symphorus nematophorus (ID #31,
Lutjanidae), as calculated from linear regressions. The P value indicates the chance
that this calculated trend is merely a result of stochastic variance.
% Immature no trend over recent years, situation stable. P: not available
% Large Mature falling over recent years, situation deteriorating. P: not available
% Mega Spawner falling over recent years, situation deteriorating. P: not available
% SPR falling over recent years, situation deteriorating. P: not available
\end{verbatim}
\end{kframe}
\includegraphics[width=\maxwidth]{/root/R-project/IFishSnapperWPP714_715/Plots/plot-LFD-39} 

\includegraphics[width=\maxwidth]{/root/R-project/IFishSnapperWPP714_715/Plots/plot-LFD-40} 
\begin{kframe}\begin{verbatim}
\end{verbatim}
\end{kframe}
\clearpage
\newpage
\begin{kframe}\begin{verbatim}The percentages of Paracaesio gonzalesi (ID #32, Lutjanidae) in 2016, n = 878
Immature (< 29cm): 3%
Small mature (>= 29cm, < 39cm): 59%
Large mature (>= 39cm): 37%
Mega spawner (>= 42.9cm): 9% (subset of large mature fish)
Spawning Potential Ratio: 2%
 
At least 90% of the fish in the catch are mature specimens that have spawned at least
once before they were caught. The fishery does not depend on immature size classes
for this species and is considered safe for this indicator. This fishery will not be
causing overfishing through over harvesting of juveniles for this species. Risk level
is low.

The trade limit is significantly lower than the length at first maturity.  This means
that the trade encourages capture of immature fish, which impairs sustainability.
Risk level is high.

The bulk of the catch includes age groups that have just matured and are about to
achieve their full growth potential. This indicates that the fishery is probably at
least being fully exploited. Risk level is medium.

Less than 20% of the catch comprises of mega spawners.  This indicates that the
population may be severely affected by the fishery, and that there is a substantial
risk of recruitment overfishing through over harvesting of the mega spawners, unless
large numbers of mega spawners would be surviving at other habitats. There is no
reason to assume that this is the case and therefore a reduction of fishing effort
may be necessary in this fishery. Risk level is high.
 
Mortality caused by fishing is greater than or equal to the natural rate of
mortality. This means that impact of fishing is severe and that fishing is unlikely
to be sustainable at the current level of effort. Risk level is high.
 
SPR is less than 25%. The fishery probably over-exploits the stock, and there is a
substantial risk that the fishery will cause severe decline of the stock if fishing
effort is not reduced. Risk level is high.
 
Trends in relative abundance by size group for Paracaesio gonzalesi (ID #32,
Lutjanidae), as calculated from linear regressions. The P value indicates the chance
that this calculated trend is merely a result of stochastic variance.
% Immature falling over recent years, situation improving. P: not available
% Large Mature rising over recent years, situation improving. P: not available
% Mega Spawner rising over recent years, situation improving. P: not available
% SPR falling over recent years, situation deteriorating. P: not available
\end{verbatim}
\end{kframe}
\includegraphics[width=\maxwidth]{/root/R-project/IFishSnapperWPP714_715/Plots/plot-LFD-41} 

\includegraphics[width=\maxwidth]{/root/R-project/IFishSnapperWPP714_715/Plots/plot-LFD-42} 
\begin{kframe}\begin{verbatim}
\end{verbatim}
\end{kframe}
\clearpage
\newpage
\begin{kframe}\begin{verbatim}The percentages of Paracaesio xanthura (ID #33, Lutjanidae) in 2016, n = 379
Immature (< 27cm): 1%
Small mature (>= 27cm, < 35cm): 26%
Large mature (>= 35cm): 73%
Mega spawner (>= 38.5cm): 44% (subset of large mature fish)
Spawning Potential Ratio: 56%
 
At least 90% of the fish in the catch are mature specimens that have spawned at least
once before they were caught. The fishery does not depend on immature size classes
for this species and is considered safe for this indicator. This fishery will not be
causing overfishing through over harvesting of juveniles for this species. Risk level
is low.

The trade limit is about the same as the length at first maturity.  This means that
the trade puts a premium on fish that have spawned at least once, which improves
sustainability of the fishery. Risk level is medium.

The majority of the catch consists of size classes around or above the optimum
harvest size. This means that the impact of the fishery is minimized for this
species. Potentially higher yields of this species could be achieved by catching them
at somewhat smaller size, although capture of smaller specimen may take place already
in other fisheries. Risk level is low.

More than 30% of the catch consists of mega spawners which indicates that this fish
population is in good health unless large amounts of much smaller fish from the same
population are caught by other fisheries. Risk level is low.
 
Mortality caused by fishing is at or below a level equal to half the natural rate of
mortality. This means that impact of fishing is minimized and this fishery is
currently probably operating at a sustainable level of effort. Risk level is low.
 
SPR is more than 40%. The stock is probably not over exploited, and the risk that the
fishery will cause further stock decline is small. Risk level is low.
 
Trends in relative abundance by size group for Paracaesio xanthura (ID #33,
Lutjanidae), as calculated from linear regressions. The P value indicates the chance
that this calculated trend is merely a result of stochastic variance.
% Immature no trend over recent years, situation stable. P: not available
% Large Mature rising over recent years, situation improving. P: not available
% Mega Spawner rising over recent years, situation improving. P: not available
% SPR rising over recent years, situation improving. P: not available
\end{verbatim}
\end{kframe}
\includegraphics[width=\maxwidth]{/root/R-project/IFishSnapperWPP714_715/Plots/plot-LFD-43} 

\includegraphics[width=\maxwidth]{/root/R-project/IFishSnapperWPP714_715/Plots/plot-LFD-44} 
\begin{kframe}\begin{verbatim}
\end{verbatim}
\end{kframe}
\clearpage
\newpage
\begin{kframe}\begin{verbatim}The percentages of Paracaesio kusakarii (ID #34, Lutjanidae) in 2016, n = 4,298
Immature (< 45cm): 13%
Small mature (>= 45cm, < 60cm): 77%
Large mature (>= 60cm): 10%
Mega spawner (>= 66cm): 3% (subset of large mature fish)
Spawning Potential Ratio: 6%
 
Between 10% and 20% of the fish in the catch are juveniles that have not yet
reproduced. There is no immediate concern in terms of overfishing through over
harvesting of juveniles, but the fishery needs to be monitored closely for any
further increase in this indicator and incentives need to be geared towards targeting
larger fish. Risk level is medium.

The trade limit is significantly lower than the length at first maturity.  This means
that the trade encourages capture of immature fish, which impairs sustainability.
Risk level is high.

The vast majority of the fish in the catch have not yet achieved their growth
potential. The harvest of small fish promotes growth overfishing and the size
distribution for this species indicates that over exploitation through growth
overfishing may already be happening. Risk level is high.

Less than 20% of the catch comprises of mega spawners.  This indicates that the
population may be severely affected by the fishery, and that there is a substantial
risk of recruitment overfishing through over harvesting of the mega spawners, unless
large numbers of mega spawners would be surviving at other habitats. There is no
reason to assume that this is the case and therefore a reduction of fishing effort
may be necessary in this fishery. Risk level is high.
 
Mortality caused by fishing is greater than or equal to the natural rate of
mortality. This means that impact of fishing is severe and that fishing is unlikely
to be sustainable at the current level of effort. Risk level is high.
 
SPR is less than 25%. The fishery probably over-exploits the stock, and there is a
substantial risk that the fishery will cause severe decline of the stock if fishing
effort is not reduced. Risk level is high.
 
Trends in relative abundance by size group for Paracaesio kusakarii (ID #34,
Lutjanidae), as calculated from linear regressions. The P value indicates the chance
that this calculated trend is merely a result of stochastic variance.
% Immature rising over recent years, situation deteriorating. P: not available
% Large Mature falling over recent years, situation deteriorating. P: not available
% Mega Spawner no trend over recent years, situation stable. P: not available
% SPR falling over recent years, situation deteriorating. P: not available
\end{verbatim}
\end{kframe}
\includegraphics[width=\maxwidth]{/root/R-project/IFishSnapperWPP714_715/Plots/plot-LFD-45} 

\includegraphics[width=\maxwidth]{/root/R-project/IFishSnapperWPP714_715/Plots/plot-LFD-46} 
\begin{kframe}\begin{verbatim}
\end{verbatim}
\end{kframe}
\clearpage
\newpage
\begin{kframe}\begin{verbatim}The percentages of Paracaesio stonei (ID #35, Lutjanidae) in 2016, n = 1,665
Immature (< 37cm): 9%
Small mature (>= 37cm, < 49cm): 71%
Large mature (>= 49cm): 20%
Mega spawner (>= 53.9cm): 6% (subset of large mature fish)
Spawning Potential Ratio: 8%
 
At least 90% of the fish in the catch are mature specimens that have spawned at least
once before they were caught. The fishery does not depend on immature size classes
for this species and is considered safe for this indicator. This fishery will not be
causing overfishing through over harvesting of juveniles for this species. Risk level
is low.

The trade limit is significantly lower than the length at first maturity.  This means
that the trade encourages capture of immature fish, which impairs sustainability.
Risk level is high.

The vast majority of the fish in the catch have not yet achieved their growth
potential. The harvest of small fish promotes growth overfishing and the size
distribution for this species indicates that over exploitation through growth
overfishing may already be happening. Risk level is high.

Less than 20% of the catch comprises of mega spawners.  This indicates that the
population may be severely affected by the fishery, and that there is a substantial
risk of recruitment overfishing through over harvesting of the mega spawners, unless
large numbers of mega spawners would be surviving at other habitats. There is no
reason to assume that this is the case and therefore a reduction of fishing effort
may be necessary in this fishery. Risk level is high.
 
Mortality caused by fishing is greater than or equal to the natural rate of
mortality. This means that impact of fishing is severe and that fishing is unlikely
to be sustainable at the current level of effort. Risk level is high.
 
SPR is less than 25%. The fishery probably over-exploits the stock, and there is a
substantial risk that the fishery will cause severe decline of the stock if fishing
effort is not reduced. Risk level is high.
 
Trends in relative abundance by size group for Paracaesio stonei (ID #35,
Lutjanidae), as calculated from linear regressions. The P value indicates the chance
that this calculated trend is merely a result of stochastic variance.
% Immature falling over recent years, situation improving. P: not available
% Large Mature rising over recent years, situation improving. P: not available
% Mega Spawner rising over recent years, situation improving. P: not available
% SPR rising over recent years, situation improving. P: not available
\end{verbatim}
\end{kframe}
\includegraphics[width=\maxwidth]{/root/R-project/IFishSnapperWPP714_715/Plots/plot-LFD-47} 

\includegraphics[width=\maxwidth]{/root/R-project/IFishSnapperWPP714_715/Plots/plot-LFD-48} 
\begin{kframe}\begin{verbatim}
\end{verbatim}
\end{kframe}
\clearpage
\newpage
\begin{kframe}\begin{verbatim}The percentages of Cephalopholis sonnerati (ID #39, Epinephelidae) in 2016, n = 635
Immature (< 23cm): 3%
Small mature (>= 23cm, < 30cm): 16%
Large mature (>= 30cm): 81%
Mega spawner (>= 33cm): 75% (subset of large mature fish)
Spawning Potential Ratio: near 100%
 
At least 90% of the fish in the catch are mature specimens that have spawned at least
once before they were caught. The fishery does not depend on immature size classes
for this species and is considered safe for this indicator. This fishery will not be
causing overfishing through over harvesting of juveniles for this species. Risk level
is low.

The trade limit is significantly higher than length at first maturity.  This means
that the trade puts a premium on fish that have spawned at least once. The trade does
not cause any concern of recruitment overfishing for this species. Risk level is low.

The majority of the catch consists of size classes around or above the optimum
harvest size. This means that the impact of the fishery is minimized for this
species. Potentially higher yields of this species could be achieved by catching them
at somewhat smaller size, although capture of smaller specimen may take place already
in other fisheries. Risk level is low.

More than 30% of the catch consists of mega spawners which indicates that this fish
population is in good health unless large amounts of much smaller fish from the same
population are caught by other fisheries. Risk level is low.
 
Mortality caused by fishing is at or below a level equal to half the natural rate of
mortality. This means that impact of fishing is minimized and this fishery is
currently probably operating at a sustainable level of effort. Risk level is low.
 
SPR is more than 40%. The stock is probably not over exploited, and the risk that the
fishery will cause further stock decline is small. Risk level is low.
 
Trends in relative abundance by size group for Cephalopholis sonnerati (ID #39,
Epinephelidae), as calculated from linear regressions. The P value indicates the
chance that this calculated trend is merely a result of stochastic variance.
% Immature rising over recent years, situation deteriorating. P: not available
% Large Mature falling over recent years, situation deteriorating. P: not available
% Mega Spawner falling over recent years, situation deteriorating. P: not available
% SPR falling over recent years, situation deteriorating. P: not available
\end{verbatim}
\end{kframe}
\includegraphics[width=\maxwidth]{/root/R-project/IFishSnapperWPP714_715/Plots/plot-LFD-49} 

\includegraphics[width=\maxwidth]{/root/R-project/IFishSnapperWPP714_715/Plots/plot-LFD-50} 
\begin{kframe}\begin{verbatim}
\end{verbatim}
\end{kframe}
\clearpage
\newpage
\begin{kframe}\begin{verbatim}The percentages of Epinephelus radiatus (ID #42, Epinephelidae) in 2016, n = 211
Immature (< 31cm): 2%
Small mature (>= 31cm, < 41cm): 23%
Large mature (>= 41cm): 75%
Mega spawner (>= 45.1cm): 55% (subset of large mature fish)
Spawning Potential Ratio: 70%
 
At least 90% of the fish in the catch are mature specimens that have spawned at least
once before they were caught. The fishery does not depend on immature size classes
for this species and is considered safe for this indicator. This fishery will not be
causing overfishing through over harvesting of juveniles for this species. Risk level
is low.

The trade limit is significantly lower than the length at first maturity.  This means
that the trade encourages capture of immature fish, which impairs sustainability.
Risk level is high.

The majority of the catch consists of size classes around or above the optimum
harvest size. This means that the impact of the fishery is minimized for this
species. Potentially higher yields of this species could be achieved by catching them
at somewhat smaller size, although capture of smaller specimen may take place already
in other fisheries. Risk level is low.

More than 30% of the catch consists of mega spawners which indicates that this fish
population is in good health unless large amounts of much smaller fish from the same
population are caught by other fisheries. Risk level is low.
 
Mortality caused by fishing is at or below a level equal to half the natural rate of
mortality. This means that impact of fishing is minimized and this fishery is
currently probably operating at a sustainable level of effort. Risk level is low.
 
SPR is more than 40%. The stock is probably not over exploited, and the risk that the
fishery will cause further stock decline is small. Risk level is low.
 
Trends in relative abundance by size group for Epinephelus radiatus (ID #42,
Epinephelidae), as calculated from linear regressions. The P value indicates the
chance that this calculated trend is merely a result of stochastic variance.
% Immature falling over recent years, situation improving. P: not available
% Large Mature rising over recent years, situation improving. P: not available
% Mega Spawner rising over recent years, situation improving. P: not available
% SPR falling over recent years, situation deteriorating. P: not available
\end{verbatim}
\end{kframe}
\includegraphics[width=\maxwidth]{/root/R-project/IFishSnapperWPP714_715/Plots/plot-LFD-51} 

\includegraphics[width=\maxwidth]{/root/R-project/IFishSnapperWPP714_715/Plots/plot-LFD-52} 
\begin{kframe}\begin{verbatim}
\end{verbatim}
\end{kframe}
\clearpage
\newpage
\begin{kframe}\begin{verbatim}The percentages of Epinephelus morrhua (ID #43, Epinephelidae) in 2016, n = 724
Immature (< 31cm): 10%
Small mature (>= 31cm, < 41cm): 49%
Large mature (>= 41cm): 41%
Mega spawner (>= 45.1cm): 22% (subset of large mature fish)
Spawning Potential Ratio: 32%
 
Between 10% and 20% of the fish in the catch are juveniles that have not yet
reproduced. There is no immediate concern in terms of overfishing through over
harvesting of juveniles, but the fishery needs to be monitored closely for any
further increase in this indicator and incentives need to be geared towards targeting
larger fish. Risk level is medium.

The trade limit is significantly lower than the length at first maturity.  This means
that the trade encourages capture of immature fish, which impairs sustainability.
Risk level is high.

The bulk of the catch includes age groups that have just matured and are about to
achieve their full growth potential. This indicates that the fishery is probably at
least being fully exploited. Risk level is medium.

The percentage of mega spawners is between 20 and 30%.  There is no immediate reason
for concern, though fishing pressure may be significantly reducing the percentage of
mega-spawners, which may negatively affect the reproductive output of this
population. Risk level is medium.
 
Mortality caused by fishing is greater than or equal to the natural rate of
mortality. This means that impact of fishing is severe and that fishing is unlikely
to be sustainable at the current level of effort. Risk level is high.
 
SPR is between 25% and 40%. The stock is heavily exploited, and there is some risk
that the fishery will cause further decline of the stock. Risk level is medium.
 
Trends in relative abundance by size group for Epinephelus morrhua (ID #43,
Epinephelidae), as calculated from linear regressions. The P value indicates the
chance that this calculated trend is merely a result of stochastic variance.
% Immature rising over recent years, situation deteriorating. P: not available
% Large Mature falling over recent years, situation deteriorating. P: not available
% Mega Spawner rising over recent years, situation improving. P: not available
% SPR falling over recent years, situation deteriorating. P: not available
\end{verbatim}
\end{kframe}
\includegraphics[width=\maxwidth]{/root/R-project/IFishSnapperWPP714_715/Plots/plot-LFD-53} 

\includegraphics[width=\maxwidth]{/root/R-project/IFishSnapperWPP714_715/Plots/plot-LFD-54} 
\begin{kframe}\begin{verbatim}
\end{verbatim}
\end{kframe}
\clearpage
\newpage
\begin{kframe}\begin{verbatim}The percentages of Epinephelus areolatus (ID #45, Epinephelidae) in 2016, n = 1,438
Immature (< 21cm): 0%
Small mature (>= 21cm, < 28cm): 14%
Large mature (>= 28cm): 86%
Mega spawner (>= 30.8cm): 71% (subset of large mature fish)
Spawning Potential Ratio: 59%
 
At least 90% of the fish in the catch are mature specimens that have spawned at least
once before they were caught. The fishery does not depend on immature size classes
for this species and is considered safe for this indicator. This fishery will not be
causing overfishing through over harvesting of juveniles for this species. Risk level
is low.

The trade limit is significantly higher than length at first maturity.  This means
that the trade puts a premium on fish that have spawned at least once. The trade does
not cause any concern of recruitment overfishing for this species. Risk level is low.

The majority of the catch consists of size classes around or above the optimum
harvest size. This means that the impact of the fishery is minimized for this
species. Potentially higher yields of this species could be achieved by catching them
at somewhat smaller size, although capture of smaller specimen may take place already
in other fisheries. Risk level is low.

More than 30% of the catch consists of mega spawners which indicates that this fish
population is in good health unless large amounts of much smaller fish from the same
population are caught by other fisheries. Risk level is low.
 
Mortality caused by fishing is lower than the natural rate of mortality but more than
half of natural mortality. This means that impact of fishing is considerable and
trends in various indicators need to be watched carefully while any increase in
fishing effort needs to be prevented. Risk level is medium.
 
SPR is more than 40%. The stock is probably not over exploited, and the risk that the
fishery will cause further stock decline is small. Risk level is low.
 
Trends in relative abundance by size group for Epinephelus areolatus (ID #45,
Epinephelidae), as calculated from linear regressions. The P value indicates the
chance that this calculated trend is merely a result of stochastic variance.
% Immature no trend over recent years, situation stable. P: not available
% Large Mature falling over recent years, situation deteriorating. P: not available
% Mega Spawner falling over recent years, situation deteriorating. P: not available
% SPR falling over recent years, situation deteriorating. P: not available
\end{verbatim}
\end{kframe}
\includegraphics[width=\maxwidth]{/root/R-project/IFishSnapperWPP714_715/Plots/plot-LFD-55} 

\includegraphics[width=\maxwidth]{/root/R-project/IFishSnapperWPP714_715/Plots/plot-LFD-56} 
\begin{kframe}\begin{verbatim}
\end{verbatim}
\end{kframe}
\clearpage
\newpage
\begin{kframe}\begin{verbatim}The percentages of Epinephelus bilobatus (ID #48, Epinephelidae) in 2016, n = 214
Immature (< 23cm): 0%
Small mature (>= 23cm, < 30cm): 2%
Large mature (>= 30cm): 98%
Mega spawner (>= 33cm): 97% (subset of large mature fish)
Spawning Potential Ratio: 93%
 
At least 90% of the fish in the catch are mature specimens that have spawned at least
once before they were caught. The fishery does not depend on immature size classes
for this species and is considered safe for this indicator. This fishery will not be
causing overfishing through over harvesting of juveniles for this species. Risk level
is low.

The trade limit is significantly higher than length at first maturity.  This means
that the trade puts a premium on fish that have spawned at least once. The trade does
not cause any concern of recruitment overfishing for this species. Risk level is low.

The majority of the catch consists of size classes around or above the optimum
harvest size. This means that the impact of the fishery is minimized for this
species. Potentially higher yields of this species could be achieved by catching them
at somewhat smaller size, although capture of smaller specimen may take place already
in other fisheries. Risk level is low.

More than 30% of the catch consists of mega spawners which indicates that this fish
population is in good health unless large amounts of much smaller fish from the same
population are caught by other fisheries. Risk level is low.
 
Mortality caused by fishing is at or below a level equal to half the natural rate of
mortality. This means that impact of fishing is minimized and this fishery is
currently probably operating at a sustainable level of effort. Risk level is low.
 
SPR is more than 40%. The stock is probably not over exploited, and the risk that the
fishery will cause further stock decline is small. Risk level is low.
 
Trends in relative abundance by size group for Epinephelus bilobatus (ID #48,
Epinephelidae), as calculated from linear regressions. The P value indicates the
chance that this calculated trend is merely a result of stochastic variance.
% Immature no trend over recent years, situation stable. P: not available
% Large Mature falling over recent years, situation deteriorating. P: not available
% Mega Spawner falling over recent years, situation deteriorating. P: not available
% SPR falling over recent years, situation deteriorating. P: not available
\end{verbatim}
\end{kframe}
\includegraphics[width=\maxwidth]{/root/R-project/IFishSnapperWPP714_715/Plots/plot-LFD-57} 

\includegraphics[width=\maxwidth]{/root/R-project/IFishSnapperWPP714_715/Plots/plot-LFD-58} 
\begin{kframe}\begin{verbatim}
\end{verbatim}
\end{kframe}
\clearpage
\newpage
\begin{kframe}\begin{verbatim}The percentages of Epinephelus coioides (ID #50, Epinephelidae) in 2016, n = 258
Immature (< 50cm): 2%
Small mature (>= 50cm, < 66cm): 22%
Large mature (>= 66cm): 76%
Mega spawner (>= 72.6cm): 56% (subset of large mature fish)
Spawning Potential Ratio: 83%
 
At least 90% of the fish in the catch are mature specimens that have spawned at least
once before they were caught. The fishery does not depend on immature size classes
for this species and is considered safe for this indicator. This fishery will not be
causing overfishing through over harvesting of juveniles for this species. Risk level
is low.

The trade limit is about the same as the length at first maturity.  This means that
the trade puts a premium on fish that have spawned at least once, which improves
sustainability of the fishery. Risk level is medium.

The majority of the catch consists of size classes around or above the optimum
harvest size. This means that the impact of the fishery is minimized for this
species. Potentially higher yields of this species could be achieved by catching them
at somewhat smaller size, although capture of smaller specimen may take place already
in other fisheries. Risk level is low.

More than 30% of the catch consists of mega spawners which indicates that this fish
population is in good health unless large amounts of much smaller fish from the same
population are caught by other fisheries. Risk level is low.
 
Mortality caused by fishing is at or below a level equal to half the natural rate of
mortality. This means that impact of fishing is minimized and this fishery is
currently probably operating at a sustainable level of effort. Risk level is low.
 
SPR is more than 40%. The stock is probably not over exploited, and the risk that the
fishery will cause further stock decline is small. Risk level is low.
 
Trends in relative abundance by size group for Epinephelus coioides (ID #50,
Epinephelidae), as calculated from linear regressions. The P value indicates the
chance that this calculated trend is merely a result of stochastic variance.
% Immature falling over recent years, situation improving. P: not available
% Large Mature falling over recent years, situation deteriorating. P: not available
% Mega Spawner falling over recent years, situation deteriorating. P: not available
% SPR falling over recent years, situation deteriorating. P: not available
\end{verbatim}
\end{kframe}
\includegraphics[width=\maxwidth]{/root/R-project/IFishSnapperWPP714_715/Plots/plot-LFD-59} 

\includegraphics[width=\maxwidth]{/root/R-project/IFishSnapperWPP714_715/Plots/plot-LFD-60} 
\begin{kframe}\begin{verbatim}
\end{verbatim}
\end{kframe}
\clearpage
\newpage
\begin{kframe}\begin{verbatim}The percentages of Epinephelus chlorostigma (ID #51, Epinephelidae) in 2016, n = 412
Immature (< 31cm): 5%
Small mature (>= 31cm, < 41cm): 26%
Large mature (>= 41cm): 68%
Mega spawner (>= 45.1cm): 49% (subset of large mature fish)
Spawning Potential Ratio: 46%
 
At least 90% of the fish in the catch are mature specimens that have spawned at least
once before they were caught. The fishery does not depend on immature size classes
for this species and is considered safe for this indicator. This fishery will not be
causing overfishing through over harvesting of juveniles for this species. Risk level
is low.

The trade limit is significantly higher than length at first maturity.  This means
that the trade puts a premium on fish that have spawned at least once. The trade does
not cause any concern of recruitment overfishing for this species. Risk level is low.

The majority of the catch consists of size classes around or above the optimum
harvest size. This means that the impact of the fishery is minimized for this
species. Potentially higher yields of this species could be achieved by catching them
at somewhat smaller size, although capture of smaller specimen may take place already
in other fisheries. Risk level is low.

More than 30% of the catch consists of mega spawners which indicates that this fish
population is in good health unless large amounts of much smaller fish from the same
population are caught by other fisheries. Risk level is low.
 
Mortality caused by fishing is lower than the natural rate of mortality but more than
half of natural mortality. This means that impact of fishing is considerable and
trends in various indicators need to be watched carefully while any increase in
fishing effort needs to be prevented. Risk level is medium.
 
SPR is more than 40%. The stock is probably not over exploited, and the risk that the
fishery will cause further stock decline is small. Risk level is low.
 
Trends in relative abundance by size group for Epinephelus chlorostigma (ID #51,
Epinephelidae), as calculated from linear regressions. The P value indicates the
chance that this calculated trend is merely a result of stochastic variance.
% Immature falling over recent years, situation improving. P: not available
% Large Mature no trend over recent years, situation stable. P: not available
% Mega Spawner falling over recent years, situation deteriorating. P: not available
% SPR rising over recent years, situation improving. P: not available
\end{verbatim}
\end{kframe}
\includegraphics[width=\maxwidth]{/root/R-project/IFishSnapperWPP714_715/Plots/plot-LFD-61} 

\includegraphics[width=\maxwidth]{/root/R-project/IFishSnapperWPP714_715/Plots/plot-LFD-62} 
\begin{kframe}\begin{verbatim}
\end{verbatim}
\end{kframe}
\clearpage
\newpage
\begin{kframe}\begin{verbatim}The percentages of Epinephelus stictus (ID #54, Epinephelidae) in 2016, n = 373
Immature (< 21cm): 0%
Small mature (>= 21cm, < 28cm): 6%
Large mature (>= 28cm): 94%
Mega spawner (>= 30.8cm): 88% (subset of large mature fish)
Spawning Potential Ratio: near 100%
 
At least 90% of the fish in the catch are mature specimens that have spawned at least
once before they were caught. The fishery does not depend on immature size classes
for this species and is considered safe for this indicator. This fishery will not be
causing overfishing through over harvesting of juveniles for this species. Risk level
is low.

The trade limit is significantly higher than length at first maturity.  This means
that the trade puts a premium on fish that have spawned at least once. The trade does
not cause any concern of recruitment overfishing for this species. Risk level is low.

The majority of the catch consists of size classes around or above the optimum
harvest size. This means that the impact of the fishery is minimized for this
species. Potentially higher yields of this species could be achieved by catching them
at somewhat smaller size, although capture of smaller specimen may take place already
in other fisheries. Risk level is low.

More than 30% of the catch consists of mega spawners which indicates that this fish
population is in good health unless large amounts of much smaller fish from the same
population are caught by other fisheries. Risk level is low.
 
Mortality caused by fishing is at or below a level equal to half the natural rate of
mortality. This means that impact of fishing is minimized and this fishery is
currently probably operating at a sustainable level of effort. Risk level is low.
 
SPR is more than 40%. The stock is probably not over exploited, and the risk that the
fishery will cause further stock decline is small. Risk level is low.
 
Trends in relative abundance by size group for Epinephelus stictus (ID #54,
Epinephelidae), as calculated from linear regressions. The P value indicates the
chance that this calculated trend is merely a result of stochastic variance.
% Immature no trend over recent years, situation stable. P: not available
% Large Mature rising over recent years, situation improving. P: not available
% Mega Spawner rising over recent years, situation improving. P: not available
% SPR falling over recent years, situation deteriorating. P: not available
\end{verbatim}
\end{kframe}
\includegraphics[width=\maxwidth]{/root/R-project/IFishSnapperWPP714_715/Plots/plot-LFD-63} 

\includegraphics[width=\maxwidth]{/root/R-project/IFishSnapperWPP714_715/Plots/plot-LFD-64} 
\begin{kframe}\begin{verbatim}
\end{verbatim}
\end{kframe}
\clearpage
\newpage
\begin{kframe}\begin{verbatim}The percentages of Epinephelus undulosus (ID #57, Epinephelidae) in 2016, n = 243
Immature (< 33cm): 0%
Small mature (>= 33cm, < 44cm): 3%
Large mature (>= 44cm): 97%
Mega spawner (>= 48.4cm): 93% (subset of large mature fish)
Spawning Potential Ratio: near 100%
 
At least 90% of the fish in the catch are mature specimens that have spawned at least
once before they were caught. The fishery does not depend on immature size classes
for this species and is considered safe for this indicator. This fishery will not be
causing overfishing through over harvesting of juveniles for this species. Risk level
is low.

The trade limit is significantly higher than length at first maturity.  This means
that the trade puts a premium on fish that have spawned at least once. The trade does
not cause any concern of recruitment overfishing for this species. Risk level is low.

The majority of the catch consists of size classes around or above the optimum
harvest size. This means that the impact of the fishery is minimized for this
species. Potentially higher yields of this species could be achieved by catching them
at somewhat smaller size, although capture of smaller specimen may take place already
in other fisheries. Risk level is low.

More than 30% of the catch consists of mega spawners which indicates that this fish
population is in good health unless large amounts of much smaller fish from the same
population are caught by other fisheries. Risk level is low.
 
Mortality caused by fishing is at or below a level equal to half the natural rate of
mortality. This means that impact of fishing is minimized and this fishery is
currently probably operating at a sustainable level of effort. Risk level is low.
 
SPR is more than 40%. The stock is probably not over exploited, and the risk that the
fishery will cause further stock decline is small. Risk level is low.
 
Trends in relative abundance by size group for Epinephelus undulosus (ID #57,
Epinephelidae), as calculated from linear regressions. The P value indicates the
chance that this calculated trend is merely a result of stochastic variance.
% Immature no trend over recent years, situation stable. P: not available
% Large Mature rising over recent years, situation improving. P: not available
% Mega Spawner rising over recent years, situation improving. P: not available
% SPR falling over recent years, situation deteriorating. P: not available
\end{verbatim}
\end{kframe}
\includegraphics[width=\maxwidth]{/root/R-project/IFishSnapperWPP714_715/Plots/plot-LFD-65} 

\includegraphics[width=\maxwidth]{/root/R-project/IFishSnapperWPP714_715/Plots/plot-LFD-66} 
\begin{kframe}\begin{verbatim}
\end{verbatim}
\end{kframe}
\clearpage
\newpage
\begin{kframe}\begin{verbatim}The percentages of Epinephelus amblycephalus (ID #58, Epinephelidae) in 2016, n = 449
Immature (< 33cm): 1%
Small mature (>= 33cm, < 44cm): 12%
Large mature (>= 44cm): 86%
Mega spawner (>= 48.4cm): 80% (subset of large mature fish)
Spawning Potential Ratio: near 100%
 
At least 90% of the fish in the catch are mature specimens that have spawned at least
once before they were caught. The fishery does not depend on immature size classes
for this species and is considered safe for this indicator. This fishery will not be
causing overfishing through over harvesting of juveniles for this species. Risk level
is low.

The trade limit is significantly higher than length at first maturity.  This means
that the trade puts a premium on fish that have spawned at least once. The trade does
not cause any concern of recruitment overfishing for this species. Risk level is low.

The majority of the catch consists of size classes around or above the optimum
harvest size. This means that the impact of the fishery is minimized for this
species. Potentially higher yields of this species could be achieved by catching them
at somewhat smaller size, although capture of smaller specimen may take place already
in other fisheries. Risk level is low.

More than 30% of the catch consists of mega spawners which indicates that this fish
population is in good health unless large amounts of much smaller fish from the same
population are caught by other fisheries. Risk level is low.
 
Mortality caused by fishing is at or below a level equal to half the natural rate of
mortality. This means that impact of fishing is minimized and this fishery is
currently probably operating at a sustainable level of effort. Risk level is low.
 
SPR is more than 40%. The stock is probably not over exploited, and the risk that the
fishery will cause further stock decline is small. Risk level is low.
 
Trends in relative abundance by size group for Epinephelus amblycephalus (ID #58,
Epinephelidae), as calculated from linear regressions. The P value indicates the
chance that this calculated trend is merely a result of stochastic variance.
% Immature falling over recent years, situation improving. P: not available
% Large Mature rising over recent years, situation improving. P: not available
% Mega Spawner rising over recent years, situation improving. P: not available
% SPR falling over recent years, situation deteriorating. P: not available
\end{verbatim}
\end{kframe}
\includegraphics[width=\maxwidth]{/root/R-project/IFishSnapperWPP714_715/Plots/plot-LFD-67} 

\includegraphics[width=\maxwidth]{/root/R-project/IFishSnapperWPP714_715/Plots/plot-LFD-68} 
\begin{kframe}\begin{verbatim}
\end{verbatim}
\end{kframe}
\clearpage
\newpage
\begin{kframe}\begin{verbatim}The percentages of Variola albimarginata (ID #62, Epinephelidae) in 2016, n = 270
Immature (< 25cm): 1%
Small mature (>= 25cm, < 33cm): 29%
Large mature (>= 33cm): 70%
Mega spawner (>= 36.3cm): 41% (subset of large mature fish)
Spawning Potential Ratio: 19%
 
At least 90% of the fish in the catch are mature specimens that have spawned at least
once before they were caught. The fishery does not depend on immature size classes
for this species and is considered safe for this indicator. This fishery will not be
causing overfishing through over harvesting of juveniles for this species. Risk level
is low.

The trade limit is significantly higher than length at first maturity.  This means
that the trade puts a premium on fish that have spawned at least once. The trade does
not cause any concern of recruitment overfishing for this species. Risk level is low.

The majority of the catch consists of size classes around or above the optimum
harvest size. This means that the impact of the fishery is minimized for this
species. Potentially higher yields of this species could be achieved by catching them
at somewhat smaller size, although capture of smaller specimen may take place already
in other fisheries. Risk level is low.

More than 30% of the catch consists of mega spawners which indicates that this fish
population is in good health unless large amounts of much smaller fish from the same
population are caught by other fisheries. Risk level is low.
 
Mortality caused by fishing is greater than or equal to the natural rate of
mortality. This means that impact of fishing is severe and that fishing is unlikely
to be sustainable at the current level of effort. Risk level is high.
 
SPR is less than 25%. The fishery probably over-exploits the stock, and there is a
substantial risk that the fishery will cause severe decline of the stock if fishing
effort is not reduced. Risk level is high.
 
Trends in relative abundance by size group for Variola albimarginata (ID #62,
Epinephelidae), as calculated from linear regressions. The P value indicates the
chance that this calculated trend is merely a result of stochastic variance.
% Immature no trend over recent years, situation stable. P: not available
% Large Mature rising over recent years, situation improving. P: not available
% Mega Spawner rising over recent years, situation improving. P: not available
% SPR falling over recent years, situation deteriorating. P: not available
\end{verbatim}
\end{kframe}
\includegraphics[width=\maxwidth]{/root/R-project/IFishSnapperWPP714_715/Plots/plot-LFD-69} 

\includegraphics[width=\maxwidth]{/root/R-project/IFishSnapperWPP714_715/Plots/plot-LFD-70} 
\begin{kframe}\begin{verbatim}
\end{verbatim}
\end{kframe}
\clearpage
\newpage
\begin{kframe}\begin{verbatim}The percentages of Lethrinus olivaceus (ID #67, Lethrinidae) in 2016, n = 789
Immature (< 45cm): 1%
Small mature (>= 45cm, < 60cm): 13%
Large mature (>= 60cm): 86%
Mega spawner (>= 66cm): 73% (subset of large mature fish)
Spawning Potential Ratio: near 100%
 
At least 90% of the fish in the catch are mature specimens that have spawned at least
once before they were caught. The fishery does not depend on immature size classes
for this species and is considered safe for this indicator. This fishery will not be
causing overfishing through over harvesting of juveniles for this species. Risk level
is low.

The trade limit is significantly lower than the length at first maturity.  This means
that the trade encourages capture of immature fish, which impairs sustainability.
Risk level is high.

The majority of the catch consists of size classes around or above the optimum
harvest size. This means that the impact of the fishery is minimized for this
species. Potentially higher yields of this species could be achieved by catching them
at somewhat smaller size, although capture of smaller specimen may take place already
in other fisheries. Risk level is low.

More than 30% of the catch consists of mega spawners which indicates that this fish
population is in good health unless large amounts of much smaller fish from the same
population are caught by other fisheries. Risk level is low.
 
Mortality caused by fishing is at or below a level equal to half the natural rate of
mortality. This means that impact of fishing is minimized and this fishery is
currently probably operating at a sustainable level of effort. Risk level is low.
 
SPR is more than 40%. The stock is probably not over exploited, and the risk that the
fishery will cause further stock decline is small. Risk level is low.
 
Trends in relative abundance by size group for Lethrinus olivaceus (ID #67,
Lethrinidae), as calculated from linear regressions. The P value indicates the chance
that this calculated trend is merely a result of stochastic variance.
% Immature rising over recent years, situation deteriorating. P: not available
% Large Mature falling over recent years, situation deteriorating. P: not available
% Mega Spawner falling over recent years, situation deteriorating. P: not available
% SPR falling over recent years, situation deteriorating. P: not available
\end{verbatim}
\end{kframe}
\includegraphics[width=\maxwidth]{/root/R-project/IFishSnapperWPP714_715/Plots/plot-LFD-71} 

\includegraphics[width=\maxwidth]{/root/R-project/IFishSnapperWPP714_715/Plots/plot-LFD-72} 
\begin{kframe}\begin{verbatim}
\end{verbatim}
\end{kframe}
\clearpage
\newpage
\begin{kframe}\begin{verbatim}The percentages of Lethrinus amboinensis (ID #68, Lethrinidae) in 2016, n = 634
Immature (< 27cm): 1%
Small mature (>= 27cm, < 36cm): 21%
Large mature (>= 36cm): 79%
Mega spawner (>= 39.6cm): 58% (subset of large mature fish)
Spawning Potential Ratio: 36%
 
At least 90% of the fish in the catch are mature specimens that have spawned at least
once before they were caught. The fishery does not depend on immature size classes
for this species and is considered safe for this indicator. This fishery will not be
causing overfishing through over harvesting of juveniles for this species. Risk level
is low.

The trade limit is about the same as the length at first maturity.  This means that
the trade puts a premium on fish that have spawned at least once, which improves
sustainability of the fishery. Risk level is medium.

The majority of the catch consists of size classes around or above the optimum
harvest size. This means that the impact of the fishery is minimized for this
species. Potentially higher yields of this species could be achieved by catching them
at somewhat smaller size, although capture of smaller specimen may take place already
in other fisheries. Risk level is low.

More than 30% of the catch consists of mega spawners which indicates that this fish
population is in good health unless large amounts of much smaller fish from the same
population are caught by other fisheries. Risk level is low.
 
Mortality caused by fishing is greater than or equal to the natural rate of
mortality. This means that impact of fishing is severe and that fishing is unlikely
to be sustainable at the current level of effort. Risk level is high.
 
SPR is between 25% and 40%. The stock is heavily exploited, and there is some risk
that the fishery will cause further decline of the stock. Risk level is medium.
 
Trends in relative abundance by size group for Lethrinus amboinensis (ID #68,
Lethrinidae), as calculated from linear regressions. The P value indicates the chance
that this calculated trend is merely a result of stochastic variance.
% Immature no trend over recent years, situation stable. P: not available
% Large Mature rising over recent years, situation improving. P: not available
% Mega Spawner rising over recent years, situation improving. P: not available
% SPR falling over recent years, situation deteriorating. P: not available
\end{verbatim}
\end{kframe}
\includegraphics[width=\maxwidth]{/root/R-project/IFishSnapperWPP714_715/Plots/plot-LFD-73} 

\includegraphics[width=\maxwidth]{/root/R-project/IFishSnapperWPP714_715/Plots/plot-LFD-74} 
\begin{kframe}\begin{verbatim}
\end{verbatim}
\end{kframe}
\clearpage
\newpage
\begin{kframe}\begin{verbatim}The percentages of Wattsia mossambica (ID #70, Lethrinidae) in 2016, n = 2,504
Immature (< 27cm): 2%
Small mature (>= 27cm, < 36cm): 47%
Large mature (>= 36cm): 51%
Mega spawner (>= 39.6cm): 36% (subset of large mature fish)
Spawning Potential Ratio: 50%
 
At least 90% of the fish in the catch are mature specimens that have spawned at least
once before they were caught. The fishery does not depend on immature size classes
for this species and is considered safe for this indicator. This fishery will not be
causing overfishing through over harvesting of juveniles for this species. Risk level
is low.

The trade limit is about the same as the length at first maturity.  This means that
the trade puts a premium on fish that have spawned at least once, which improves
sustainability of the fishery. Risk level is medium.

The majority of the catch consists of size classes around or above the optimum
harvest size. This means that the impact of the fishery is minimized for this
species. Potentially higher yields of this species could be achieved by catching them
at somewhat smaller size, although capture of smaller specimen may take place already
in other fisheries. Risk level is low.

More than 30% of the catch consists of mega spawners which indicates that this fish
population is in good health unless large amounts of much smaller fish from the same
population are caught by other fisheries. Risk level is low.
 
Mortality caused by fishing is lower than the natural rate of mortality but more than
half of natural mortality. This means that impact of fishing is considerable and
trends in various indicators need to be watched carefully while any increase in
fishing effort needs to be prevented. Risk level is medium.
 
SPR is more than 40%. The stock is probably not over exploited, and the risk that the
fishery will cause further stock decline is small. Risk level is low.
 
Trends in relative abundance by size group for Wattsia mossambica (ID #70,
Lethrinidae), as calculated from linear regressions. The P value indicates the chance
that this calculated trend is merely a result of stochastic variance.
% Immature falling over recent years, situation improving. P: not available
% Large Mature rising over recent years, situation improving. P: not available
% Mega Spawner no trend over recent years, situation stable. P: not available
% SPR falling over recent years, situation deteriorating. P: not available
\end{verbatim}
\end{kframe}
\includegraphics[width=\maxwidth]{/root/R-project/IFishSnapperWPP714_715/Plots/plot-LFD-75} 

\includegraphics[width=\maxwidth]{/root/R-project/IFishSnapperWPP714_715/Plots/plot-LFD-76} 
\begin{kframe}\begin{verbatim}
\end{verbatim}
\end{kframe}
\clearpage
\newpage
\begin{kframe}\begin{verbatim}The percentages of Gymnocranius grandoculis (ID #71, Lethrinidae) in 2016, n = 1,665
Immature (< 36cm): 2%
Small mature (>= 36cm, < 48cm): 16%
Large mature (>= 48cm): 82%
Mega spawner (>= 52.8cm): 68% (subset of large mature fish)
Spawning Potential Ratio: 68%
 
At least 90% of the fish in the catch are mature specimens that have spawned at least
once before they were caught. The fishery does not depend on immature size classes
for this species and is considered safe for this indicator. This fishery will not be
causing overfishing through over harvesting of juveniles for this species. Risk level
is low.

The trade limit is significantly lower than the length at first maturity.  This means
that the trade encourages capture of immature fish, which impairs sustainability.
Risk level is high.

The majority of the catch consists of size classes around or above the optimum
harvest size. This means that the impact of the fishery is minimized for this
species. Potentially higher yields of this species could be achieved by catching them
at somewhat smaller size, although capture of smaller specimen may take place already
in other fisheries. Risk level is low.

More than 30% of the catch consists of mega spawners which indicates that this fish
population is in good health unless large amounts of much smaller fish from the same
population are caught by other fisheries. Risk level is low.
 
Mortality caused by fishing is at or below a level equal to half the natural rate of
mortality. This means that impact of fishing is minimized and this fishery is
currently probably operating at a sustainable level of effort. Risk level is low.
 
SPR is more than 40%. The stock is probably not over exploited, and the risk that the
fishery will cause further stock decline is small. Risk level is low.
 
Trends in relative abundance by size group for Gymnocranius grandoculis (ID #71,
Lethrinidae), as calculated from linear regressions. The P value indicates the chance
that this calculated trend is merely a result of stochastic variance.
% Immature no trend over recent years, situation stable. P: not available
% Large Mature falling over recent years, situation deteriorating. P: not available
% Mega Spawner falling over recent years, situation deteriorating. P: not available
% SPR falling over recent years, situation deteriorating. P: not available
\end{verbatim}
\end{kframe}
\includegraphics[width=\maxwidth]{/root/R-project/IFishSnapperWPP714_715/Plots/plot-LFD-77} 

\includegraphics[width=\maxwidth]{/root/R-project/IFishSnapperWPP714_715/Plots/plot-LFD-78} 
\begin{kframe}\begin{verbatim}
\end{verbatim}
\end{kframe}
\clearpage
\newpage
\begin{kframe}\begin{verbatim}The percentages of Carangoides chrysophrys (ID #76, Carangidae) in 2016, n = 527
Immature (< 36cm): 2%
Small mature (>= 36cm, < 48cm): 27%
Large mature (>= 48cm): 71%
Mega spawner (>= 52.8cm): 51% (subset of large mature fish)
Spawning Potential Ratio: 35%
 
At least 90% of the fish in the catch are mature specimens that have spawned at least
once before they were caught. The fishery does not depend on immature size classes
for this species and is considered safe for this indicator. This fishery will not be
causing overfishing through over harvesting of juveniles for this species. Risk level
is low.

The trade limit is significantly higher than length at first maturity.  This means
that the trade puts a premium on fish that have spawned at least once. The trade does
not cause any concern of recruitment overfishing for this species. Risk level is low.

The majority of the catch consists of size classes around or above the optimum
harvest size. This means that the impact of the fishery is minimized for this
species. Potentially higher yields of this species could be achieved by catching them
at somewhat smaller size, although capture of smaller specimen may take place already
in other fisheries. Risk level is low.

More than 30% of the catch consists of mega spawners which indicates that this fish
population is in good health unless large amounts of much smaller fish from the same
population are caught by other fisheries. Risk level is low.
 
Mortality caused by fishing is greater than or equal to the natural rate of
mortality. This means that impact of fishing is severe and that fishing is unlikely
to be sustainable at the current level of effort. Risk level is high.
 
SPR is between 25% and 40%. The stock is heavily exploited, and there is some risk
that the fishery will cause further decline of the stock. Risk level is medium.
 
Trends in relative abundance by size group for Carangoides chrysophrys (ID #76,
Carangidae), as calculated from linear regressions. The P value indicates the chance
that this calculated trend is merely a result of stochastic variance.
% Immature falling over recent years, situation improving. P: not available
% Large Mature falling over recent years, situation deteriorating. P: not available
% Mega Spawner falling over recent years, situation deteriorating. P: not available
% SPR falling over recent years, situation deteriorating. P: not available
\end{verbatim}
\end{kframe}
\includegraphics[width=\maxwidth]{/root/R-project/IFishSnapperWPP714_715/Plots/plot-LFD-79} 

\includegraphics[width=\maxwidth]{/root/R-project/IFishSnapperWPP714_715/Plots/plot-LFD-80} 
\begin{kframe}\begin{verbatim}
\end{verbatim}
\end{kframe}
\clearpage
\newpage
\begin{kframe}\begin{verbatim}The percentages of Carangoides gymnostethus (ID #77, Carangidae) in 2016, n = 263
Immature (< 41cm): 2%
Small mature (>= 41cm, < 54cm): 15%
Large mature (>= 54cm): 83%
Mega spawner (>= 59.4cm): 73% (subset of large mature fish)
Spawning Potential Ratio: near 100%
 
At least 90% of the fish in the catch are mature specimens that have spawned at least
once before they were caught. The fishery does not depend on immature size classes
for this species and is considered safe for this indicator. This fishery will not be
causing overfishing through over harvesting of juveniles for this species. Risk level
is low.

The trade limit is about the same as the length at first maturity.  This means that
the trade puts a premium on fish that have spawned at least once, which improves
sustainability of the fishery. Risk level is medium.

The majority of the catch consists of size classes around or above the optimum
harvest size. This means that the impact of the fishery is minimized for this
species. Potentially higher yields of this species could be achieved by catching them
at somewhat smaller size, although capture of smaller specimen may take place already
in other fisheries. Risk level is low.

More than 30% of the catch consists of mega spawners which indicates that this fish
population is in good health unless large amounts of much smaller fish from the same
population are caught by other fisheries. Risk level is low.
 
Mortality caused by fishing is at or below a level equal to half the natural rate of
mortality. This means that impact of fishing is minimized and this fishery is
currently probably operating at a sustainable level of effort. Risk level is low.
 
SPR is more than 40%. The stock is probably not over exploited, and the risk that the
fishery will cause further stock decline is small. Risk level is low.
 
Trends in relative abundance by size group for Carangoides gymnostethus (ID #77,
Carangidae), as calculated from linear regressions. The P value indicates the chance
that this calculated trend is merely a result of stochastic variance.
% Immature falling over recent years, situation improving. P: not available
% Large Mature no trend over recent years, situation stable. P: not available
% Mega Spawner falling over recent years, situation deteriorating. P: not available
% SPR falling over recent years, situation deteriorating. P: not available
\end{verbatim}
\end{kframe}
\includegraphics[width=\maxwidth]{/root/R-project/IFishSnapperWPP714_715/Plots/plot-LFD-81} 

\includegraphics[width=\maxwidth]{/root/R-project/IFishSnapperWPP714_715/Plots/plot-LFD-82} 
\begin{kframe}\begin{verbatim}
\end{verbatim}
\end{kframe}
\clearpage
\newpage
\begin{kframe}\begin{verbatim}The percentages of Caranx ignobilis (ID #79, Carangidae) in 2016, n = 259
Immature (< 63cm): 7%
Small mature (>= 63cm, < 84cm): 65%
Large mature (>= 84cm): 27%
Mega spawner (>= 92.4cm): 9% (subset of large mature fish)
Spawning Potential Ratio: 13%
 
At least 90% of the fish in the catch are mature specimens that have spawned at least
once before they were caught. The fishery does not depend on immature size classes
for this species and is considered safe for this indicator. This fishery will not be
causing overfishing through over harvesting of juveniles for this species. Risk level
is low.

The trade limit is significantly lower than the length at first maturity.  This means
that the trade encourages capture of immature fish, which impairs sustainability.
Risk level is high.

The vast majority of the fish in the catch have not yet achieved their growth
potential. The harvest of small fish promotes growth overfishing and the size
distribution for this species indicates that over exploitation through growth
overfishing may already be happening. Risk level is high.

Less than 20% of the catch comprises of mega spawners.  This indicates that the
population may be severely affected by the fishery, and that there is a substantial
risk of recruitment overfishing through over harvesting of the mega spawners, unless
large numbers of mega spawners would be surviving at other habitats. There is no
reason to assume that this is the case and therefore a reduction of fishing effort
may be necessary in this fishery. Risk level is high.
 
Mortality caused by fishing is greater than or equal to the natural rate of
mortality. This means that impact of fishing is severe and that fishing is unlikely
to be sustainable at the current level of effort. Risk level is high.
 
SPR is less than 25%. The fishery probably over-exploits the stock, and there is a
substantial risk that the fishery will cause severe decline of the stock if fishing
effort is not reduced. Risk level is high.
 
Trends in relative abundance by size group for Caranx ignobilis (ID #79, Carangidae),
as calculated from linear regressions. The P value indicates the chance that this
calculated trend is merely a result of stochastic variance.
% Immature rising over recent years, situation deteriorating. P: not available
% Large Mature falling over recent years, situation deteriorating. P: not available
% Mega Spawner falling over recent years, situation deteriorating. P: not available
% SPR falling over recent years, situation deteriorating. P: not available
\end{verbatim}
\end{kframe}
\includegraphics[width=\maxwidth]{/root/R-project/IFishSnapperWPP714_715/Plots/plot-LFD-83} 

\includegraphics[width=\maxwidth]{/root/R-project/IFishSnapperWPP714_715/Plots/plot-LFD-84} 
\begin{kframe}\begin{verbatim}
\end{verbatim}
\end{kframe}
\clearpage
\newpage
\begin{kframe}\begin{verbatim}The percentages of Caranx sexfasciatus (ID #81, Carangidae) in 2016, n = 619
Immature (< 38cm): 1%
Small mature (>= 38cm, < 51cm): 7%
Large mature (>= 51cm): 92%
Mega spawner (>= 56.1cm): 79% (subset of large mature fish)
Spawning Potential Ratio: 86%
 
At least 90% of the fish in the catch are mature specimens that have spawned at least
once before they were caught. The fishery does not depend on immature size classes
for this species and is considered safe for this indicator. This fishery will not be
causing overfishing through over harvesting of juveniles for this species. Risk level
is low.

The trade limit is significantly higher than length at first maturity.  This means
that the trade puts a premium on fish that have spawned at least once. The trade does
not cause any concern of recruitment overfishing for this species. Risk level is low.

The majority of the catch consists of size classes around or above the optimum
harvest size. This means that the impact of the fishery is minimized for this
species. Potentially higher yields of this species could be achieved by catching them
at somewhat smaller size, although capture of smaller specimen may take place already
in other fisheries. Risk level is low.

More than 30% of the catch consists of mega spawners which indicates that this fish
population is in good health unless large amounts of much smaller fish from the same
population are caught by other fisheries. Risk level is low.
 
Mortality caused by fishing is at or below a level equal to half the natural rate of
mortality. This means that impact of fishing is minimized and this fishery is
currently probably operating at a sustainable level of effort. Risk level is low.
 
SPR is more than 40%. The stock is probably not over exploited, and the risk that the
fishery will cause further stock decline is small. Risk level is low.
 
Trends in relative abundance by size group for Caranx sexfasciatus (ID #81,
Carangidae), as calculated from linear regressions. The P value indicates the chance
that this calculated trend is merely a result of stochastic variance.
% Immature no trend over recent years, situation stable. P: not available
% Large Mature falling over recent years, situation deteriorating. P: not available
% Mega Spawner falling over recent years, situation deteriorating. P: not available
% SPR falling over recent years, situation deteriorating. P: not available
\end{verbatim}
\end{kframe}
\includegraphics[width=\maxwidth]{/root/R-project/IFishSnapperWPP714_715/Plots/plot-LFD-85} 

\includegraphics[width=\maxwidth]{/root/R-project/IFishSnapperWPP714_715/Plots/plot-LFD-86} 
\begin{kframe}\begin{verbatim}
\end{verbatim}
\end{kframe}
\clearpage
\newpage
\begin{kframe}\begin{verbatim}The percentages of Caranx tille (ID #82, Carangidae) in 2016, n = 402
Immature (< 41cm): 0%
Small mature (>= 41cm, < 54cm): 2%
Large mature (>= 54cm): 98%
Mega spawner (>= 59.4cm): 95% (subset of large mature fish)
Spawning Potential Ratio: 51%
 
At least 90% of the fish in the catch are mature specimens that have spawned at least
once before they were caught. The fishery does not depend on immature size classes
for this species and is considered safe for this indicator. This fishery will not be
causing overfishing through over harvesting of juveniles for this species. Risk level
is low.

The trade limit is significantly higher than length at first maturity.  This means
that the trade puts a premium on fish that have spawned at least once. The trade does
not cause any concern of recruitment overfishing for this species. Risk level is low.

The majority of the catch consists of size classes around or above the optimum
harvest size. This means that the impact of the fishery is minimized for this
species. Potentially higher yields of this species could be achieved by catching them
at somewhat smaller size, although capture of smaller specimen may take place already
in other fisheries. Risk level is low.

More than 30% of the catch consists of mega spawners which indicates that this fish
population is in good health unless large amounts of much smaller fish from the same
population are caught by other fisheries. Risk level is low.
 
Mortality caused by fishing is greater than or equal to the natural rate of
mortality. This means that impact of fishing is severe and that fishing is unlikely
to be sustainable at the current level of effort. Risk level is high.
 
SPR is more than 40%. The stock is probably not over exploited, and the risk that the
fishery will cause further stock decline is small. Risk level is low.
 
Trends in relative abundance by size group for Caranx tille (ID #82, Carangidae), as
calculated from linear regressions. The P value indicates the chance that this
calculated trend is merely a result of stochastic variance.
% Immature no trend over recent years, situation stable. P: not available
% Large Mature falling over recent years, situation deteriorating. P: not available
% Mega Spawner falling over recent years, situation deteriorating. P: not available
% SPR falling over recent years, situation deteriorating. P: not available
\end{verbatim}
\end{kframe}
\includegraphics[width=\maxwidth]{/root/R-project/IFishSnapperWPP714_715/Plots/plot-LFD-87} 

\includegraphics[width=\maxwidth]{/root/R-project/IFishSnapperWPP714_715/Plots/plot-LFD-88} 
\begin{kframe}\begin{verbatim}
\end{verbatim}
\end{kframe}
\clearpage
\newpage
\begin{kframe}\begin{verbatim}The percentages of Elagatis bipinnulata (ID #83, Carangidae) in 2016, n = 416
Immature (< 50cm): 1%
Small mature (>= 50cm, < 66cm): 23%
Large mature (>= 66cm): 75%
Mega spawner (>= 72.6cm): 56% (subset of large mature fish)
Spawning Potential Ratio: 70%
 
At least 90% of the fish in the catch are mature specimens that have spawned at least
once before they were caught. The fishery does not depend on immature size classes
for this species and is considered safe for this indicator. This fishery will not be
causing overfishing through over harvesting of juveniles for this species. Risk level
is low.

The trade limit is significantly higher than length at first maturity.  This means
that the trade puts a premium on fish that have spawned at least once. The trade does
not cause any concern of recruitment overfishing for this species. Risk level is low.

The majority of the catch consists of size classes around or above the optimum
harvest size. This means that the impact of the fishery is minimized for this
species. Potentially higher yields of this species could be achieved by catching them
at somewhat smaller size, although capture of smaller specimen may take place already
in other fisheries. Risk level is low.

More than 30% of the catch consists of mega spawners which indicates that this fish
population is in good health unless large amounts of much smaller fish from the same
population are caught by other fisheries. Risk level is low.
 
Mortality caused by fishing is at or below a level equal to half the natural rate of
mortality. This means that impact of fishing is minimized and this fishery is
currently probably operating at a sustainable level of effort. Risk level is low.
 
SPR is more than 40%. The stock is probably not over exploited, and the risk that the
fishery will cause further stock decline is small. Risk level is low.
 
Trends in relative abundance by size group for Elagatis bipinnulata (ID #83,
Carangidae), as calculated from linear regressions. The P value indicates the chance
that this calculated trend is merely a result of stochastic variance.
% Immature rising over recent years, situation deteriorating. P: not available
% Large Mature falling over recent years, situation deteriorating. P: not available
% Mega Spawner falling over recent years, situation deteriorating. P: not available
% SPR falling over recent years, situation deteriorating. P: not available
\end{verbatim}
\end{kframe}
\includegraphics[width=\maxwidth]{/root/R-project/IFishSnapperWPP714_715/Plots/plot-LFD-89} 

\includegraphics[width=\maxwidth]{/root/R-project/IFishSnapperWPP714_715/Plots/plot-LFD-90} 
\begin{kframe}\begin{verbatim}
\end{verbatim}
\end{kframe}
\clearpage
\newpage
\begin{kframe}\begin{verbatim}The percentages of Seriola rivoliana (ID #85, Carangidae) in 2016, n = 995
Immature (< 61cm): 30%
Small mature (>= 61cm, < 81cm): 50%
Large mature (>= 81cm): 20%
Mega spawner (>= 89.1cm): 8% (subset of large mature fish)
Spawning Potential Ratio: 18%
 
Between 30% and 50% of the fish in the catch are immature and have not had a chance
to reproduce before capture. The fishery is in immediate danger of overfishing
through overharvesting of juveniles, if fishing pressure is high.  Catching small and
immature fish needs to be actively avoided and a limit on overall fishing pressure is
warranted. Risk level is high.

The trade limit is about the same as the length at first maturity.  This means that
the trade puts a premium on fish that have spawned at least once, which improves
sustainability of the fishery. Risk level is medium.

The vast majority of the fish in the catch have not yet achieved their growth
potential. The harvest of small fish promotes growth overfishing and the size
distribution for this species indicates that over exploitation through growth
overfishing may already be happening. Risk level is high.

Less than 20% of the catch comprises of mega spawners.  This indicates that the
population may be severely affected by the fishery, and that there is a substantial
risk of recruitment overfishing through over harvesting of the mega spawners, unless
large numbers of mega spawners would be surviving at other habitats. There is no
reason to assume that this is the case and therefore a reduction of fishing effort
may be necessary in this fishery. Risk level is high.
 
Mortality caused by fishing is greater than or equal to the natural rate of
mortality. This means that impact of fishing is severe and that fishing is unlikely
to be sustainable at the current level of effort. Risk level is high.
 
SPR is less than 25%. The fishery probably over-exploits the stock, and there is a
substantial risk that the fishery will cause severe decline of the stock if fishing
effort is not reduced. Risk level is high.
 
Trends in relative abundance by size group for Seriola rivoliana (ID #85,
Carangidae), as calculated from linear regressions. The P value indicates the chance
that this calculated trend is merely a result of stochastic variance.
% Immature falling over recent years, situation improving. P: not available
% Large Mature falling over recent years, situation deteriorating. P: not available
% Mega Spawner falling over recent years, situation deteriorating. P: not available
% SPR falling over recent years, situation deteriorating. P: not available
\end{verbatim}
\end{kframe}
\includegraphics[width=\maxwidth]{/root/R-project/IFishSnapperWPP714_715/Plots/plot-LFD-91} 

\includegraphics[width=\maxwidth]{/root/R-project/IFishSnapperWPP714_715/Plots/plot-LFD-92} 
\begin{kframe}\begin{verbatim}
\end{verbatim}
\end{kframe}
\clearpage
\newpage
\begin{kframe}\begin{verbatim}The percentages of Erythrocles schlegelii (ID #86, Emmelichthydae) in 2016, n = 1,157
Immature (< 41cm): 0%
Small mature (>= 41cm, < 54cm): 22%
Large mature (>= 54cm): 78%
Mega spawner (>= 59.4cm): 50% (subset of large mature fish)
Spawning Potential Ratio: 62%
 
At least 90% of the fish in the catch are mature specimens that have spawned at least
once before they were caught. The fishery does not depend on immature size classes
for this species and is considered safe for this indicator. This fishery will not be
causing overfishing through over harvesting of juveniles for this species. Risk level
is low.

The trade limit is significantly higher than length at first maturity.  This means
that the trade puts a premium on fish that have spawned at least once. The trade does
not cause any concern of recruitment overfishing for this species. Risk level is low.

The majority of the catch consists of size classes around or above the optimum
harvest size. This means that the impact of the fishery is minimized for this
species. Potentially higher yields of this species could be achieved by catching them
at somewhat smaller size, although capture of smaller specimen may take place already
in other fisheries. Risk level is low.

More than 30% of the catch consists of mega spawners which indicates that this fish
population is in good health unless large amounts of much smaller fish from the same
population are caught by other fisheries. Risk level is low.
 
Mortality caused by fishing is at or below a level equal to half the natural rate of
mortality. This means that impact of fishing is minimized and this fishery is
currently probably operating at a sustainable level of effort. Risk level is low.
 
SPR is more than 40%. The stock is probably not over exploited, and the risk that the
fishery will cause further stock decline is small. Risk level is low.
 
Trends in relative abundance by size group for Erythrocles schlegelii (ID #86,
Emmelichthydae), as calculated from linear regressions. The P value indicates the
chance that this calculated trend is merely a result of stochastic variance.
% Immature no trend over recent years, situation stable. P: not available
% Large Mature falling over recent years, situation deteriorating. P: not available
% Mega Spawner rising over recent years, situation improving. P: not available
% SPR rising over recent years, situation improving. P: not available
\end{verbatim}
\end{kframe}
\includegraphics[width=\maxwidth]{/root/R-project/IFishSnapperWPP714_715/Plots/plot-LFD-93} 

\includegraphics[width=\maxwidth]{/root/R-project/IFishSnapperWPP714_715/Plots/plot-LFD-94} 
\begin{kframe}\begin{verbatim}
\end{verbatim}
\end{kframe}
\clearpage
\newpage
\begin{kframe}\begin{verbatim}The percentages of Dentex carpenteri (ID #88, Sparidae) in 2016, n = 521
Immature (< 20cm): 0%
Small mature (>= 20cm, < 27cm): 5%
Large mature (>= 27cm): 95%
Mega spawner (>= 29.7cm): 91% (subset of large mature fish)
Spawning Potential Ratio: 60%
 
At least 90% of the fish in the catch are mature specimens that have spawned at least
once before they were caught. The fishery does not depend on immature size classes
for this species and is considered safe for this indicator. This fishery will not be
causing overfishing through over harvesting of juveniles for this species. Risk level
is low.

The trade limit is significantly higher than length at first maturity.  This means
that the trade puts a premium on fish that have spawned at least once. The trade does
not cause any concern of recruitment overfishing for this species. Risk level is low.

The majority of the catch consists of size classes around or above the optimum
harvest size. This means that the impact of the fishery is minimized for this
species. Potentially higher yields of this species could be achieved by catching them
at somewhat smaller size, although capture of smaller specimen may take place already
in other fisheries. Risk level is low.

More than 30% of the catch consists of mega spawners which indicates that this fish
population is in good health unless large amounts of much smaller fish from the same
population are caught by other fisheries. Risk level is low.
 
Mortality caused by fishing is lower than the natural rate of mortality but more than
half of natural mortality. This means that impact of fishing is considerable and
trends in various indicators need to be watched carefully while any increase in
fishing effort needs to be prevented. Risk level is medium.
 
SPR is more than 40%. The stock is probably not over exploited, and the risk that the
fishery will cause further stock decline is small. Risk level is low.
 
Trends in relative abundance by size group for Dentex carpenteri (ID #88, Sparidae),
as calculated from linear regressions. The P value indicates the chance that this
calculated trend is merely a result of stochastic variance.
% Immature no trend over recent years, situation stable. P: not available
% Large Mature rising over recent years, situation improving. P: not available
% Mega Spawner rising over recent years, situation improving. P: not available
% SPR falling over recent years, situation deteriorating. P: not available
\end{verbatim}
\end{kframe}
\includegraphics[width=\maxwidth]{/root/R-project/IFishSnapperWPP714_715/Plots/plot-LFD-95} 

\includegraphics[width=\maxwidth]{/root/R-project/IFishSnapperWPP714_715/Plots/plot-LFD-96} 
\begin{kframe}\begin{verbatim}
\end{verbatim}
\end{kframe}
\clearpage
\newpage
\begin{kframe}\begin{verbatim}The percentages of Glaucosoma buergeri (ID #89, Glaucosomatidae) in 2016, n = 496
Immature (< 32cm): 2%
Small mature (>= 32cm, < 42cm): 24%
Large mature (>= 42cm): 74%
Mega spawner (>= 46.2cm): 65% (subset of large mature fish)
Spawning Potential Ratio: 63%
 
At least 90% of the fish in the catch are mature specimens that have spawned at least
once before they were caught. The fishery does not depend on immature size classes
for this species and is considered safe for this indicator. This fishery will not be
causing overfishing through over harvesting of juveniles for this species. Risk level
is low.

The trade limit is about the same as the length at first maturity.  This means that
the trade puts a premium on fish that have spawned at least once, which improves
sustainability of the fishery. Risk level is medium.

The majority of the catch consists of size classes around or above the optimum
harvest size. This means that the impact of the fishery is minimized for this
species. Potentially higher yields of this species could be achieved by catching them
at somewhat smaller size, although capture of smaller specimen may take place already
in other fisheries. Risk level is low.

More than 30% of the catch consists of mega spawners which indicates that this fish
population is in good health unless large amounts of much smaller fish from the same
population are caught by other fisheries. Risk level is low.
 
Mortality caused by fishing is at or below a level equal to half the natural rate of
mortality. This means that impact of fishing is minimized and this fishery is
currently probably operating at a sustainable level of effort. Risk level is low.
 
SPR is more than 40%. The stock is probably not over exploited, and the risk that the
fishery will cause further stock decline is small. Risk level is low.
 
Trends in relative abundance by size group for Glaucosoma buergeri (ID #89,
Glaucosomatidae), as calculated from linear regressions. The P value indicates the
chance that this calculated trend is merely a result of stochastic variance.
% Immature falling over recent years, situation improving. P: not available
% Large Mature rising over recent years, situation improving. P: not available
% Mega Spawner rising over recent years, situation improving. P: not available
% SPR rising over recent years, situation improving. P: not available
\end{verbatim}
\end{kframe}
\includegraphics[width=\maxwidth]{/root/R-project/IFishSnapperWPP714_715/Plots/plot-LFD-97} 

\includegraphics[width=\maxwidth]{/root/R-project/IFishSnapperWPP714_715/Plots/plot-LFD-98} 
\begin{kframe}\begin{verbatim}
\end{verbatim}
\end{kframe}
\clearpage
\newpage
\begin{kframe}\begin{verbatim}The percentages of Diagramma pictum (ID #91, Haemulidae) in 2016, n = 303
Immature (< 38cm): 0%
Small mature (>= 38cm, < 51cm): 1%
Large mature (>= 51cm): 99%
Mega spawner (>= 56.1cm): 96% (subset of large mature fish)
Spawning Potential Ratio: 75%
 
At least 90% of the fish in the catch are mature specimens that have spawned at least
once before they were caught. The fishery does not depend on immature size classes
for this species and is considered safe for this indicator. This fishery will not be
causing overfishing through over harvesting of juveniles for this species. Risk level
is low.

The trade limit is about the same as the length at first maturity.  This means that
the trade puts a premium on fish that have spawned at least once, which improves
sustainability of the fishery. Risk level is medium.

The majority of the catch consists of size classes around or above the optimum
harvest size. This means that the impact of the fishery is minimized for this
species. Potentially higher yields of this species could be achieved by catching them
at somewhat smaller size, although capture of smaller specimen may take place already
in other fisheries. Risk level is low.

More than 30% of the catch consists of mega spawners which indicates that this fish
population is in good health unless large amounts of much smaller fish from the same
population are caught by other fisheries. Risk level is low.
 
Mortality caused by fishing is lower than the natural rate of mortality but more than
half of natural mortality. This means that impact of fishing is considerable and
trends in various indicators need to be watched carefully while any increase in
fishing effort needs to be prevented. Risk level is medium.
 
SPR is more than 40%. The stock is probably not over exploited, and the risk that the
fishery will cause further stock decline is small. Risk level is low.
 
Trends in relative abundance by size group for Diagramma pictum (ID #91, Haemulidae),
as calculated from linear regressions. The P value indicates the chance that this
calculated trend is merely a result of stochastic variance.
% Immature no trend over recent years, situation stable. P: not available
% Large Mature falling over recent years, situation deteriorating. P: not available
% Mega Spawner falling over recent years, situation deteriorating. P: not available
% SPR falling over recent years, situation deteriorating. P: not available
\end{verbatim}
\end{kframe}
\includegraphics[width=\maxwidth]{/root/R-project/IFishSnapperWPP714_715/Plots/plot-LFD-99} 

\includegraphics[width=\maxwidth]{/root/R-project/IFishSnapperWPP714_715/Plots/plot-LFD-100} 
\begin{kframe}\begin{verbatim}
\end{verbatim}
\end{kframe}
\clearpage
\newpage
\begin{kframe}\begin{verbatim}The percentages of Cookeolus japonicus (ID #93, Priacanthidae) in 2016, n = 262
Immature (< 29cm): 2%
Small mature (>= 29cm, < 39cm): 25%
Large mature (>= 39cm): 73%
Mega spawner (>= 42.9cm): 60% (subset of large mature fish)
Spawning Potential Ratio: 63%
 
At least 90% of the fish in the catch are mature specimens that have spawned at least
once before they were caught. The fishery does not depend on immature size classes
for this species and is considered safe for this indicator. This fishery will not be
causing overfishing through over harvesting of juveniles for this species. Risk level
is low.

The trade limit is about the same as the length at first maturity.  This means that
the trade puts a premium on fish that have spawned at least once, which improves
sustainability of the fishery. Risk level is medium.

The majority of the catch consists of size classes around or above the optimum
harvest size. This means that the impact of the fishery is minimized for this
species. Potentially higher yields of this species could be achieved by catching them
at somewhat smaller size, although capture of smaller specimen may take place already
in other fisheries. Risk level is low.

More than 30% of the catch consists of mega spawners which indicates that this fish
population is in good health unless large amounts of much smaller fish from the same
population are caught by other fisheries. Risk level is low.
 
Mortality caused by fishing is at or below a level equal to half the natural rate of
mortality. This means that impact of fishing is minimized and this fishery is
currently probably operating at a sustainable level of effort. Risk level is low.
 
SPR is more than 40%. The stock is probably not over exploited, and the risk that the
fishery will cause further stock decline is small. Risk level is low.
 
Trends in relative abundance by size group for Cookeolus japonicus (ID #93,
Priacanthidae), as calculated from linear regressions. The P value indicates the
chance that this calculated trend is merely a result of stochastic variance.
% Immature falling over recent years, situation improving. P: not available
% Large Mature rising over recent years, situation improving. P: not available
% Mega Spawner rising over recent years, situation improving. P: not available
% SPR falling over recent years, situation deteriorating. P: not available
\end{verbatim}
\end{kframe}
\end{knitrout}

% latex table generated in R 3.2.2 by xtable 1.7-4 package
% Wed Apr 19 00:14:30 2017
\begin{table}[ht]
\centering
\caption{Values of Indicator in 2016 Length-Based Assessment} 
{\small
\begin{tabular}{cccccccc}
  \hline
\#ID & Species & Immature & Trade Limit & Exploitation & Mega Spawn & F vs M & SPR \\ 
  {} & {} & {\%} & {Prop. Lmat} & {\%imm+\%smat} & {\%} & {Ratio} & {\%} \\ \hline
  1 & Aphareus rutilans & 32.99 & 0.75 & 93.22 & 1.32 & 4.04 & 3 \\ 
    2 & Aprion virescens & 7.99 & 0.79 & 52.75 & 16.06 & 4.08 & 6 \\ 
    4 & Etelis sp. & 71.30 & 0.48 & 98.10 & 0.42 & 3.58 & 2 \\ 
    5 & Etelis radiosus & 40.11 & 0.77 & 89.25 & 4.09 & 2.50 & 6 \\ 
    6 & Etelis coruscans & 67.81 & 0.55 & 95.40 & 0.92 & 3.03 & 3 \\ 
    7 & Pristipomoides multidens & 30.50 & 0.66 & 93.13 & 1.02 & 3.16 & 7 \\ 
    8 & Pristipomoides typus & 9.89 & 0.80 & 66.94 & 13.61 & 2.65 & 15 \\ 
    9 & Pristipomoides filamentosus & 55.47 & 0.69 & 92.30 & 2.43 & 2.85 & 7 \\ 
   10 & Pristipomoides sieboldii & 4.12 & 0.91 & 81.53 & 2.91 & 5.55 & 2 \\ 
   15 & Lutjanus argentimaculatus & 8.74 & 0.60 & 76.77 & 7.81 & 2.14 & 17 \\ 
   16 & Lutjanus bohar & 27.98 & 0.65 & 77.88 & 7.94 & 1.96 & 15 \\ 
   17 & Lutjanus malabaricus & 14.47 & 0.62 & 65.37 & 16.91 & 1.33 & 27 \\ 
   18 & Lutjanus sebae & 17.61 & 0.59 & 70.89 & 15.26 & 1.27 & 26 \\ 
   19 & Lutjanus timorensis & 9.09 & 1.04 & 78.16 & 6.85 & 3.13 & 7 \\ 
   20 & Lutjanus gibbus & 1.27 & 1.15 & 45.56 & 29.82 & 2.45 & 16 \\ 
   21 & Lutjanus erythropterus & 0.09 & 0.86 & 5.89 & 69.54 & 1.06 & 52 \\ 
   22 & Pinjalo lewisi & 0.13 & 1.02 & 54.54 & 18.11 & 3.10 & 11 \\ 
   27 & Lutjanus vitta & 3.96 & 1.15 & 40.97 & 28.63 & 1.11 & 28 \\ 
   31 & Symphorus nematophorus & 0.00 & 0.76 & 7.91 & 62.66 & 0.94 & 53 \\ 
   32 & Paracaesio gonzalesi & 3.08 & 0.86 & 62.53 & 8.54 & 6.54 & 2 \\ 
   33 & Paracaesio xanthura & 0.79 & 1.01 & 26.65 & 44.06 & 0.44 & 56 \\ 
   34 & Paracaesio kusakarii & 12.89 & 0.77 & 89.51 & 2.77 & 4.27 & 6 \\ 
   35 & Paracaesio stonei & 9.49 & 0.87 & 80.36 & 5.65 & 3.25 & 8 \\ 
   39 & Cephalopholis sonnerati & 3.15 & 1.12 & 18.74 & 75.43 & near 0 & near 100 \\ 
   42 & Epinephelus radiatus & 1.90 & 0.83 & 25.12 & 54.98 & 0.30 & 70 \\ 
   43 & Epinephelus morrhua & 10.22 & 0.83 & 59.25 & 22.10 & 1.02 & 32 \\ 
   45 & Epinephelus areolatus & 0.42 & 1.37 & 14.33 & 71.49 & 0.69 & 59 \\ 
   48 & Epinephelus bilobatus & 0.00 & 1.21 & 1.87 & 96.73 & 0.08 & 93 \\ 
   50 & Epinephelus coioides & 1.55 & 0.94 & 24.03 & 55.81 & 0.13 & 83 \\ 
   51 & Epinephelus chlorostigma & 5.10 & 1.12 & 31.55 & 49.03 & 0.67 & 46 \\ 
   54 & Epinephelus stictus & 0.00 & 1.34 & 5.63 & 88.47 & near 0 & near 100 \\ 
   57 & Epinephelus undulosus & 0.00 & 1.52 & 2.88 & 93.00 & near 0 & near 100 \\ 
   58 & Epinephelus amblycephalus & 1.11 & 1.39 & 13.59 & 80.18 & near 0 & near 100 \\ 
   62 & Variola albimarginata & 0.74 & 1.22 & 30.00 & 41.11 & 1.96 & 19 \\ 
   67 & Lethrinus olivaceus & 1.01 & 0.61 & 14.45 & 73.13 & near 0 & near 100 \\ 
   68 & Lethrinus amboinensis & 0.63 & 1.04 & 21.29 & 58.36 & 1.60 & 36 \\ 
   70 & Wattsia mossambica & 1.84 & 1.09 & 48.56 & 36.02 & 0.61 & 50 \\ 
   71 & Gymnocranius grandoculis & 1.86 & 0.85 & 18.14 & 67.99 & 0.33 & 68 \\ 
   76 & Carangoides chrysophrys & 2.09 & 1.17 & 29.41 & 51.23 & 1.22 & 35 \\ 
   77 & Carangoides gymnostethus & 1.90 & 1.01 & 17.11 & 73.00 & near 0 & near 100 \\ 
   79 & Caranx ignobilis & 7.34 & 0.86 & 72.59 & 8.88 & 2.11 & 13 \\ 
   81 & Caranx sexfasciatus & 0.81 & 1.30 & 7.92 & 79.16 & 0.16 & 86 \\ 
   82 & Caranx tille & 0.50 & 1.21 & 2.49 & 94.78 & 1.04 & 51 \\ 
   83 & Elagatis bipinnulata & 1.44 & 1.11 & 24.52 & 56.01 & 0.29 & 70 \\ 
   85 & Seriola rivoliana & 30.45 & 0.98 & 80.00 & 8.34 & 1.22 & 18 \\ 
   86 & Erythrocles schlegelii & 0.35 & 1.31 & 22.47 & 49.87 & 0.46 & 62 \\ 
   88 & Dentex carpenteri & 0.00 & 1.38 & 5.18 & 91.17 & 0.86 & 60 \\ 
   89 & Glaucosoma buergeri & 2.42 & 0.95 & 26.21 & 65.12 & 0.50 & 63 \\ 
   91 & Diagramma pictum & 0.00 & 0.97 & 0.99 & 95.71 & 0.51 & 75 \\ 
   93 & Cookeolus japonicus & 2.29 & 0.95 & 27.10 & 59.92 & 0.35 & 63 \\ 
   \hline
\end{tabular}
}
\end{table}

\clearpage
\newpage

% latex table generated in R 3.2.2 by xtable 1.7-4 package
% Wed Apr 19 00:14:30 2017
\begin{table}[ht]
\centering
\caption{Risk Level in Fisheries by Species and by Indicator for 2016} 
{\small
\begin{tabular}{cccccccc}
  \hline
\#ID & Species & Immature & Trade Limit & Exploitation & Mega Spawn & F vs M & SPR \\ 
  \hline
  1 & Aphareus rutilans & \textcolor{red}{\textbf{high}} & \textcolor{red}{\textbf{high}} & \textcolor{red}{\textbf{high}} & \textcolor{red}{\textbf{high}} & \textcolor{red}{\textbf{high}} & \textcolor{red}{\textbf{high}} \\ 
    2 & Aprion virescens & \textcolor{green}{\textbf{low}} & \textcolor{red}{\textbf{high}} & \textcolor{blue}{\textbf{medium}} & \textcolor{red}{\textbf{high}} & \textcolor{red}{\textbf{high}} & \textcolor{red}{\textbf{high}} \\ 
    4 & Etelis sp. & \textcolor{red}{\textbf{high}} & \textcolor{red}{\textbf{high}} & \textcolor{red}{\textbf{high}} & \textcolor{red}{\textbf{high}} & \textcolor{red}{\textbf{high}} & \textcolor{red}{\textbf{high}} \\ 
    5 & Etelis radiosus & \textcolor{red}{\textbf{high}} & \textcolor{red}{\textbf{high}} & \textcolor{red}{\textbf{high}} & \textcolor{red}{\textbf{high}} & \textcolor{red}{\textbf{high}} & \textcolor{red}{\textbf{high}} \\ 
    6 & Etelis coruscans & \textcolor{red}{\textbf{high}} & \textcolor{red}{\textbf{high}} & \textcolor{red}{\textbf{high}} & \textcolor{red}{\textbf{high}} & \textcolor{red}{\textbf{high}} & \textcolor{red}{\textbf{high}} \\ 
    7 & Pristipomoides multidens & \textcolor{red}{\textbf{high}} & \textcolor{red}{\textbf{high}} & \textcolor{red}{\textbf{high}} & \textcolor{red}{\textbf{high}} & \textcolor{red}{\textbf{high}} & \textcolor{red}{\textbf{high}} \\ 
    8 & Pristipomoides typus & \textcolor{green}{\textbf{low}} & \textcolor{red}{\textbf{high}} & \textcolor{red}{\textbf{high}} & \textcolor{red}{\textbf{high}} & \textcolor{red}{\textbf{high}} & \textcolor{red}{\textbf{high}} \\ 
    9 & Pristipomoides filamentosus & \textcolor{red}{\textbf{high}} & \textcolor{red}{\textbf{high}} & \textcolor{red}{\textbf{high}} & \textcolor{red}{\textbf{high}} & \textcolor{red}{\textbf{high}} & \textcolor{red}{\textbf{high}} \\ 
   10 & Pristipomoides sieboldii & \textcolor{green}{\textbf{low}} & \textcolor{blue}{\textbf{medium}} & \textcolor{red}{\textbf{high}} & \textcolor{red}{\textbf{high}} & \textcolor{red}{\textbf{high}} & \textcolor{red}{\textbf{high}} \\ 
   15 & Lutjanus argentimaculatus & \textcolor{green}{\textbf{low}} & \textcolor{red}{\textbf{high}} & \textcolor{red}{\textbf{high}} & \textcolor{red}{\textbf{high}} & \textcolor{red}{\textbf{high}} & \textcolor{red}{\textbf{high}} \\ 
   16 & Lutjanus bohar & \textcolor{blue}{\textbf{medium}} & \textcolor{red}{\textbf{high}} & \textcolor{red}{\textbf{high}} & \textcolor{red}{\textbf{high}} & \textcolor{red}{\textbf{high}} & \textcolor{red}{\textbf{high}} \\ 
   17 & Lutjanus malabaricus & \textcolor{blue}{\textbf{medium}} & \textcolor{red}{\textbf{high}} & \textcolor{red}{\textbf{high}} & \textcolor{red}{\textbf{high}} & \textcolor{red}{\textbf{high}} & \textcolor{blue}{\textbf{medium}} \\ 
   18 & Lutjanus sebae & \textcolor{blue}{\textbf{medium}} & \textcolor{red}{\textbf{high}} & \textcolor{red}{\textbf{high}} & \textcolor{red}{\textbf{high}} & \textcolor{red}{\textbf{high}} & \textcolor{blue}{\textbf{medium}} \\ 
   19 & Lutjanus timorensis & \textcolor{green}{\textbf{low}} & \textcolor{blue}{\textbf{medium}} & \textcolor{red}{\textbf{high}} & \textcolor{red}{\textbf{high}} & \textcolor{red}{\textbf{high}} & \textcolor{red}{\textbf{high}} \\ 
   20 & Lutjanus gibbus & \textcolor{green}{\textbf{low}} & \textcolor{green}{\textbf{low}} & \textcolor{green}{\textbf{low}} & \textcolor{blue}{\textbf{medium}} & \textcolor{red}{\textbf{high}} & \textcolor{red}{\textbf{high}} \\ 
   21 & Lutjanus erythropterus & \textcolor{green}{\textbf{low}} & \textcolor{red}{\textbf{high}} & \textcolor{green}{\textbf{low}} & \textcolor{green}{\textbf{low}} & \textcolor{red}{\textbf{high}} & \textcolor{green}{\textbf{low}} \\ 
   22 & Pinjalo lewisi & \textcolor{green}{\textbf{low}} & \textcolor{blue}{\textbf{medium}} & \textcolor{blue}{\textbf{medium}} & \textcolor{red}{\textbf{high}} & \textcolor{red}{\textbf{high}} & \textcolor{red}{\textbf{high}} \\ 
   27 & Lutjanus vitta & \textcolor{green}{\textbf{low}} & \textcolor{green}{\textbf{low}} & \textcolor{green}{\textbf{low}} & \textcolor{blue}{\textbf{medium}} & \textcolor{red}{\textbf{high}} & \textcolor{blue}{\textbf{medium}} \\ 
   31 & Symphorus nematophorus & \textcolor{green}{\textbf{low}} & \textcolor{red}{\textbf{high}} & \textcolor{green}{\textbf{low}} & \textcolor{green}{\textbf{low}} & \textcolor{blue}{\textbf{medium}} & \textcolor{green}{\textbf{low}} \\ 
   32 & Paracaesio gonzalesi & \textcolor{green}{\textbf{low}} & \textcolor{red}{\textbf{high}} & \textcolor{blue}{\textbf{medium}} & \textcolor{red}{\textbf{high}} & \textcolor{red}{\textbf{high}} & \textcolor{red}{\textbf{high}} \\ 
   33 & Paracaesio xanthura & \textcolor{green}{\textbf{low}} & \textcolor{blue}{\textbf{medium}} & \textcolor{green}{\textbf{low}} & \textcolor{green}{\textbf{low}} & \textcolor{green}{\textbf{low}} & \textcolor{green}{\textbf{low}} \\ 
   34 & Paracaesio kusakarii & \textcolor{blue}{\textbf{medium}} & \textcolor{red}{\textbf{high}} & \textcolor{red}{\textbf{high}} & \textcolor{red}{\textbf{high}} & \textcolor{red}{\textbf{high}} & \textcolor{red}{\textbf{high}} \\ 
   35 & Paracaesio stonei & \textcolor{green}{\textbf{low}} & \textcolor{red}{\textbf{high}} & \textcolor{red}{\textbf{high}} & \textcolor{red}{\textbf{high}} & \textcolor{red}{\textbf{high}} & \textcolor{red}{\textbf{high}} \\ 
   39 & Cephalopholis sonnerati & \textcolor{green}{\textbf{low}} & \textcolor{green}{\textbf{low}} & \textcolor{green}{\textbf{low}} & \textcolor{green}{\textbf{low}} & \textcolor{green}{\textbf{low}} & \textcolor{green}{\textbf{low}} \\ 
   42 & Epinephelus radiatus & \textcolor{green}{\textbf{low}} & \textcolor{red}{\textbf{high}} & \textcolor{green}{\textbf{low}} & \textcolor{green}{\textbf{low}} & \textcolor{green}{\textbf{low}} & \textcolor{green}{\textbf{low}} \\ 
   43 & Epinephelus morrhua & \textcolor{blue}{\textbf{medium}} & \textcolor{red}{\textbf{high}} & \textcolor{blue}{\textbf{medium}} & \textcolor{blue}{\textbf{medium}} & \textcolor{red}{\textbf{high}} & \textcolor{blue}{\textbf{medium}} \\ 
   45 & Epinephelus areolatus & \textcolor{green}{\textbf{low}} & \textcolor{green}{\textbf{low}} & \textcolor{green}{\textbf{low}} & \textcolor{green}{\textbf{low}} & \textcolor{blue}{\textbf{medium}} & \textcolor{green}{\textbf{low}} \\ 
   48 & Epinephelus bilobatus & \textcolor{green}{\textbf{low}} & \textcolor{green}{\textbf{low}} & \textcolor{green}{\textbf{low}} & \textcolor{green}{\textbf{low}} & \textcolor{green}{\textbf{low}} & \textcolor{green}{\textbf{low}} \\ 
   50 & Epinephelus coioides & \textcolor{green}{\textbf{low}} & \textcolor{blue}{\textbf{medium}} & \textcolor{green}{\textbf{low}} & \textcolor{green}{\textbf{low}} & \textcolor{green}{\textbf{low}} & \textcolor{green}{\textbf{low}} \\ 
   51 & Epinephelus chlorostigma & \textcolor{green}{\textbf{low}} & \textcolor{green}{\textbf{low}} & \textcolor{green}{\textbf{low}} & \textcolor{green}{\textbf{low}} & \textcolor{blue}{\textbf{medium}} & \textcolor{green}{\textbf{low}} \\ 
   54 & Epinephelus stictus & \textcolor{green}{\textbf{low}} & \textcolor{green}{\textbf{low}} & \textcolor{green}{\textbf{low}} & \textcolor{green}{\textbf{low}} & \textcolor{green}{\textbf{low}} & \textcolor{green}{\textbf{low}} \\ 
   57 & Epinephelus undulosus & \textcolor{green}{\textbf{low}} & \textcolor{green}{\textbf{low}} & \textcolor{green}{\textbf{low}} & \textcolor{green}{\textbf{low}} & \textcolor{green}{\textbf{low}} & \textcolor{green}{\textbf{low}} \\ 
   58 & Epinephelus amblycephalus & \textcolor{green}{\textbf{low}} & \textcolor{green}{\textbf{low}} & \textcolor{green}{\textbf{low}} & \textcolor{green}{\textbf{low}} & \textcolor{green}{\textbf{low}} & \textcolor{green}{\textbf{low}} \\ 
   62 & Variola albimarginata & \textcolor{green}{\textbf{low}} & \textcolor{green}{\textbf{low}} & \textcolor{green}{\textbf{low}} & \textcolor{green}{\textbf{low}} & \textcolor{red}{\textbf{high}} & \textcolor{red}{\textbf{high}} \\ 
   67 & Lethrinus olivaceus & \textcolor{green}{\textbf{low}} & \textcolor{red}{\textbf{high}} & \textcolor{green}{\textbf{low}} & \textcolor{green}{\textbf{low}} & \textcolor{green}{\textbf{low}} & \textcolor{green}{\textbf{low}} \\ 
   68 & Lethrinus amboinensis & \textcolor{green}{\textbf{low}} & \textcolor{blue}{\textbf{medium}} & \textcolor{green}{\textbf{low}} & \textcolor{green}{\textbf{low}} & \textcolor{red}{\textbf{high}} & \textcolor{blue}{\textbf{medium}} \\ 
   70 & Wattsia mossambica & \textcolor{green}{\textbf{low}} & \textcolor{blue}{\textbf{medium}} & \textcolor{green}{\textbf{low}} & \textcolor{green}{\textbf{low}} & \textcolor{blue}{\textbf{medium}} & \textcolor{green}{\textbf{low}} \\ 
   71 & Gymnocranius grandoculis & \textcolor{green}{\textbf{low}} & \textcolor{red}{\textbf{high}} & \textcolor{green}{\textbf{low}} & \textcolor{green}{\textbf{low}} & \textcolor{green}{\textbf{low}} & \textcolor{green}{\textbf{low}} \\ 
   76 & Carangoides chrysophrys & \textcolor{green}{\textbf{low}} & \textcolor{green}{\textbf{low}} & \textcolor{green}{\textbf{low}} & \textcolor{green}{\textbf{low}} & \textcolor{red}{\textbf{high}} & \textcolor{blue}{\textbf{medium}} \\ 
   77 & Carangoides gymnostethus & \textcolor{green}{\textbf{low}} & \textcolor{blue}{\textbf{medium}} & \textcolor{green}{\textbf{low}} & \textcolor{green}{\textbf{low}} & \textcolor{green}{\textbf{low}} & \textcolor{green}{\textbf{low}} \\ 
   79 & Caranx ignobilis & \textcolor{green}{\textbf{low}} & \textcolor{red}{\textbf{high}} & \textcolor{red}{\textbf{high}} & \textcolor{red}{\textbf{high}} & \textcolor{red}{\textbf{high}} & \textcolor{red}{\textbf{high}} \\ 
   81 & Caranx sexfasciatus & \textcolor{green}{\textbf{low}} & \textcolor{green}{\textbf{low}} & \textcolor{green}{\textbf{low}} & \textcolor{green}{\textbf{low}} & \textcolor{green}{\textbf{low}} & \textcolor{green}{\textbf{low}} \\ 
   82 & Caranx tille & \textcolor{green}{\textbf{low}} & \textcolor{green}{\textbf{low}} & \textcolor{green}{\textbf{low}} & \textcolor{green}{\textbf{low}} & \textcolor{red}{\textbf{high}} & \textcolor{green}{\textbf{low}} \\ 
   83 & Elagatis bipinnulata & \textcolor{green}{\textbf{low}} & \textcolor{green}{\textbf{low}} & \textcolor{green}{\textbf{low}} & \textcolor{green}{\textbf{low}} & \textcolor{green}{\textbf{low}} & \textcolor{green}{\textbf{low}} \\ 
   85 & Seriola rivoliana & \textcolor{red}{\textbf{high}} & \textcolor{blue}{\textbf{medium}} & \textcolor{red}{\textbf{high}} & \textcolor{red}{\textbf{high}} & \textcolor{red}{\textbf{high}} & \textcolor{red}{\textbf{high}} \\ 
   86 & Erythrocles schlegelii & \textcolor{green}{\textbf{low}} & \textcolor{green}{\textbf{low}} & \textcolor{green}{\textbf{low}} & \textcolor{green}{\textbf{low}} & \textcolor{green}{\textbf{low}} & \textcolor{green}{\textbf{low}} \\ 
   88 & Dentex carpenteri & \textcolor{green}{\textbf{low}} & \textcolor{green}{\textbf{low}} & \textcolor{green}{\textbf{low}} & \textcolor{green}{\textbf{low}} & \textcolor{blue}{\textbf{medium}} & \textcolor{green}{\textbf{low}} \\ 
   89 & Glaucosoma buergeri & \textcolor{green}{\textbf{low}} & \textcolor{blue}{\textbf{medium}} & \textcolor{green}{\textbf{low}} & \textcolor{green}{\textbf{low}} & \textcolor{green}{\textbf{low}} & \textcolor{green}{\textbf{low}} \\ 
   91 & Diagramma pictum & \textcolor{green}{\textbf{low}} & \textcolor{blue}{\textbf{medium}} & \textcolor{green}{\textbf{low}} & \textcolor{green}{\textbf{low}} & \textcolor{blue}{\textbf{medium}} & \textcolor{green}{\textbf{low}} \\ 
   93 & Cookeolus japonicus & \textcolor{green}{\textbf{low}} & \textcolor{blue}{\textbf{medium}} & \textcolor{green}{\textbf{low}} & \textcolor{green}{\textbf{low}} & \textcolor{green}{\textbf{low}} & \textcolor{green}{\textbf{low}} \\ 
   \hline
\end{tabular}
}
\end{table}

\clearpage
\newpage

\setlength{\tabcolsep}{5pt}

% latex table generated in R 3.2.2 by xtable 1.7-4 package
% Wed Apr 19 00:14:30 2017
\begin{table}[ht]
\centering
\caption{Trends in Relative Abundance by Size Group and Species Over Recent Years} 
{\small
\begin{tabular}{cccccc}
  \hline
\#ID & Species & \% Immature & \% Large Mature & \% Mega Spawner & \% SPR \\ 
  \hline
  1 & Aphareus rutilans & \textcolor{red}{\textbf{deteriorating}} & \textcolor{blue}{\textbf{stable}} & \textcolor{blue}{\textbf{stable}} & \textcolor{green}{\textbf{improving}} \\ 
    2 & Aprion virescens & \textcolor{green}{\textbf{improving}} & \textcolor{red}{\textbf{deteriorating}} & \textcolor{green}{\textbf{improving}} & \textcolor{red}{\textbf{deteriorating}} \\ 
    4 & Etelis sp. & \textcolor{green}{\textbf{improving}} & \textcolor{red}{\textbf{deteriorating}} & \textcolor{blue}{\textbf{stable}} & \textcolor{blue}{\textbf{stable}} \\ 
    5 & Etelis radiosus & \textcolor{blue}{\textbf{stable}} & \textcolor{green}{\textbf{improving}} & \textcolor{red}{\textbf{deteriorating}} & \textcolor{green}{\textbf{improving}} \\ 
    6 & Etelis coruscans & \textcolor{red}{\textbf{deteriorating}} & \textcolor{blue}{\textbf{stable}} & \textcolor{blue}{\textbf{stable}} & \textcolor{green}{\textbf{improving}} \\ 
    7 & Pristipomoides multidens & \textcolor{green}{\textbf{improving}} & \textcolor{green}{\textbf{improving}} & \textcolor{green}{\textbf{improving}} & \textcolor{green}{\textbf{improving}} \\ 
    8 & Pristipomoides typus & \textcolor{green}{\textbf{improving}} & \textcolor{green}{\textbf{improving}} & \textcolor{green}{\textbf{improving}} & \textcolor{green}{\textbf{improving}} \\ 
    9 & Pristipomoides filamentosus & \textcolor{green}{\textbf{improving}} & \textcolor{green}{\textbf{improving}} & \textcolor{green}{\textbf{improving}} & \textcolor{green}{\textbf{improving}} \\ 
   10 & Pristipomoides sieboldii & \textcolor{green}{\textbf{improving}} & \textcolor{red}{\textbf{deteriorating}} & \textcolor{red}{\textbf{deteriorating}} & \textcolor{red}{\textbf{deteriorating}} \\ 
   15 & Lutjanus argentimaculatus & \textcolor{red}{\textbf{deteriorating}} & \textcolor{red}{\textbf{deteriorating}} & \textcolor{red}{\textbf{deteriorating}} & \textcolor{blue}{\textbf{stable}} \\ 
   16 & Lutjanus bohar & \textcolor{blue}{\textbf{stable}} & \textcolor{green}{\textbf{improving}} & \textcolor{green}{\textbf{improving}} & \textcolor{red}{\textbf{deteriorating}} \\ 
   17 & Lutjanus malabaricus & \textcolor{blue}{\textbf{stable}} & \textcolor{red}{\textbf{deteriorating}} & \textcolor{red}{\textbf{deteriorating}} & \textcolor{red}{\textbf{deteriorating}} \\ 
   18 & Lutjanus sebae & \textcolor{red}{\textbf{deteriorating}} & \textcolor{red}{\textbf{deteriorating}} & \textcolor{red}{\textbf{deteriorating}} & \textcolor{red}{\textbf{deteriorating}} \\ 
   19 & Lutjanus timorensis & \textcolor{green}{\textbf{improving}} & \textcolor{green}{\textbf{improving}} & \textcolor{green}{\textbf{improving}} & \textcolor{green}{\textbf{improving}} \\ 
   20 & Lutjanus gibbus & \textcolor{green}{\textbf{improving}} & \textcolor{red}{\textbf{deteriorating}} & \textcolor{green}{\textbf{improving}} & \textcolor{red}{\textbf{deteriorating}} \\ 
   21 & Lutjanus erythropterus & \textcolor{blue}{\textbf{stable}} & \textcolor{blue}{\textbf{stable}} & \textcolor{green}{\textbf{improving}} & \textcolor{green}{\textbf{improving}} \\ 
   22 & Pinjalo lewisi & \textcolor{blue}{\textbf{stable}} & \textcolor{green}{\textbf{improving}} & \textcolor{green}{\textbf{improving}} & \textcolor{blue}{\textbf{stable}} \\ 
   27 & Lutjanus vitta & \textcolor{blue}{\textbf{stable}} & \textcolor{blue}{\textbf{stable}} & \textcolor{blue}{\textbf{stable}} & \textcolor{blue}{\textbf{stable}} \\ 
   31 & Symphorus nematophorus & \textcolor{blue}{\textbf{stable}} & \textcolor{red}{\textbf{deteriorating}} & \textcolor{red}{\textbf{deteriorating}} & \textcolor{red}{\textbf{deteriorating}} \\ 
   32 & Paracaesio gonzalesi & \textcolor{green}{\textbf{improving}} & \textcolor{green}{\textbf{improving}} & \textcolor{green}{\textbf{improving}} & \textcolor{red}{\textbf{deteriorating}} \\ 
   33 & Paracaesio xanthura & \textcolor{blue}{\textbf{stable}} & \textcolor{green}{\textbf{improving}} & \textcolor{green}{\textbf{improving}} & \textcolor{green}{\textbf{improving}} \\ 
   34 & Paracaesio kusakarii & \textcolor{red}{\textbf{deteriorating}} & \textcolor{red}{\textbf{deteriorating}} & \textcolor{blue}{\textbf{stable}} & \textcolor{red}{\textbf{deteriorating}} \\ 
   35 & Paracaesio stonei & \textcolor{green}{\textbf{improving}} & \textcolor{green}{\textbf{improving}} & \textcolor{green}{\textbf{improving}} & \textcolor{green}{\textbf{improving}} \\ 
   39 & Cephalopholis sonnerati & \textcolor{red}{\textbf{deteriorating}} & \textcolor{red}{\textbf{deteriorating}} & \textcolor{red}{\textbf{deteriorating}} & \textcolor{red}{\textbf{deteriorating}} \\ 
   42 & Epinephelus radiatus & \textcolor{green}{\textbf{improving}} & \textcolor{green}{\textbf{improving}} & \textcolor{green}{\textbf{improving}} & \textcolor{red}{\textbf{deteriorating}} \\ 
   43 & Epinephelus morrhua & \textcolor{red}{\textbf{deteriorating}} & \textcolor{red}{\textbf{deteriorating}} & \textcolor{green}{\textbf{improving}} & \textcolor{red}{\textbf{deteriorating}} \\ 
   45 & Epinephelus areolatus & \textcolor{blue}{\textbf{stable}} & \textcolor{red}{\textbf{deteriorating}} & \textcolor{red}{\textbf{deteriorating}} & \textcolor{red}{\textbf{deteriorating}} \\ 
   48 & Epinephelus bilobatus & \textcolor{blue}{\textbf{stable}} & \textcolor{red}{\textbf{deteriorating}} & \textcolor{red}{\textbf{deteriorating}} & \textcolor{red}{\textbf{deteriorating}} \\ 
   50 & Epinephelus coioides & \textcolor{green}{\textbf{improving}} & \textcolor{red}{\textbf{deteriorating}} & \textcolor{red}{\textbf{deteriorating}} & \textcolor{red}{\textbf{deteriorating}} \\ 
   51 & Epinephelus chlorostigma & \textcolor{green}{\textbf{improving}} & \textcolor{blue}{\textbf{stable}} & \textcolor{red}{\textbf{deteriorating}} & \textcolor{green}{\textbf{improving}} \\ 
   54 & Epinephelus stictus & \textcolor{blue}{\textbf{stable}} & \textcolor{green}{\textbf{improving}} & \textcolor{green}{\textbf{improving}} & \textcolor{red}{\textbf{deteriorating}} \\ 
   57 & Epinephelus undulosus & \textcolor{blue}{\textbf{stable}} & \textcolor{green}{\textbf{improving}} & \textcolor{green}{\textbf{improving}} & \textcolor{red}{\textbf{deteriorating}} \\ 
   58 & Epinephelus amblycephalus & \textcolor{green}{\textbf{improving}} & \textcolor{green}{\textbf{improving}} & \textcolor{green}{\textbf{improving}} & \textcolor{red}{\textbf{deteriorating}} \\ 
   62 & Variola albimarginata & \textcolor{blue}{\textbf{stable}} & \textcolor{green}{\textbf{improving}} & \textcolor{green}{\textbf{improving}} & \textcolor{red}{\textbf{deteriorating}} \\ 
   67 & Lethrinus olivaceus & \textcolor{red}{\textbf{deteriorating}} & \textcolor{red}{\textbf{deteriorating}} & \textcolor{red}{\textbf{deteriorating}} & \textcolor{red}{\textbf{deteriorating}} \\ 
   68 & Lethrinus amboinensis & \textcolor{blue}{\textbf{stable}} & \textcolor{green}{\textbf{improving}} & \textcolor{green}{\textbf{improving}} & \textcolor{red}{\textbf{deteriorating}} \\ 
   70 & Wattsia mossambica & \textcolor{green}{\textbf{improving}} & \textcolor{green}{\textbf{improving}} & \textcolor{blue}{\textbf{stable}} & \textcolor{red}{\textbf{deteriorating}} \\ 
   71 & Gymnocranius grandoculis & \textcolor{blue}{\textbf{stable}} & \textcolor{red}{\textbf{deteriorating}} & \textcolor{red}{\textbf{deteriorating}} & \textcolor{red}{\textbf{deteriorating}} \\ 
   76 & Carangoides chrysophrys & \textcolor{green}{\textbf{improving}} & \textcolor{red}{\textbf{deteriorating}} & \textcolor{red}{\textbf{deteriorating}} & \textcolor{red}{\textbf{deteriorating}} \\ 
   77 & Carangoides gymnostethus & \textcolor{green}{\textbf{improving}} & \textcolor{blue}{\textbf{stable}} & \textcolor{red}{\textbf{deteriorating}} & \textcolor{red}{\textbf{deteriorating}} \\ 
   79 & Caranx ignobilis & \textcolor{red}{\textbf{deteriorating}} & \textcolor{red}{\textbf{deteriorating}} & \textcolor{red}{\textbf{deteriorating}} & \textcolor{red}{\textbf{deteriorating}} \\ 
   81 & Caranx sexfasciatus & \textcolor{blue}{\textbf{stable}} & \textcolor{red}{\textbf{deteriorating}} & \textcolor{red}{\textbf{deteriorating}} & \textcolor{red}{\textbf{deteriorating}} \\ 
   82 & Caranx tille & \textcolor{blue}{\textbf{stable}} & \textcolor{red}{\textbf{deteriorating}} & \textcolor{red}{\textbf{deteriorating}} & \textcolor{red}{\textbf{deteriorating}} \\ 
   83 & Elagatis bipinnulata & \textcolor{red}{\textbf{deteriorating}} & \textcolor{red}{\textbf{deteriorating}} & \textcolor{red}{\textbf{deteriorating}} & \textcolor{red}{\textbf{deteriorating}} \\ 
   85 & Seriola rivoliana & \textcolor{green}{\textbf{improving}} & \textcolor{red}{\textbf{deteriorating}} & \textcolor{red}{\textbf{deteriorating}} & \textcolor{red}{\textbf{deteriorating}} \\ 
   86 & Erythrocles schlegelii & \textcolor{blue}{\textbf{stable}} & \textcolor{red}{\textbf{deteriorating}} & \textcolor{green}{\textbf{improving}} & \textcolor{green}{\textbf{improving}} \\ 
   88 & Dentex carpenteri & \textcolor{blue}{\textbf{stable}} & \textcolor{green}{\textbf{improving}} & \textcolor{green}{\textbf{improving}} & \textcolor{red}{\textbf{deteriorating}} \\ 
   89 & Glaucosoma buergeri & \textcolor{green}{\textbf{improving}} & \textcolor{green}{\textbf{improving}} & \textcolor{green}{\textbf{improving}} & \textcolor{green}{\textbf{improving}} \\ 
   91 & Diagramma pictum & \textcolor{blue}{\textbf{stable}} & \textcolor{red}{\textbf{deteriorating}} & \textcolor{red}{\textbf{deteriorating}} & \textcolor{red}{\textbf{deteriorating}} \\ 
   93 & Cookeolus japonicus & \textcolor{green}{\textbf{improving}} & \textcolor{green}{\textbf{improving}} & \textcolor{green}{\textbf{improving}} & \textcolor{red}{\textbf{deteriorating}} \\ 
   \hline
\end{tabular}
}
\end{table}

\clearpage
\newpage


\chapter{Discussion and conclusions}
Bottom long line fishing in WPP 712 and WPP 713 occurs on the shelf areas and tops of slopes, mainly in the area where the Java Sea meets the Makassar Strait. Preferred bottom long line fishing grounds have a relatively flat bottom profile at depths ranging from 50 to 150 meters. Deepwater drop line fishing in this general area occurs mainly in WPP 713, in the Makassar Strait area and into the Bali Sea, Flores Sea and Bay of Bone, on deep slopes at depths between 50 and 500 meters. Bottom long line fishing grounds overlap with those previously heavily fished with bottom trawlers, a practice which is now prohibited throughout Indonesia. It is unclear however how much bottom trawling still continues ``illegally'', especially in the Java Sea, with dragging gear that is simply given different names.

The deep water drop line fishery for snappers, groupers and emperors is a fairly ``clean'' fishery when it comes to the species spectrum in the catch, even though it is much more species-rich then sometimes assumed, also within the ``snapper'' category, which forms the main target group. The bottom long line fishery is characterized by a more substantial by-catch of small sharks, cobia and trevallies, which are currently not preferred by the processors who are buying the target species. By-catch species are usually sun-dried by the crew and sold separately, outside of the catch of snappers, groupers and emperors, which belongs to the owner of the boat and goes to the processors for middle and higher end local and export markets.

Drop line fisheries are characterized by a very low impact on habitat at the fishing grounds, whereas some more (but still very limited) impact from entanglement can be expected from bottom long lines. Nothing near the habitat impact from destructive dragging gears is evident from either one of the two deep hook and line fisheries. However, due to limited available habitat (fishing grounds) and predictable locations of fish concentrations, combined with a very high fishing effort on the best known fishing grounds, as well as the targeting of juveniles, there is a very high potential for overfishing in the deep slope fisheries for snappers groupers and emperors.

Risks of overfishing are high for all the larger snappers which are common on deep slopes in Eastern Indonesia, especially for those species which complete their life cycle in the habitats covered by the fishing grounds and which at the same time are easily caught with drop line and bottom long line gears. The snapper feeding aggregations are at predictable and well known locations and the snappers are therefore among the most vulnerable species in these fisheries. Fishing mortality (from deep slope hook and line fisheries) relative to natural mortality in all major target snapper species seems to be unacceptably high while the catches of these species include large percentages of relatively small and immature specimen. For many species of snappers and several species of emperors and carangids, sizes are consistently targeted and landed well below the size where these fish reach maturity. Large specimen of the major target species are already becoming extremely rare on the main fishing grounds.

Interestingly, the groupers seem to be less vulnerable to the deep slope hook and line fisheries than the snappers are. Impact by the deep slope drop line and longline fisheries on grouper populations is limited compared to the snappers. This may be because most groupers are staying closer to high rugosity bottom habitat, which is avoided by longline vessels due to risk of entanglement, while drop line fishers are targeting schooling snappers that are hovering higher in the water column, above the grouper habitat. Fishing mortality (from deep slope hook and line fisheries) relative to natural mortality in large mature groupers seems to be considerably lower than what we see for the snappers.

Groupers generally mature as females at a size relative to their maximum size which is lower than for snappers. This strategy enables them to reproduce before they are being caught, although fecundity is still relatively low at sizes below the optimum length. Fecundity for the population as a whole peaks at the optimum size for each species, and this is also the size around which sex change from females to males happens in groupers. Separate analysis of all grouper data shows that most groupers have already reached or passed their optimum size (and the size where sex change takes place) when they are caught by the deep slope hook and line fisheries.

For those grouper species which spend all or most of their life cycle in these habitats, the relatively low vulnerability to the deep slope hook and line fisheries is very good news. For other grouper species which spend major parts of their life cycle in shallower habitats, like coral reefs or mangroves or estuaries for example, the reality is that their populations in general are in extremely bad shape due to excessive fishing pressure by small scale fisheries in those shallower habitats. This situation is also evident for a few snapper species such as for example the mangrove jack.

Overall there is a clear scope for some straightforward fisheries improvements supported by relatively uncomplicated fisheries management policies and regulations. Our first recommendation for industry led fisheries improvements is for traders to adjust trading limits (incentives to fishers) species by species (which they are basically doing already) to the length at maturity for each species. For a number of important species the trade limits need adjustments upwards, with government support through regulations on minimum allowable sizes. Many of the deep water snappers are traded at sizes that are too small, and this impairs sustainability. The impact is clearly visible already in landed catches.

Adjustment upwards of trading limits towards the size at first maturity would be a straightforward improvement in these fisheries. By refusing undersized fish in high value supply lines, the market can provide incentives for captains of catcher boats to target larger specimen. The captains can certainly do this by using their day to day experiences, selecting locations, fishing depths, habitat types, hook sizes, etc. Literature data shows habitat separation between size groups in many species, as well as size selectivity of specific hook sizes. Captains know about this from experience.

Market preference for certain (small) size classes (like ``plate size'' and ``golden size'') could potentially be adjusted by awareness campaigns that clarify to the public that such sizes for many species actually represent immature juveniles and targeting these specifically will impair fisheries sustainability. Filleting techniques for larger fish can be adjusted to relatively thin slicing under an angle to produce similar cuts as ``plate size'' fillets, instead of the currently more common cutting of thick ``portions'' from large fillets, which are less preferred in some markets. This could support an increased focus on larger fish by fishing companies, especially if supported by size based policies and regulation like minimum sizes.

Some of the less well known snapper species (such as some of the Paracaesio species) are actually good quality fish that are caught in great quantities, but are under-valued in the trade as they are simply not known by high end buyers and lack the valuable color red. Awareness campaigns (including tasting tests) on the quality of these species could help to support fishing companies obtain better prices for these species and offset with that some of the temporary losses that may occur when undersized fish will be actively avoided.

Besides size selectivity, fishing effort is a very important factor in resulting overall catch and size frequency of the catch. All major target snappers show a rapid decline in numbers above the size where the species becomes most vulnerable to the fisheries. This rapid decline in numbers, as visible in the LFD graphs, indicates a high fishing mortality for the vulnerable size classes. Fishing effort is probably too high to be sustainable and many species seem to be at risk in the deep drop and long line fisheries, judging from a number of indicators as presented in this report. At present these fisheries show clear signs of over-exploitation in WPP 712 and WPP 713.

One urgently needed fisheries management intervention is to cap fishing effort (number of boats) at current level and to start looking at incentives for effort reductions. A reduction of effort will need to be supported and implemented by government to ensure an even playing field among fishing companies. An improved licensing system and an effort control system based on the Indonesia's mandatory Vessel Monitoring System, using more accurate data on Gross Tonnage for all fishing boats, could be used to better manage fishing effort. Continuous monitoring of trends in the various presented indicators will show in which direction these fisheries are heading and what the effects are of any fisheries management measures in future years.

Government policies and regulations are needed and can be formulated to support fishers and traders with the implementation of improvements across the sector. Our recommendations for supporting government policies in relation to the snapper fisheries include:
\parskip=1pt
\begin{itemize}[noitemsep,topsep=0pt,parsep=0pt,partopsep=0pt]
\item Use scientific (Latin) fish names in fisheries management and in trade.
\item Incorporate length-based assessments in management of specific fisheries.
\item Develop species-specific length based regulations for these fisheries.
\item Implement a controlled access management system for regulation of fishing effort on specific fishing grounds.
\item Increase public awareness on unknown species and preferred size classes by species.
\item Incorporate traceability systems in fleet management by fisheries and by fishing ground.
\end{itemize}

\noindent Recommendations for specific regulations may include:
\begin{itemize}[noitemsep,topsep=0pt,parsep=0pt,partopsep=0pt]
\item Make mandatory correct display of scientific name (correct labeling) of all traded fish (besides market name).
\item Adopt legal minimum sizes for specific or even all traded species, at the length at maturity for each species.
\item Make mandatory for each fishing vessel of all sizes to carry a simple GPS tracking device that needs to be functioning at all times. Indonesia already has a mandatory Vessel Monitoring System for vessels larger than 30 GT, so Indonesia could consider expanding this requirement to fishing vessels of smaller sizes.
\item Cap fishing effort in the snapper fisheries at the current level and explore options to reduce effort to more sustainable levels.
\end{itemize}


%\newpage

%\chapter{Next steps}
%\input{/root/R-project/IFishSnapperWPP714_715/Texs/6_nextsteps.tex}

\newpage

\chapter{References}
Ehrhardt, N.M. and Ault, J.S. 1992. Analysis of two length-based mortality models applied to bounded catch length frequencies. Trans. Am. Fish. Soc. 121:115-122.

Froese, R. 2004. Keep it simple: three indicators to deal with overfishing. Fish and Fisheries 5: 86-91.

Froese, R. and Binohlan C. 2000. Empirical relationships to estimate asymptotic length, length at first maturity and length at maximum yield per recruit in fishes, with a simple method to evaluate length frequency data. J. Fish  Biol. 56:758-773.

Froese, R. and D. Pauly, (eds.) 2000. FishBase 2000: concepts, design and data sources. ICLARM, Los Ba�os, Laguna, Philippines. 344 p.

Froese, R., Winker, H., Gascuel, D., Sumaila, U.R. and Pauly, D. 2016. Minimizing the impact of ?shing. Fish and Fisheries DOI: 10.1111/faf.12146.

Fujita, R., Karr, K., Apel, A. and Mateo, I. 2012. Guide to the use of Froese sustainability indicators to assess and manage data-limited fish stocks. Oceans Program, Environmental Defense Fund, Research and Development Team.

Martinez-Andrade F., 2003. A comparison of life histories and ecological aspects among snappers (Pisces: lutjanidae). Dissertation http://etd.lsu.edu/docs/available/etd-1113103-230518/unrestricted/Martinez-Andrade\_dis.pdf

Meester G.A., Ault J.S., Smith S.G., Mehrotra A. 2001. An integrated simulation modeling and operations research approach to spatial management decision making. Sarsia 86:543-558.

Quinn, T.J. and Deriso R.B. 1999. Quantitative Fish Dynamics. New York: Oxford University Press.

Vasilakopoulos, P., O'Neill, F. G. and Marshall, C. T. 2011. Misspent youth: does catching immature fish affect fisheries sustainability? - ICES Journal of Marine Science, 68: 1525-1534.

Wallace, R.K. and Fletcher, K.M. 2001. Understanding Fisheries Management: A Manual for understanding the Federal Fisheries Management Process, Including Analysis of the 1996 Sustainable Fisheries Act. Second Edition. Auburn University and the University of Mississippi. 62 pp.

Zhang, C.I., Kim, S., Gunderson, D., Marasco, R., Lee, J.B., Park, H.W. and Lee, J.H. 2009. An ecosystem-based fisheries assessment approach for Korean fisheries. Fisheries Research 100: 26-41.

\end{document}
