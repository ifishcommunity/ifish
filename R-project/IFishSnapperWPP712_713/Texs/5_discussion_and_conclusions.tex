Bottom long line fishing in WPP 712 and WPP 713 occurs on the shelf areas and tops of slopes, mainly in the area where the Java Sea meets the Makassar Strait. Preferred bottom long line fishing grounds have a relatively flat bottom profile at depths ranging from 50 to 150 meters. Deepwater drop line fishing in this general area occurs mainly in WPP 713, in the Makassar Strait area and into the Bali Sea, Flores Sea and Bay of Bone, on deep slopes at depths between 50 and 500 meters. Bottom long line fishing grounds overlap with those previously heavily fished with bottom trawlers, a practice which is now prohibited throughout Indonesia. It is unclear however how much bottom trawling still continues ``illegally'', especially in the Java Sea, with dragging gear that is simply given different names.

The deep water drop line fishery for snappers, groupers and emperors is a fairly ``clean'' fishery when it comes to the species spectrum in the catch, even though it is much more species-rich then sometimes assumed, also within the ``snapper'' category, which forms the main target group. The bottom long line fishery is characterized by a more substantial by-catch of small sharks, cobia and trevallies, which are currently not preferred by the processors who are buying the target species. By-catch species are usually sun-dried by the crew and sold separately, outside of the catch of snappers, groupers and emperors, which belongs to the owner of the boat and goes to the processors for middle and higher end local and export markets.

Drop line fisheries are characterized by a very low impact on habitat at the fishing grounds, whereas some more (but still very limited) impact from entanglement can be expected from bottom long lines. Nothing near the habitat impact from destructive dragging gears is evident from either one of the two deep hook and line fisheries. However, due to limited available habitat (fishing grounds) and predictable locations of fish concentrations, combined with a very high fishing effort on the best known fishing grounds, as well as the targeting of juveniles, there is a very high potential for overfishing in the deep slope fisheries for snappers groupers and emperors.

Risks of overfishing are high for all the larger snappers which are common on deep slopes in Eastern Indonesia, especially for those species which complete their life cycle in the habitats covered by the fishing grounds and which at the same time are easily caught with drop line and bottom long line gears. The snapper feeding aggregations are at predictable and well known locations and the snappers are therefore among the most vulnerable species in these fisheries. Fishing mortality (from deep slope hook and line fisheries) relative to natural mortality in all major target snapper species seems to be unacceptably high while the catches of these species include large percentages of relatively small and immature specimen. For many species of snappers and several species of emperors and carangids, sizes are consistently targeted and landed well below the size where these fish reach maturity. Large specimen of the major target species are already becoming extremely rare on the main fishing grounds.

Interestingly, the groupers seem to be less vulnerable to the deep slope hook and line fisheries than the snappers are. Impact by the deep slope drop line and longline fisheries on grouper populations is limited compared to the snappers. This may be because most groupers are staying closer to high rugosity bottom habitat, which is avoided by longline vessels due to risk of entanglement, while drop line fishers are targeting schooling snappers that are hovering higher in the water column, above the grouper habitat. Fishing mortality (from deep slope hook and line fisheries) relative to natural mortality in large mature groupers seems to be considerably lower than what we see for the snappers.

Groupers generally mature as females at a size relative to their maximum size which is lower than for snappers. This strategy enables them to reproduce before they are being caught, although fecundity is still relatively low at sizes below the optimum length. Fecundity for the population as a whole peaks at the optimum size for each species, and this is also the size around which sex change from females to males happens in groupers. Separate analysis of all grouper data shows that most groupers have already reached or passed their optimum size (and the size where sex change takes place) when they are caught by the deep slope hook and line fisheries.

For those grouper species which spend all or most of their life cycle in these habitats, the relatively low vulnerability to the deep slope hook and line fisheries is very good news. For other grouper species which spend major parts of their life cycle in shallower habitats, like coral reefs or mangroves or estuaries for example, the reality is that their populations in general are in extremely bad shape due to excessive fishing pressure by small scale fisheries in those shallower habitats. This situation is also evident for a few snapper species such as for example the mangrove jack.

Overall there is a clear scope for some straightforward fisheries improvements supported by relatively uncomplicated fisheries management policies and regulations. Our first recommendation for industry led fisheries improvements is for traders to adjust trading limits (incentives to fishers) species by species (which they are basically doing already) to the length at maturity for each species. For a number of important species the trade limits need adjustments upwards, with government support through regulations on minimum allowable sizes. Many of the deep water snappers are traded at sizes that are too small, and this impairs sustainability. The impact is clearly visible already in landed catches.

Adjustment upwards of trading limits towards the size at first maturity would be a straightforward improvement in these fisheries. By refusing undersized fish in high value supply lines, the market can provide incentives for captains of catcher boats to target larger specimen. The captains can certainly do this by using their day to day experiences, selecting locations, fishing depths, habitat types, hook sizes, etc. Literature data shows habitat separation between size groups in many species, as well as size selectivity of specific hook sizes. Captains know about this from experience.

Market preference for certain (small) size classes (like ``plate size'' and ``golden size'') could potentially be adjusted by awareness campaigns that clarify to the public that such sizes for many species actually represent immature juveniles and targeting these specifically will impair fisheries sustainability. Filleting techniques for larger fish can be adjusted to relatively thin slicing under an angle to produce similar cuts as ``plate size'' fillets, instead of the currently more common cutting of thick ``portions'' from large fillets, which are less preferred in some markets. This could support an increased focus on larger fish by fishing companies, especially if supported by size based policies and regulation like minimum sizes.

Some of the less well known snapper species (such as some of the Paracaesio species) are actually good quality fish that are caught in great quantities, but are under-valued in the trade as they are simply not known by high end buyers and lack the valuable color red. Awareness campaigns (including tasting tests) on the quality of these species could help to support fishing companies obtain better prices for these species and offset with that some of the temporary losses that may occur when undersized fish will be actively avoided.

Besides size selectivity, fishing effort is a very important factor in resulting overall catch and size frequency of the catch. All major target snappers show a rapid decline in numbers above the size where the species becomes most vulnerable to the fisheries. This rapid decline in numbers, as visible in the LFD graphs, indicates a high fishing mortality for the vulnerable size classes. Fishing effort is probably too high to be sustainable and many species seem to be at risk in the deep drop and long line fisheries, judging from a number of indicators as presented in this report. At present these fisheries show clear signs of over-exploitation in WPP 712 and WPP 713.

One urgently needed fisheries management intervention is to cap fishing effort (number of boats) at current level and to start looking at incentives for effort reductions. A reduction of effort will need to be supported and implemented by government to ensure an even playing field among fishing companies. An improved licensing system and an effort control system based on the Indonesia's mandatory Vessel Monitoring System, using more accurate data on Gross Tonnage for all fishing boats, could be used to better manage fishing effort. Continuous monitoring of trends in the various presented indicators will show in which direction these fisheries are heading and what the effects are of any fisheries management measures in future years.

Government policies and regulations are needed and can be formulated to support fishers and traders with the implementation of improvements across the sector. Our recommendations for supporting government policies in relation to the snapper fisheries include:
\parskip=1pt
\begin{itemize}[noitemsep,topsep=0pt,parsep=0pt,partopsep=0pt]
\item Use scientific (Latin) fish names in fisheries management and in trade.
\item Incorporate length-based assessments in management of specific fisheries.
\item Develop species-specific length based regulations for these fisheries.
\item Implement a controlled access management system for regulation of fishing effort on specific fishing grounds.
\item Increase public awareness on unknown species and preferred size classes by species.
\item Incorporate traceability systems in fleet management by fisheries and by fishing ground.
\end{itemize}

\noindent Recommendations for specific regulations may include:
\begin{itemize}[noitemsep,topsep=0pt,parsep=0pt,partopsep=0pt]
\item Make mandatory correct display of scientific name (correct labeling) of all traded fish (besides market name).
\item Adopt legal minimum sizes for specific or even all traded species, at the length at maturity for each species.
\item Make mandatory for each fishing vessel of all sizes to carry a simple GPS tracking device that needs to be functioning at all times. Indonesia already has a mandatory Vessel Monitoring System for vessels larger than 30 GT, so Indonesia could consider expanding this requirement to fishing vessels of smaller sizes.
\item Cap fishing effort in the snapper fisheries at the current level and explore options to reduce effort to more sustainable levels.
\end{itemize}
