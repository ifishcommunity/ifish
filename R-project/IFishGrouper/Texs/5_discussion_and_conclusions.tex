%Peter Mous, Jos Pet, January 25, 2017

Whereas I-Fish generates daily updates of the graphs and conclusions on indicator values by species as presented in this report, it does not provide discussions on overall status and trends in the fisheries. In this chapter, we discuss broad conclusions based on the status on January 25, 2017. Versions of this report generated after January 25, 2017, are based on a mix of data collected before this date as well as data collected thereafter. It is therefore likely that some details presented in this chapter differ somewhat from the more up-to-date values presented in chapter 3.

Overall, the status of deepwater grouper drop line and bottom long line fisheries in central and eastern Indonesia is not as bleak as might be expected for an open-access fishery on valuable species. The impact of these fisheries on groupers in general seems much less than it is on snapper resources which are the main target of these fisheries. Without a doubt, several of the grouper species covered in this report are severely at risk from the impact of other fisheries, in different habitats in Eastern Indonesia, but the impact that can be contributed to the deep slope hook and line fisheries seems limited.

Several of the grouper species that occur in the deep slope fisheries are under severe pressure from shallow-water reef fisheries in our region, mostly by small-scale vessels (less than 5 GT), which are currently unregulated. Recruitment of species that start their life-history on shallow coral reefs to the deeper waters will be severely depressed by intensive shallow-water fisheries. Improvements in fisheries management in shallow water habitats and especially in small scale coastal fisheries, could potentially greatly improve outputs from deeper water fisheries that target groupers among other species.

The indicators for \textit{Plectropomus leopardus}, for example, show a ``low'' to ``medium'' level of risk from our target fisheries, whereas it is common knowledge that this species has suffered severe depletion in our region. The relatively low risk level reported here, indicates the impact from deep slope fisheries only. The few \textit{P. leopardus} who survive the extremely high fishing mortality in shallow waters, and who make it to deeper waters, are subject to only moderate exploitation pressure by the deepwater snapper and grouper fishery. A low risk level as indicated in this report refers to the effect of the deepwater dropline and longline fishery only, and a low risk level does not mean that the population as a whole is in good shape.

Some species covered in this report do indicate reasons for serious concern, including for especially \textit{Epinephelus morrhua} and possibly also \textit{Variola albimarginata}. The important species of \textit{Epinephelus morrhua} (2nd most abundant species of grouper in our target fisheries) seems to be highly vulnerable to the deep slope hook and line fisheries and is being caught in relatively high numbers just around its size of female maturation, mostly before reaching the optimum harvesting length. Fishing mortality is very high from the moment this species enters the catch and very few specimen reach lengths anywhere near "mega spawner" size. As is the case for a number of other species, the current trading limit (the minimum size at which traders start paying premium price) is well below the size of female maturation and far below the optimum size for harvesting. The trading limit clearly needs to be increased to improve the status of this species. In general, low trading limits, below size of female maturation, pose a significant potential threat to the status of a large number of species in the deep slope drop line and long line fisheries.
\clearpage
\newpage
\textit{Variola albimarginata} shows an extremely low level of SPR, indicating that the potential for recovery of this species has been severely diminished. \textit{Epinephelus morrhua} shows medium to high levels of threat for most indicators. A few other species like \textit{Epinephelus areolatus} show medium to high levels of risk for some of the indicators. Most other species of grouper seem to be impacted only at acceptable levels by the deep slope hook and line fisheries.

The highly productive \textit{E. areolatus} is the most abundant grouper species in the deep slope hook and line fisheries, and it is the number 3 most important species in these fisheries overall. However, for \textit{E. areolatus} most indicators show only low levels of risk, while only F/M risk is high. Fishing mortality is very high once this species enters the catch, but it has already reached optimum harvest size at that time. This species is doing so well because it has a small maximum size and matures at 21 cm total length, spawning and even reaching optimum harvest size before it is seriously impacted by the fisheries.

For most groupers in our target fisheries, the percentage of immature fish in the catch appears to be low, even where risk levels for other indicators are medium to high, and even though juveniles are well within the commercial size range (as can be seen from the number of species showing medium to high risk level in respect to the Trade Limit). This may indicate that the deepwater grouper fishery is not targeting these species in areas where juveniles occur. This finding is consistent with the common tendency among many groupers to inhabit shallower water as juveniles, and move to deeper waters as they grow to a larger size. Note that in the same deep slope fishery, the percentage of immature specimen of Eteline snappers, which complete their life cycle in deeper waters, is much higher (see other I-Fish reports on these fisheries).

For most of the grouper species targeted by the deep slope hook and line fisheries all the above means that regulation of fishing effort in combination with industry agreements on a minimum size may suffice to guarantee sustainable harvesting by these fisheries. Other than that, in order to improve the status of the stocks of these species, fisheries management needs to start dealing with the immense problem of over-fishing by small scale fisheries in shallower coastal waters, where juveniles as well as adults of many species are being decimated.

This report on groupers pools all observations from the entire study area of the TNC Indonesia Fisheries Conservation Program, which means that it does not show differences between fishery management areas (WPPs). However, such differences do exist. For example, the I-Fish reports of WPPs 573, 714-715, and 718 show that the percentage of immature fish in the catch of \textit{Epinephelus morrhua} is highest in WPP 573 (19\%), and substantially lower in WPP 714, 715, and 718 (10-11\%). The indicators for WPP 573 and WPP 718 suggest that a relatively high number of species are at higher risk in WPP 573 and WPP 718 compared to WPP 714-715. This may be a consequence of a relatively intense fishery in WPP 573, with many fishing vessels concentrating in Indonesian waters around the Sahul banks and the Arafura Sea.

A more detailed analysis is necessary to disentangle the effects of fishing gear and depth of fishing grounds. This report pools data from longliners and dropliners, where longliners tend to fish somewhat shallower waters. Hence, the size compositions in the catch are affected by the relative contribution of each of these two gears.